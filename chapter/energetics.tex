% !TEX root = ../notes_template.tex
\chapter{Muscle Energetics}\label{chp:energetics}
Updated on \today
\minitoc

All prior chapters include important molecular processes for muscle function that require the use of adenosine tri-phosphate (ATP) to provide required energy. Even the resting state of the muscle fiber is in continuous need for ATP to maintain the membrane potential. This chapter covers the biochemical pathways utilized by muscle fibers to convert fuel (macro-nutrients) into ATP.

\vspace{5mm}

\textbf{Objectives include:}
\begin{enumerate}
    \item Explain the structure and function of energetic components of muscle fibers.
    \item Compare and contrast the energetic system pathways that transform macro nutrient substrates to ATP.
    \item Compare and contrast muscle fiber differentiation (types) based on energetic structures and resultant functional capacities.
    \item  Apply the concepts of mass balance, flow gradients, and energy to the analysis of patient/client problems related to the muscular system.
     \item Evaluate the different energetic pathways used in different activities, and analyze the response to different activities to energetic pathways.
     \item Explain the energetic basis of fatigue.
     \item Explain the energetic basis of ischemia.
\end{enumerate}

\section{Transformation of Substrate Energy to ATP}

ATP promotes three types of cell function: (1) membrane transport, as occurs with the sodium-potassium pump, which transports sodium out of the cell and potassium into the cell; (2) synthesis of chemical compounds throughout the cell; and (3) mechanical work, as occurs with the contraction of muscle fibers or with ciliary and ameboid motion.


\subsection{Aerobic (sustained) Energetics (Internal Respiration)}

Mitochondria Extract Energy From Nutrients (p. 24) The principal substances from which the cells extract energy are oxygen and one or more of the foodstuffs—carbohydrates, fats, and proteins—that react with oxygen. In humans, almost all carbohydrates are converted to glucose by the digestive tract and liver before they reach the cell; similarly, proteins are converted to amino acids, and fats are converted to fatty acids. Inside the cell, these substances react chemically with oxygen under the influence of enzymes that control the rates of reaction and channel the released energy in the proper direction.

Mitochondria Extract Energy From Nutrients (p. 24) The principal substances from which the cells extract energy are oxygen and one or more of the foodstuffs—carbohydrates, fats, and proteins—that react with oxygen. In humans, almost all carbohydrates are converted to glucose by the digestive tract and liver before they reach the cell; similarly, proteins are converted to amino acids, and fats are converted to fatty acids. Inside the cell, these substances react chemically with oxygen under the influence of enzymes that control the rates of reaction and channel the released energy in the proper direction.
Oxidative Reactions Occur Inside the Mitochondria, and Energy Released Is Used to Form ATP ATP is a nucleotide composed of the nitrogenous base adenine, the pentose sugar ribose, and three phosphate radicals. The last two phosphate radicals are connected with the remainder of the molecule by high-energy phosphate bonds, each of which contains about 12,000 calories of energy per mole of ATP under the usual conditions of the body. The high-energy phosphate bonds are labile so they can be split instantly whenever energy is required to promote other cellular reactions. When ATP releases its energy, a phosphoric acid radical is split away, and adenosine diphosphate (ADP) is formed. Energy derived from cell nutrients causes ADP and phosphoric acid to recombine to form new ATP, with the entire process continuing over and over again. Most of the ATP Produced in the Cell Is Formed in Mitochondria After entry into the cells, glucose is subjected to enzymes in the cytoplasm that convert it to pyruvic acid, a process called glycolysis. Less than 5\% of ATP formed in the cell occurs via glycolysis. Pyruvic acid derived from carbohydrates, fatty acids derived from lipids, and amino acids derived from proteins are all eventually converted to the compound acetyl coenzyme A (acetyl-CoA) in the mitochondria matrix. This substance is then acted on by another series of enzymes in a sequence of chemical reactions called the citric acid cycle, or Krebs cycle. In the citric acid cycle, acetyl-CoA is split into hydrogen ions and carbon dioxide. Hydrogen ions are highly reactive and eventually combine with oxygen that has diffused into the mitochondria. This reaction releases a tremendous amount of energy, which is used to convert large amounts of ADP to ATP. This requires large numbers of protein enzymes that are integral parts of the mitochondria. The initial event in ATP formation is removal of an electron from the hydrogen atom, thereby converting it to a hydrogen ion. The terminal event is movement of the hydrogen ion through large globular proteins called ATP synthetase, which protrude through the membranes of the mitochondrial membranous shelves, which themselves protrude into the mitochondrial matrix. ATP synthetase is an enzyme that uses the energy from movement of the hydrogen ions to convert ADP to ATP, and hydrogen ions combine with oxygen to form water. The newly formed ATP is transported out of the mitochondria to all parts of the cell cytoplasm and nucleoplasm, where it is used to energize the functions of the cell. This overall process is called the chemiosmotic mechanism of ATP formation.


\subsection{Glycolytic (variable rate) Energetics}

\subsection{$\Beta$-Oxidation Energetics}

\subsection{Pathways for Protein Energetics}

\subsection{Creatine / ATP (immediate) Energetics}

\section{Motor Unit \& Muscle Fiber Types}

\subsection{Slow Oxidative (S/SO/Type 1}

\subsection{Fast Glycolytic (FF/FG/Type 2a}

\subsection{Fast Oxidative Glycolytic (FR/FOG/Type 2x}


\section{\textit{Clinical Physiology Connections}}

\subsection{Fatigue}

\subsection{Ischemia, Hypoxemia \& Hypoxia}

The oxidative energetic pathways provide most of the ATP for cellular functions and are critically involved in the restoration of the energetic resting state following short term rate increases in ATP production with CP or glycolosis. These pathways require oxygen and macro-nutrient substrate (seconds-minutes). Over longer time periods (hours-days) they require cellular synthesis of enzymes which require ATP and amino-acids (to build proteins). And over longer time periods (days-weeks) they require maintenance of the health of mitochondria and replacing damaged or dead mitochondria. Interruptions in the availability of O2 interrupts the process of ATP production with these pathways and can lead to the accumulation of metabolic waste products that alter the pH of the cell and the extra-cellular fluid. These interruptions form the basis of a large number of relatively common and life-threatening chronic medical conditions such as heart disease, stroke, peripheral vascular disease, COVID-19, pulmonary disease, and hematologic (blood) conditions; as well as sudden onset (acute) conditions such as a heart attack and acute altitude sickness. 
Some of the conditions have a rapid onset and immediately threaten life; others exert their effect gradually. The variation is related to rate of onset of O2 deprivation, the magnitude of O2 deprivation (how much deficit, how many cells), and whether the impact is just in the availability of O2 or whether there is also an impairment in waste product removal. But they can all be analyzed based on an understanding of cellular energetics and the role that ATP plays in cellular fidelity, efficacy and integrity.

\paragraph{Hypoxia}
Hypoxia refers to the situation in which there is not enough O2 getting to cells for them to sustain the oxidative production of ATP. Hypoxia can be caused by a variety of situations. It is a local condition because it depends on local O2 levels and local O2 needs. Local O2 needs are dependent on local metabolism. Cells of the body that have relatively high and constant O2 needs, and are therefore more susceptible to hypoxia, are the heart and brain. While metabolism is always related to cellular activity, these two organs tend to have a higher resting metabolism than other body cells. For example, muscle metabolism can far exceed that of both heart and brain, that is only during periods of high muscle activity which require high levels of ATP production. 
The two primary causes of hypoxia are hypoxemia and ischemia.

\paragraph{Hypoxemia}
Hypoxemia is a specific situation in which the blood isn't carrying adequate oxygen to the body’s tissues. Common causes of hypoxemia include pulmonary and blood conditions (for example, obstructive pulmonary disease and anemia) or environmental conditions (altitude, carbon monoxide). 

Hypoxemia can cause hypoxia. The severity of hypoxia in a cell caused by hypoxemia is dependent on the severity of hypoxemia as well as the metabolic activity of the cell (O2 demand), which fluctuates based on several factors. In someone with mild hypoxemia there may be no hypoxia in cardiac or skeletal muscles. However, with exertion that increases cardiac and skeletal muscle activity and thus need for O2 (O2 demand) there may be hypoxia. In these situations cardiac and skeletal muscle function will be impaired by the lack of O2. The cellular adjustment will be to provide ATP using a higher rate of CP and glycolosis which is not sustainable. The by-products of these activities will decrease the cellular and extra-cellular pH which will further impair the production of ATP. The acidosis and reduced ATP relative to need can impact the ability of the cells to repolarize, reduce the frequency of excitations, reduce the pumping of Ca+ back into the sarcoplasmic reticulum, reduce the rate of myosin head release. The consequences of these changes include lower tension production, spasm and potentially damage to the cell membrane. But typically, if the problem originates with hypoxemia that is adequate for resting levels of O2 demand then simply ceasing the activity will restore balance and not result in damage to the cell membrane.
 
\paragraph{Ischemia}
Ischemia is a specific situation in which  blood supply to cells is reduced. The extent of cellular involvement depends on the extent of the reduction. For example, if an entire artery is impacted than an entire limb, or muscle can be involved. If the reduction occurs in capillaries then the reduction in blood flow is to far fewer cells. Common causes of ischemia are atherosclerosis, arteriosclerosis,\footnotemark\footnotetext{Arteriosclerosis refers to thick and stiff arteries that can restrict blood flow. Atherosclerosis is a type of arteriosclerosis that includes buildup of fats, cholesterol and other substances in and on artery walls (plaque). The subsequent reduction in vessel diameter can limit blood flow, and if a plaque becomes an embolus it can lodge and completely block blood flow (blood clot).} blood clots (arterial thrombosis or embolus, or in the case of pulmonary blood flow venous thrombosis or embolus), blood vessel spasm, and micro-circulatory inflammation (a clinical manifestation of hypoxia itself and seen in COVID-19). 

Ischemia can cause hypoxia. The severity of hypoxia in a cell caused by ischemia is dependent on the severity of ischemia as well as the metabolic activity of the cell (O2 demand). A sudden and complete blockage of blood flow to cells, with no alternative pathways to provide blood flow to the cell, is a serious situation that results in cellular death due to the inability to produce ATP for cell membrane functions (sudden complete hypoxia) and to remove waste products. The combination of these two situations results first in reversible damage to the cell membrane and then to irreverisble damage to the cell membrane. Without the cell membrane the cell has lost its integrity. Less extreme reductions in blood flow can create a wide variety of hypoxic and waste removal situations that allow sustained but reduced function for a cell (resulting in long term problems in cell maintenance), and reduced function of the cells. For example, a limit on how much activity the cardiac or skeletal muscle can perform prior to having hypoxia. It is common to simply refer to such situations as ischemia (which means local blood flow is reduced and O2 demands are high enough to cause hypoxia. 

Given the variable nature of blood flow supply and cellular O2 demand there are two situations that can arise for cardiac muscle cells in particular. Stable ischemia refers to the situation that blood flow is sufficient for resting O2 demand, but not sufficient during elevated O2 demand (increased cardiac muscle activity). The situation is considered stable because simply reducing cardiac activity will reduce cardiac O2 demand and restore balance to allow recovery. Unstable ischemia refers to the situation that blood flow is not sufficient for resting O2 demand. Unstable ischemia is unstable because balance cannot be restored by reducing the cardiac muscle activity back to rest since it is the resting demand that cannot be met. In such situations blood flow must be restored, or cardiac muscle O2 demand must be reduced below resting levels. Restoring blood flow is highly situational and can involve breaking up a blood clot (thrombus) with medications (thrombolytics); or restoring the diameter of blood vessels with an angioplasty or by re routing the blood (by-pass graft. Reducing cardiac muscle O2 demand can be accomplished by lowering blood pressure (blood pressure is the resistance that the cardiac muscle must work). Nitroglycerine is a very powerful and fast dilator of blood vessels that quickly lowers blood pressure and allows the cardiac O2 demand to be lowered below resting values. The hope is this restores balance between O2 supply and O2 demand and the cells to recover before damage.

\paragraph{Summary}
Hypoxia is the basis of the homeostatic imbalances caused by many conditions and diseases. It is based on the cellular requirements for ATP, which are based on the mitochrondrial requirements for O2. Cells can produce ATP without O2, however they cannot sustain the production of ATP without O2. Understanding these mechanisms and that of hypoxia offers a wellspring of conceptual insights for many diseases and pathophysiological conditions that go well beyond the above discussion. Hypoxia caused by hypoxemia or ischemia continues to be topic for several of the upcoming chapters on Muscle Support.



\section{Summary \& Next Steps}




\printbibliography[heading=subbibintoc]