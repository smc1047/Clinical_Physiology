% !TEX root = ../notes_template.tex
\chapter{Ventilation}\label{chp:alveolar_oxygen}
Updated on \today
\minitoc

Oxygen intake and carbon dioxide output are necessary for skeletal muscle energetics. The starting place of the intake of $O_2$ is drawing air into the lungs (ventilation) and the alveoli (alveolar ventilation). Similar, but opposite, the final expiration of $CO_2$ is releasing air out of the alveoli and lungs. This chapter covers alveolar ventilation ($V_A$), ventilation ($V_e$), ventilation/perfusion ($V/Q$) matching, the regulation of respiration through ventilation, and pulmonary insufficiency.  

\vspace{5mm}

\textbf{Objectives include:}
\begin{enumerate}
    \item Know and understand the material in the chapter.
    \item Connect and integrate material from other chapters with this chapter.
    \item Connect and integrate the material from this chapter with other chapters.
\end{enumerate}

\section{Ventilation Overview}

Chapter \ref{chp:blood_oxygen} on Respiration established the need for alveolar ventilation and alveolar circulation. There are three recurring and important preliminary concepts: alveolar ventilation ($V_A$), ventilation ($V_e$) and ventilation / perfusion matching ($V/Q$).

\subsection{Important Preliminary Concept}

\begin{itemize}
    \item Alveolar ventilation ($V_A$) is the movement of air into, and out of, the alveoli. Without continuous $V_A$ there is no alveolar respiration. 
    \item $V_e$ is the movement of air into, and out of, the lungs. $V_e$ is necessary for $V_A$. Without ventilation ($V_e$), there is no $V_A$. 
    \item Pulmonary perfusion is based on circulation ($Q$) and must be primarily distributed to alveoli receiving $V_A$. $V/Q$ matching refers to the matching of $V_A$ and pulmonary $Q$. Both $V_A$ and $Q$ are necessary for alveolar respiration. Without $V_A$ matching with pulmonary $Q$ there is no alveolar respiration.
 \end{itemize}

$V_e$ is the volume of air inspired (breathe in) or expired (breathe out) in a time period. It is most commonly measured in $mL/min$ or $L/min$. As a volume per minute, it is the volume of air per breath times the number of breaths per minute (bpm):
\vspace{3mm}
\begin{equation}
    V_e (mL/min) = V_t (mL) \times RR (bpm)
    \label{Ve}
\end{equation}
\vspace{3mm}

In Equation \ref{Ve}, the abbreviation $V_t$ stands for tidal volume. Tidal volume is the volume of air inspired (or expired) with one breath. The underlying assumption for both $V_e$ and $V_t$ is that the volume of air inspired equals the volume of air expired. While this might not be the case breath by breath, it tends to be the case over a minute (or longer) time periods.

\paragraph{Question:} What would happen if the volume inspired did not equal the volume expired over a long period of time? Try this, breathe more in than out, how long can you do that? Now breathe more out than in, how long can you do that?

\paragraph{} If there is an imbalance between $V_t$ in and $V_t$ out, it tends to be limited to transient periods that involve dynamic changes in $V_t$. For example, when $V_t$ increases in the transient periods from rest to exercise, or during exercise when there are variations in $O_2$ needs.

\paragraph{Not all $V_e$ ends in the alveoli ($V_A$)}, either per breath ($V_t$), or per minute ($V_e$). And, not all $V_A$ is going to be to alveoli with perfusion (receive capillary circulation). Figure \ref{fig:ve_va.jpg} depicts the general concept that $V_e$ is going to be greater than $V_A$; and that $V_A$ is going to be greater than the $V/Q$. Pulmonary regulation is largely based on minimizing the discrepancy between $V_e$, $V_A$ and $V/Q$.

\begin{figure}
    \centering
    \includegraphics[width = 0.5\linewidth]{./figure/ve_va.jpg}
    \caption{Caption}
    \label{fig:ve_va.jpg}
\end{figure}

\paragraph{The importance of this concept} is that the alveolar respiration relies on the match between alveolar ventilation ($V_A$) and perfusion ($Q$). The consequences of all pulmonary (lung), and many cardiovascular, conditions ultimately rest on how they impact the ability to ventilate ($V_e$), the ability to turn as much $V_e$ into alveolar ventilation ($V_A$) as possible, or the ability to have as much $V_A$ perfusion as possible.

\subsection{Dead Space Ventilation ($V_D$}
Dead space ventilation that does not reach the alveoli (anatomical dead space), or that reaches the alveoli but does not receive perfusion (alveolar dead space). The combination of anatomical and alveolar dead space is called physiological dead space ($V_D$. To understand anatomical dead space ($VD_{ana}$) it is important to understand the structure and function of the airways.

\section{Airways}

The pathway of $O_2$ from the environment to the alveoli (or alveoli to the environment when considering $CO_2$) involves passage through a set of airways. It is important to recognize that only the alveoli are capable of gas exchange (diffusion, respiration). The difference between $V_e$ and $V_A$ starts with the airways.  

Figure \ref{fig:airways} includes the airways connecting the environment to the alveoli.

\begin{figure}[!h]
    \centering
    \includegraphics[width=1.0\linewidth]{./figure/airways.png}
    \caption{Passage to and from Alveoli}
    \label{fig:airways}
\end{figure}

\subsection{Anatomical Dead Space ($VD_{ana}$)}

Any air that enters the airways is part of $V_e$, but it is not part of $V_A$ unless it reaches an alveoli. For example, at the end of an inspiration the air remaining in the upper airways (nasal passages, the back of the throat, trachea) is part of $V_e$, but not part of $V_A$. The volume of air that is taken into the airways that does not reach the alveoli is referred to as anatomical dead space ($VD_{ana}$) and represents the conduction (non respiration) zone of the airways. $VD_{ana}$ is considered a fixed volume ($mL$). For example, if $VD_{ana} = 100 mL$, and 400 $mL$ is inspired with tidal volume ($V_T$), then 300 ($mL$) reaches alveoli ($V_A$). 

\paragraph{Question:}

Does $V_A (mL/min)$ change if $V_e (mL/min)$ stays the same but is achieved with different volume ($mL$) per breath at different breathing frequencies (respiratory rates)? Answer this by figuring out the $V_A (mL/min)$ and $V_e (mL/min)$ if $VD_{ana} = 100 mL$ when breathing 24 breaths per minute at 250 $mL/breath$; as compared to breathing 12 breathes per minutes at 500 $mL/breath$; as compared to breathing 6 breaths per minute at 1000 $mL/breath$. With each breathing pattern, $VD_{ana} = 100 mL$ since this is a fixed volume.

\subsection{Upper Airways}

The upper airways include the nasal passageways, the nasopharynx, the pharynx. Up to this point (the pharynx) the oral passages are also an option (but suboptimal) for breathing. The nasopharynx passes into the oropharynx which passes into the hypopharynx (laryngopharynx). At this point the epiglottis is the gatekeeper between the larynx (for air) and the esophagus (for food and drink).  The larynx includes the vocal cords and then continues into the trachea (beginning of the lower airways). 

All airway passages down as far as the terminal bronchioles are lined with epithelial cells that include, in various locations, mucus production cells and cilia. Cilia are small cellular extensions that promote the movement of substances such as mucus across the membrane surface. Cilia beat continually in a direction that moves mucus. The movement is toward the pharynx. Mucus and cilia in the airways are combined as a defense mechanism referred to as mucociliary activity. Together they promote the trapping of particles in mucus (including pathogens) and moving the mucus for removal. Cilia in the lower airways (trachea and bronchi) beat upward, whereas those in the upper airways beat downward. Continual beating moves the mucus toward the pharynx. The mucus and its trapped particles are then swallowed or coughed out of the body.

\subsubsection{Bronchopulmonary Hygiene \& Airway Clearance}
Mucocilary activity is an important component of what is called bronchopulmonary hygiene (BPH), while sneezing and coughing are called airway clearance. Techniques to facilitate airway clearance are airway clearance techniques (ACT). BPH and ACT include several manual techniques within the physical therapy scope of practice.\footnotemark\footnotetext{Please don't tell me that "respiratory therapy" does this, of course they do. But that does not negate the fact that this is part of the PT scope of practice. If PTs did not do the things in their scope of practice that other professionals do, physical therapists would not do anything. Saying "respiratory therapy" does this - is like saying "massage therapists" do this about soft tissue mobilization; or that "chiropractors" do this about joint mobs/manipulations; or that "athletic trainers" or "OTs" do this about rehabilitation; or that "exercise physiologists" or "personal trainers" or "strength and conditioning specialists" do this about exercise prescriptions. Of course that is all true. Doing one thing is NOT what makes any of those professionals the professionals that they are, it is doing those things in context and the variety of things being done that makes each of these professions, including physical therapy. The PT uses BPH-ACT in in the plan of care for patients receiving PT, which will undoubtably include many other interventions based on client needs.} Just as increasing tissue extensibility through manual techniques may facilitate the range of motion necessary to move during therapeutic exercise to then maintain tissue extensibility; improving ventilation and respiration through manual techniques may facilitate the respiration necessary to move during exercise to then maintain respiration.  

\subsubsection{Nasal Passageways} 
Nasal passageways are lined with epithelial cells rich in olfactory sensors (sense of smell) and goblet cells which are responsible for creating the mucus membrane. The passages include a set of turbinates that create turbulent flow of air through the passageways which is important for immunity, humidification, adding nitric oxide and the sense of smell. The turbulent flow of air encourages interaction between the air entering the the nasal passages and the cellular epithelial mucus membrane. Large particles (dust, many pathogens, allergens) never move past becoming stuck in the mucus of the nasal passage membranes, serving an important immune and protective function. Interaction with the nasal passage membrane also humidifies the air (adds water) and adds nitric oxide (NO). NO enhances local defense mechanisms via direct inhibition of pathogen growth and stimulation of mucociliary activity. NO from the nose and sinuses is added to the air with every breath taken through the nose. NO reaches the alveoli in a more diluted form and enhances alveolar respiration by local vasodilation of pulmonary capillaries \cite{tornberg_nasal_2002, lundberg_nitric_2008}.

\subsubsection{Lower Airways}
The trachea is the airway that bridges the upper airways to the many airway divisions that will terminate in alveoli. The trachea is supported by cartilage rings that keep it open during wide fluctuations in thoracic pressures (coughing, sneezing, valsalva manuever). The trachea divides at the carina into two main bronchi (right and left) for the right and left lungs (respectively). 

The main, and larger bronchi, also contain cartilage rings for support along with smooth muscle. Each bronchi splits into two branches between 20-25 generations until ending with alveoli \cite{hall_guyton_2020}. The final branch of the bronchi is a bronchiole and is sometimes referred to as the respiratory bronchiole. The lower airways are also lined with epithelial cells that produce mucus and have cilia, and thus also have mucociliary activity that contributes to bronchopulmonary hygiene (BPH) and airway clearance. 

Similar to the artery to arteriole tree, each branch of the the bronchi to bronchiole tree gets progressively smaller, while the sum of space gets larger; and the influence of smooth muscle on the radius gets more important. For example, the main bronchi is larger than any single branch, but the branches, summed together, offer far more space than the main bronchi. As the branches get smaller with each generation of splitting the walls get thinner and the influence of smooth muscle on the walls gets more influential. Most bronchodilation and bronchoconstriction occurs in the smaller bronchioles than in the larger bronchi.

Under normal conditions resistance to air flow is highest in the larger bronchi because there are relatively few bronchi compared to bronchioles. But with various pulmonary conditions the smaller bronchioles can play a greater role in determining air flow resistance for two reasons. First, they are easily occluded because of their small size. Second, they are easily constricted because they have a greater proportion of smooth muscle fibers in their walls in relation to their radius.

The factors impacting air flow, like blood flow, through a cylinder such as an airway (or blood vessel) can be considered with Pouiselle's Law:
\vspace{3mm}
\begin{equation}
    Flow = \frac{\pi \times \Delta P \times r^4}{\eta \times L}
\end{equation}
\vspace{3mm}
And similar to blood flow, the resistance to airflow is largely determined, and manipulated, by changes to the radius through bronchodilation and bronchoconstriction.


\paragraph{Bronchodilation \& bronchoconstriction} are regulated by sympathetic nervous system innervation and endocrine function (circulating epinephrine and norepinephrine). Control of the bronchioles by sympathetic nerve fibers is relatively weak because few nerve fibers penetrate deeply and reach the bronchioles. However, all bronchi and bronchioles are exposed to circulating norepinephrine and epinephrine released from the adrenal gland medulla. These hormones, and in particular epinephrine because of its greater stimulation of β-adrenergic receptors, dilate bronchi and bronchioles. A few parasympathetic nerve fibers, from the vagus nerve, reach the bronchi and bronchioles. These nerves secrete acetylcholine, which constrict the bronchi and bronchioles. 
Conditions that create bronchoconstriction, such as asthma, are therefore commonly treated with adrenergic agonist inhalers (inhaled medication), sometimes in combination with cholinergic anatogonist inhalers or oral medications.  

\section{Volumes \& Capacities}

Ventilatory volumes and capacities are depicted in Figure \ref{fig:ventilation_volumes}. Capacities are combinations of volumes.

\begin{figure}[!h]
    \centering
    \includegraphics[width=1.0 \linewidth]{./figure/ventilation_volumes.png}
    \caption{Ventilatory Volumes \& Capacities}
    \label{fig:ventilation_volumes}
\end{figure}

\begin{itemize}
    \item Tidal Volume ($V_t$): Volume of breathing. $V_T$ varies in order to change ventilation $V_e$, for the purpose of changing $V_A$ to meet the needs of alveolar respiration as it maintains $Pa_CO_2$ and $P_aO_2$; or adjusts $Pa_CO_2$ as part of acid base balance regulation.
    \item Inspiratory Reserve Volume ($IRV$) is the volume that can be inhaled after a normal $V_t$ inspiration
    \item Expiratory Reserve Volume ($ERV$) is the volume that can be exhaled after the end of a normal $V_t$ expiration
    \item Residual Volume ($RV$) is the volume that cannot be exhaled
    \item Vital Capacity ($VC$) is $V_t + ERV + IRV$, it is the largest volume that can be breathed in after a complete exhalation, or out after a complete inspiration
    \item Functional Residual Capacity ($FRC$) is $ERV + RV$, it is the volume of air remaining in the lungs after a normal expiration
    \item Total Lung Capacity ($TLC$) is $IRV + V_t + ERV + RV$, it total capacity for air in the lungs
\end{itemize}

\paragraph{The Musculoskeletal Side of Ventilation:} Volumes \& capacities are measures the amount of air that can be ventilated (or that cannot be ventilated in the case of $RV$ which cannot be expired). They are used to better understand the functional capabilities of the lungs and describe changes to the lung in various states of disease. For example, with the obstructive pulmonary diseases the $RV$ tends to be higher, which reduces $VC$ (and all other values, except $FRC$). The volumes \& capacities also represent a functional measure of the thoracic cavity musculoskeletal range of motion (chest wall mobility) which includes the diaphragm, muscles of the ribs (intercostal muscles), the mobility of the ribs, and the posture of the spine. For example, in situations with diaphragm weakness, reductions in $VC$ are directly related to an inability of the diaphragm to fully descend against the combined resistance pulmonary elasticity and abdominal contents. In patients with many different neurological conditions there is involvement of ventilatory muscles or spinal posture which greatly influence the ventilatory volume \& capacities that further limit activities and result in pulmonary conditions such as pneumonia.

\paragraph{Volumes \& Capacities as range of motion (ROM) - Chest Wall Excursion (CWE):} Occasionally the volumes and capacities are utilized as an indication of the chest wall range of motion (ROM), which is referred to as chest wall excursion (CWE). During testing of the strength of inspiratory muscles patients are asked to expire all their air prior to the test because it is best to test those muscles when they are stretched. Those muscles are stretched when they are in the CWE associated with and at the ROM of, $RV$. When doing a forced expiration, or when training someone to improve their cough, they are asked to take in a full breath, they breath in until they are at the CWE associated with $TLC$. CWE can be measured as a circumference, but not without limitations. One limitation is that the CWE with a tape measure is measuring circumference in one plane (even if it is an oblique plane, it is only one plane). Whereas the change in thoracic cavity size occurs in all three planes (all three dimensions). If CWE is measured with a tape measure, it is usually recommended that it is measured in more than one place (for example, above, at and below the xiphoid process). CWE is also assessed qualitatively simply by placing the hands on various locations of the chest wall and assessing the symmetry, amount and quality of movement.

\paragraph{Homeostatic Regulation with Volitional Control} Ventilation is certainly part of an intricate set of homeostatic regulation mechanisms. Yet it is also under volitional control. The volitional control allows speech (vocalization), and it also allows the opportunity to use breathing to manipulate (hack) homeostatic systems that cannot otherwise be regulated directly (e.g. the use of deep breathing to increase parasympathetic nervous activity and lower heart rate). Physical therapists contribute uniquely to problems related to breathing and ventilation because of how much these processes rely on the neuromuscular mechanics of the breathing muscles. Like posture, gait and many movements, the breathing muscles and patterns of ventilation are mostly automatic yet are able to be voluntarily controlled, and therefore manipulated, adapted and trained.

% Morgan Ford, Ashley Ney, Colin Maberry for the conversation about the benefits of volitional control

\section{Mechanics: The Ventilatory Pump}

The ventilatory pump that creates the pressure for ventilation involves the musculoskeletal system of the thorax (chest wall), including the diaphragm as the floor of the thoracic cavity (and roof of the abdominal cavity, See Figure \ref{fig:diaphragm_lateral}). The diaphragm is positioned to contribute to the variation of pressures between the thoracic and abdominal cavities which makes it uniquely involved in breathing as well as the coordination of breathing with other aspects of dynamic core stability. 

\begin{figure}[!h]
    \centering
    \includegraphics[width=0.5 \linewidth]{./figure/diaphragm_lateral.png}
    \caption{Lateral View of the diaphragm dome and the separation between the thoracic and abdominal cavity \footnotesize{(CCBYSA4.0, WikiMedia Commons)}}
    \label{fig:diaphragm_lateral}
\end{figure}

\paragraph{Inspiration: A Negative Pressure Pump} For breathing air into the lungs the ventilatory pump is a negative pressure pump. Rather than generating positive pressure that pushes air into the lungs, it generates negative pressure that pulls air into the lungs. It does this by expanding the volume of the thoracic cavity, which lowers the pressure below the atmospheric pressure to establish the necessary pressure gradient for air flow. Based on the now, hopefully, familiar equation: $Flow = (P_1 - P_2) / R$; the flow is directly proportional to the pressure gradient created, and inversely proportional to the resistance (and the resistance is largely dependent on the radius of the airways). 

\paragraph{Expiration: A Positive Pressure Pump} For breathing air out of the lungs the ventilatory pump is a positive pressure pump. Positive pressure is generated by contracting the volume of the thoracic cavity, which increases the pressure above the atmospheric pressure to establish the necessary pressure gradient for air flow. Air flow is still directly proportional to the pressure gradient, and inversely proportional to resistance. Resistance is still largely dependent on the radius of the airways. 
One difference with expiration is that the positive pressure, combined with the fact that during expiration the lungs and airways are getting smaller, it is much more likely that resistance to expiratory flow will impact exhalation (than resistance to inspiratory flow will impact inhalation). The practical implications are that with any condition that impacts airway radius, such as asthma, emphysema and chronic bronchitis, the functional impact is on expiratory flow rates more so than inspiratory flow rates. When a limitation is made on expiratory flow rates the term utilized to describe the pattern is obstructive. 
Forced expiratory activities - such as airway clearance with sneezing or coughing - requires a rapid increase in positive thoracic pressure. The difference between a sneeze (besides being typically through the nose) and a cough is closure of the epiglottis to allow for a larger build up of positive pressure in the thoracic cavity and a sudden release of that pressure when the epiglottis opens.  

\paragraph{Pressure Equilibration} Airflow through airways is rapid enough that equilibration can occur at any thoracic cavity volume. Once equilibration occurs, airflow stops. This means that at any thoracic cavity volume, other than $TLC$ or $RV$, changes in volume can cause either inspiration or expiration depending on how the changes in volume change the pressure. At this point it's helpful to remind readers of the Ideal Gas Law (in fact, this needs to be introduced earlier to avoid confusion in future versions....). 

The common form of the Ideal Gas Law is: 

\begin{equation}
    PV = nRT
    \label{ideal_gas}
\end{equation}

Where, $P$ is pressure; $V$ is volume, $n$ is the number of moles in the gas (the amount of gas), $R$ is a gas constant, and $T$ is the temperature in Kelvin. For our purposes we can simplify by removing $R$ and $T$ (although there are slight changes during ventilation from the temperature of atmospheric air and the warmed air at body temperature, these are relatively small on the Kelvin scale, and since we are not actually calculating just considering relationships, we'll be fine).

A simplified version of the Ideal Gas Law, that solves for $P$ is:

\begin{equation}
    P = \frac{n}{V}
\end{equation}

An increase in the thoracic cavity $V$ will reduce the $P$, but air will flow into the thorax through airways until the amount of air in the lungs $n$ equilibrates the equation so that $P$ really does equal $\frac{n}{V}$. Once $P$ in the lungs equals the atmospheric $P$, airflow stops. At that point, if thoracic cavity $V$ can be increased more then $P$ will drop and more air will flow into the lungs until $n$ increases to the point that the equation is once again balanced (equilibrated). This occurs until $P$ in the lungs equals atmospheric $P$. 
The same process can be considered for exhalation. A reduction in thoracic cavity $V$ increases the $P$, air flows out of the lungs until $n$ is reduced to the point of the equation being balanced and the lung $P$ equals the atmospheric $P$. 

\paragraph{Breathing in, Breathing out} is a cyclic variation in the difference between the thoracic (pulmonary) $P$ and the atmospheric $P$, brought on by the cyclic variation in the thoracic (pulmonary) $V$, which is equilibrated by the movement of air into (increase $n$) and out of (decrease $n$) the lungs.

\subsection{$FRC$: The Resting Position}

When considering the mechanics of the ventilatory pump during breathing it is useful to understand the conditions surrounding the functional residual capacity ($FRC$). $FRC$ is the resting position (volume) of the thoracic cavity, and therefore the amount of air in the lungs at this position equals the $FRC$. (That's a nice, intentionally, circular explanation.) 
$FRC$ is the $V$ of the thoracic cavity when there is no active tension being developed by the breathing muscles. The $V$ of the thoracic cavity can still change based on body position, but it is not changing due to active tension of breathing muscles (either inspiratory or expiratory muscles). If you "stop breathing" - and there is no airflow - you are at $FRC$. If you do this in standing and then sit down there may be slightly less $V$ at $FRC$. If you are seated upright and slouch or lean forward then you reduce the $V$ and $FRC$. The $FRC$ is the $V$ in the lungs when there is no active attempt to breath, but it is positionally (supine, prone, sitting, standing) and posturally (spine neutral, flexed, extended) dependent.

\paragraph{$FRC$ is the balance point} between an inward elastic recoil force of the lung tissue, and an outward elastic recoil force of the musculoskeletal thoracic cavity and is influenced by other forces such as gravity (and the impact that gravity has on the thoracic cavity). If the lungs were removed from the body (or detached due to a loss in negative pleural pressure), they would be smaller (less $V$) than at the $FRC$. Without the influence of the lungs, the thoracic cavity would be larger (more $V$) than at the $FRC$. 
The balance of forces point of $FRC$ has a large influence on breathing patterns because it has a large influence on optimal breathing volumes. During resting tidal volume ($V_t$) breathing each expiration ends at $FRC$ and therefore is a passive process. Another way to say this is that a resting $V_t$ expiration ends at $FRC$ so that it can be a passive process. Inspiration requires muscles of inspiration to expand the thoracic cavity $V$, working against the passive recoil force of the lung tissue. But then the passive recoil force of the lung returns the $V$ to $FRC$ which promotes expiration. Therefore, with resting $V_t$ breathing only the muscles of inspiration must be utilized. 
During resting $V_t$ breathing, initial increases in thoracic cavity $V$ are easier to perform because the thoracic cavity is naturally larger than at $FRC$. However, with increases in thoracic cavity $V$ both the lungs and the thoracic cavity create a passive elastic force that the muscles of inspiration must oppose for continued inspiration. This is one of the reasons why during exercise, breathing patterns typically increase $V_t$ by using more of the $ERV$ as compared to the $IRV$.

Forceful expiration requires a larger and more rapid change in $P$, which requires a larger and more rapid change in $V$, which requires active tension in the muscles of expiration. Increasing the rate of breathing $RR$ alone, or along with an increase in $V_t$, requires active tension of the expiratory muscles so that it occurs more completely (volumes less than $FRC$), and more quickly. 

\subsection{Muscles of Inspiration: Increase Thoracic Cavity Volume}

\subsubsection{Diaphragm}

The diaphragm is a membranous muscle and is the most important inspiratory muscle, with motor innervation solely from the phrenic nerves (C3-C5). The diaphragm is extremely active. Muscle fibres within the diaphragm can reduce their length by up to 40\% between residual volume and total lung capacity and spend 35\% of each day contracting, compared with only 14\% for the soleus muscle (Cite Nunn). The origins of the crural part of the diaphragm are the lumbar vertebrae and the arcuate ligaments, while the costal parts arise from the lower ribs and xiphisternum. Both parts are inserted into the central tendon. 
Under normal circumstances, a zone of apposition exists around the outside of the diaphragm where it is in direct contact with the inside of the rib cage, with no lung in between. The parietal pleura continues to allow movement of the diaphragm. When at the upright $FRC$ approximately 55\% of the diaphragm surface area is in the zone of apposition.

There are three primary ways that the diaphragm creates increased thoracic cavity $V$ and these three ways are combined during breathing. Diaphragm mechanics can be considered with the analogy to a piston in a cylinder. With the diaphragm being the piston, and the thoracic cavity being the cylinder. 

\begin{enumerate}
    \item Piston: Downward movement of the diaphragm simply by shortening the zone of apposition around the whole cylinder and leaving the dome shape unchanged. This action very efficiently converts diaphragm muscle shortening into changes in $V$.
    \item Non Piston: Diaphragm tension reduces the curvature and therefore the dome shape. This does not change the zone of apposition but results in an, albeit smaller, increases the $V$.
    \item Piston in an Expanding Cylinder: The diaphragm pulls upward on the lower rib cage resulting in expansion of the lower thorax and therefore thoracic cavity $V$. This contributes to the "bucket handle" movement of the lower rib cage during breathing.
\end{enumerate}

There are many factors that influence how the above three ways that diaphragm mechanics contribute to inspiration. What is clear is that the larger the inspiratory volume, the more diaphragm activity is required and the likelihood that all three mechanical approaches are involved. The two most important factors influencing these mechanics are activity (which influences the amount of air that must be ventilated ($V_e$)); and the posture. For example, in supine the "Piston in an Expanding Cylinder" approach tends to dominate because abdominal mass reduces downward movement of the diaphragm which increases upward pull on the lower ribs.

\subsubsection{Intercostals \& Scalenes}

The exact coordination and involvement of the external and internal intercostals during breathing is complicated and most descriptions (including this one) is a simplification \cite{lumb_nunns_2020, de_troyer_respiratory_2005}. There are several contradictory sources about which set of intercostals are involved with inspiration vs. expiration \cite{lumb_nunns_2020, hall_guyton_2020}. To avoid those contradictions, and since there is little practical value to those arguments, we simply consider the intercostals and do not attempt to assign a particular breathing role to the external or internal sets. We rather consider both the internal and external as contributing to either lifting the ribs (for inspiration) or lowering the ribs (for expiration) from their axes at the costotransverse and costovertebral joints in the thoracic spine. We consider this based on motor coordination patterns.

The inspiration motor coordination pattern for the intercostals starts with the activation of the scalenes. Scalenes are active even during quiet breathing. They stabilize the first rib. The first rib is then the anchor for each rib below it, in succession, to pull upward on the rib below it. They support, or pull up, rib 1. From this base of support the intercostals between rib 1 and 2 stabilize, or pull up, rib 2. From the base of support at rib 2, the intercostals between rib 2 and 4 stabilize, or pull up, rib 3. The pattern continues to rib 12 - the stabilized rib n, allows the intercostals between rib n and n+1 to stabilize, or pull up, rib n+1.

Based on the costotransverse and costovertebral joint axes, and in combination with attachment to the sternum, the upward movement of the upper ribs (1 - 6 or so), is described as "pump handle" which includes a change in the anteriorposterior (AP) diameter of the thoracic cavity. The upward movement of the lower ribs (7ish through 12) is described as "bucket handle" which includes a change in the medial-lateral (ML) diameter of the thoracic cavity.

Even during quiet breathing there is active tension generated in the scalenes and intercostals, even if there is no movement of the ribs. These isometric contributions at the very minimum, prevent inward movement of the ribs due to the drop in thoracic pressure by the piston action of the diaphragm. In premature infants with underdeveloped intercostal muscles, or people with C5 to C8 spinal cord injury that do not have innervation of the intercostal muscles, a thoracic paradox breathing pattern can impair ventilation. This breathing pattern includes a diaphragm piston like activation that reduces thoracic pressure and pulls the ribs inward which reduces the change in thoracic $V$ and limits ventilation. It is thoracic paradox because the thoracic is not doing what is expected during inspiration. Can you figure out what an abdominal paradox breathing pattern is and when it might occur?

\subsubsection{Accessory Inspiratory Muscles}
Accessory muscles are not active during normal breathing, but as ventilation increases there is a greater need to increase the thoracic cavity size, and therefore greater chest wall excursion (CWE). %% here

Increasing the dimensions of the thoracic cavity (chest walland the inspiratory muscles contract more vigorously accessory muscles are recruited. Other reasons for accessory muscle activation may be when there are limitations in the primary muscles or when there are pulmonary conditions making ventilation less efficient. Accessory muscles include the sternocleomastoids, extensors of the vertebral column, pectoralis minor (and to a lesser extent major), trapezius and the serratus anterior and posterior muscles. For many of these muscles to serve an accessory role to inspiration the upper extremity must be stabilized so that tension in the muscle moves the thorax, not the limbs. In people with severe pulmonary disease that rely on accessory muscle activation for breathing, activities that use the upper extremities can be considerably limited.

\paragraph{Spinal Extension} increases thoracic cavity size. Think of the rib cage as an accordion. With spinal extension the accordion opens. Spinal extension, and the spinal extension is therefore an inspiratory accessory movement (and therefore spinal extensors are accessory muscles).


\subsection{Muscles of Expiration: Decrease Thoracic Cavity Volume}

\subsubsection{Intercostals}

The motor coordination pattern for using the intercostals to decrease thoracic cavity volume starts with the abdominal muscles (rectus abdominus, external and internal obliques). The abdominal muscles stabilize the lower rib cage so that intercostal contraction pulls down on the rib above it to reduce the size of the thoracic cavity (cylinder). This is the opposite coordination pattern of inspiration when the scalenes stabilized the first rib. 

\subsubsection{Abdominal Muscles}

The abdominal muscles have three roles in expiration (particularly forced expiration).

\begin{enumerate}
\item Stabilize the lower ribs so that intercostal activation pulls ribs down.
\item Pull the lower ribs down (they pull the lower ribs down much more than the scalenes can pull the first rib up - so the role of the scalenes for inspiration, while important, is really limited to stabilizing that first rib). The pulling down on the lower ribs significantly contributes to reducing the size of the thoracic cavity (the cylinder).
\item Reduce abdominal cavity volume $\rightarrow$ increases abdominal cavity pressure $\rightarrow$ pushes up the relaxed diaphragm $\rightarrow$ reduces thoracic cavity volume and resets the piston (stretches the diaphragm for the next inspiration
\end{enumerate}

\subsubsection{Accessory Expiratory Movement}

As you probably figured out, if spinal extension is an accessory inspiratory movement then spinal flexion should be an accessory inspiratory movement. Of course, the abdominal muscles (collectively) are the spinal flexors and they are primary expiratory muscles, so they are not also accessory expiratory muscles. It is simply the case that if spinal flexion accompanies expiration, either intentionally to get as much air out as possible, or unintentionally due to the force of the abdominal muscles during expiration, then it is an expiratory accessory movement. Note: forced expiration that must include abdominal muscle activation but does not include spinal flexion, must involve spinal extensors to generate an equal, but opposite torque so that the spinal flexor muscles perform the three tasks they perform for expiration, but do not flex the spine.

\subsection{Pulmonary \& Pleural Pressures}
So far the discussion about the mechanics of ventilation have rightly assumed coherence between the lungs and the thoracic cavity. When the thoracic cavity increases volume, the lungs increase volume which increases the volume of the alveoli (and that is the case). And when the thoracic cavity decreases its volume, the lungs decrease volume which decreases the volume of the alveoli (and that is the case). Under all normal circumstances the events above hold true. 

\paragraph{But....} the anatomical fact is that there is no direct connection between the lungs and the inner wall of the thoracic cavity. There is a space, it's a small space and it's filled with fluid, but it's a space. It is the pleural space and it is filled with pleural fluid. The lungs are lined with a lung pleural membrane and the thoracic cavity is lined with a thoracic pleural membrane. Between them is a thin space with pleural fluid. This arrangement allows the lungs to move within the thoracic cavity with relatively little friction (similar to the pericardial space and the pericardial fluid around the heart). 
To keep the lungs inflated and moving along with the thoracic cavity the pleural space is kept at a slightly negative pressure. Negative in comparison to the atmosphere (atmosphere is taken to be 0 cm of water ($cm H_2O$). This is called the pleural pressure. The normal pleural pressure at the beginning of inspiration is about −5 ($cm H_2O$), which is the amount of suction required to hold the lungs at their resting volume. During inspiration the thoracic cavity expands and the pleural pressure drops to approximately -7.5 $cm H_2O$. These values are being shared simply to convey that they are slightly negative. 

During regular (resting $V_t$) expiration the lungs recoil and pull the thoracic cavity thus maintaining the negative pressure of the pleural space. Due to the movement of the thoracic cavity along with the lungs the pleural pressure returns to -5 $cm H_2O$.

During forced expiration the pleural pressures become slightly positive as the thoracic cavity pushes on the lungs to reduce their volume. The positive pleural pressure can result in a greater degree volume loss in alveoli, and reduction in the size of bronchioles and bronchi than due to the reduction in the volume of the thoracic cavity. In other words, during forced expiration the thoracic cavity directly reduces the lung size and this reduction is further influenced by a positive pleural pressure that also decreases lung airways and alveoli volume. This effect is more pronounced with a greater force for forced expiration because there is a higher positive pleural pressure.

\paragraph{Pneumothorax} is the collapse of a lung which can be precipitated by a loss of negative pleural pressure due to trauma. In certain trauma scenarios air may escape the lung and enter the pleural cavity resulting in positive pleural pressure which collapses the lung. This situation is called a tension pneumothorax and requires the positive pleural pressure to be removed with a tube placed into the chest (chest tube). Chest tubes are commonly used after thoracic surgery (any surgery that opens the thoracic cavity) to prevent the accumulation of edema and also to maintain negative pressure via suction to keep the lungs inflated.

\paragraph{Pleural Effusion} is the accumulation of fluid in the pleural space which then compresses the lung.

\subsubsection{Lung Compliance} 

Lung Compliance is the change in lung volume for each unit of change in transpulmonary pressure. Transpulmonary pressure is the difference between the alveolar and pleural pressures. The normal total compliance of both lungs together in a healthy adult is about 200 $mL/cm H_2O$ (for each 1  $cm H_2O$ change in pressure there would be a 200 $mL$ change in volume). 
Lung compliance depends on the elastic recoil forces of the lung. The elastic recoil forces are determined by the elastin and collagen fibers and the surface tension of alveoli. Increased elastic recoil forces (including increased surface tension) decreases compliance. For example, if there is more elastic recoil the compliance may be 100 $mL/cm H_2O$ which means only 100 $mL$ of volume for a 1 $cm H_2O$ change in pressure).

\paragraph{Surface tension of the alveoli} is the tendency for the $H_2O$ molecules on the surface to attract each other and contract (reduce the volume) of the alveoli. This force is strong enough that it can result in the collapse of the alveoli (collapse of alveoli is called atelectasis).

\paragraph{Surfactant} is secreted by alveolar epithelial cells and reduces the surface tension in the alveoli by to $\frac{1}{12}$. This substantial reduction in surface tension helps to maintain normal lung compliance, and to keep alveoli open. Since there is still some surface tension alveoli have a tendency to get smaller over time (hours). As alveoli get smaller there is less surface area which leads to increased surface tension because surfactant is pushed out of the alveoli. This reduces compliance and increases the risk of collapse (atelectasis). A protective mechanism against this is an occasional deep breath (cycles in inspiratory volume) open the alveoli which allows surfactant to move back into the higher volume, larger surface area alveoli. Following surgery with general aneasthesia there is a reduced drive to breath which lowers breathing volume and reduces the occasional deep breath from occurring (yawns, sighs, or just from moving around). This places individuals at a higher risk of atelectasis, which causes problems with oxygenation and also can be a precursor to pneumonia. Another effect is that clearing the anaesthia from the body is facilitated by respiration and ventilation, so impaired respiration slows down clearance of the anaesthetic that is impairing the drive to breathe more deeply in the first place. Therefore, it is common for post operative rooms and surgical units to regularly encourage deep breathing as patients recover from surgery with exercises and even biofeedback such as an incentive spirometer (provides visual feedback about how much volume has been inhaled). 

\paragraph{Neonatal Respiratory Distress Syndrome (NRDS)} occurs in premature infants that have not yet started to create and release surfactant. They have very low lung compliance and require mechanical ventilation for breathing. But even the mechanical ventilation is challenging because as the ventilator pushes a sufficient volume of air into the lungs for ventilation the pressure rises substantially, because of the low compliance, which can then cause pressure trauma (baro-trauma) and worsen the overall pulmonary situation. The availability and delivery technologies of artificial surfactant has significantly improved the outcomes in such situations.


\section{Ventilation / Perfusion Matching}

Ventilation serves respiration. This is not the only function of ventilation, but it is certainly a primary function. For ventilation to serve respiration it must result in alveolar ventilation, $V_A$. And most of that fresh $V_A$ must match pulmonary capillary perfusion ($Q$). $V/Q$ matching requires minimizing the situations that result in mismatching. There are two general situations that result in mismatching, dead space ventilation ($VD$) and right-left pulmonary shunt (RL shunt). Minimizing these situations involves the pulmonary circulation regulation discussed in Chapter \ref{chp:blood_oxygen} (Respiration) on the Regulation of Pulmonary Circulation. This is a good time to review that section of the Respiration Chapter if needed.

\subsection{Dead Space Ventilation ($VD$)}

As introduced in the Ventilation Overview section, there are three classifications for dead space: anatomical, alveolar and physiological.  
\vspace{3mm}
To summarize:
\begin{itemize}
       \item Anatomical dead space ventilation ($VD_{ana}$) is air ventilated but that does not make it into alveoli, $VD_{ana} = V_e - V_A$
    \item Alveolar dead space ventilation ($VD_{alv}$) occurs in alveoli that receive ventilation but do not receive alveolar circulation
    \item Physiological dead space ventilation ($VD$) is the sum of anatomical and physiological dead space, $VD = VD_{ana} + VD_{alv}$
\end{itemize}

\paragraph{Anatomical dead space ventilation ($VD_{ana}$)} was discussed in the section on Airways, since it is the volume of air that is ventilated but remains in the conductive zone of the airways (does not make it to the alveoli). $VD_{ana}$ is minimized by regulating, and optimizing, the breathing pattern volume and rate. Larger volumes certainly reduce $VD_{ana}$, but if volumes are too large then the work of breathing becomes more difficult for three reasons:

\begin{enumerate}
    \item Inspiratory muscles shorten and it's more difficult to generate the required active tension based on their length - tension curve.
    \item The thoracic cavity increases and starts to resist further increases due to passive tension so the muscles must work harder (at a time that it is harder for them to work based on #1
    \item Lung compliance starts to decrease because as lungs expand the lung elastic recoil forces (like passive tension) increase (and recall that lung elastic forces are inversely proportional to compliance
\end{enumerate}

For all of those reasons - optimizing breathing pattern requires a balance between higher volumes and lower rates (which minimize $VD_{ana}$ but increase the word of breathing), and lower volumes and higher rates (which maximize $VD_{ana}$, but lower the work of breathing).

\paragraph{Alveolar dead space ventilation ($VD_{alv}$)} is minimized by the regulation of pulmonary circulation. Pulmonary arterioles dilate in areas with increased $O_2$ and constrict in areas with decreased $O_2$. 

Gravity and therefore body position also influence $VD_{alv}$ because gravity influences the distribution ventilation and perfusion. And, the distribution of alveolar residual volume influences perfusion. The lungs can be broken into three zones that are based on gravity.
\vspace{3mm}
The zones are:
\begin{enumerate}
    \item Upper Zone: receives less new $V_A$ because during expiration the alveoli do not decrease in size as much, so there is less overall change in their volume and less overall change in air ventilated; receives less $Q$ because the pulmonary circulation is a lower pressure system so its arterial flow is influenced more by gravity (harder to pump blood up) as well as the residual volume of the alveoli. The Upper Zone has more of the $FRC$ and $RV$. With a higher residual volume of air in the alveoli the pulmonary capillaries are compressed (in addition to the constriction of the arterioles due to lower $O_2$ delivery.
    \item Middle: receives more $V_A$ than the Upper, but less than the Lower Zones; and more $Q$ than the upper, but less than the Lower. There is a lower alveoli residual volume than in the Upper zone, and more than in the Lower zone.
    \item Lower: receives more new $V_A$ because during expiration the alveoli decrease in size substantially so there is more overall change in their volume and more overall ventilation; receives more $Q$ because the pulmonary circulation is influenced by gravity. The Lower Zone has less of the $FRC$ and $RV$. With less residual volume the pulmonary capillaries are not compressed, and in combination with the dilated arterioles (due to the higher $O_2$ values from the increased ventilation) $Q$ is greater than the other zones.
\end{enumerate}
\vspace{3mm}
The combination of pulmonary circulation regulation and pulmonary zones whereby gravity influences $V$ and $Q$ results in normal resting $V/Q$ matching of approximately 80\%. This is an empirical figure based on the use of a ventilation/perfusion lung scan, also called a $V/Q$ lung scan.  The imaging uses medical isotopes to evaluate the circulation of air and blood within the lungs to determine the ventilation/perfusion matching. 

\paragraph{} \textit{Pulmonary Embolism} is a blood clot that travels to and lodges in a pulmonary artery or arteriole that cuts off blood blow to a section of alveoli capillary circulation. This is a life threatening situation that immediately and substantially causes alveolar dead space and thus physiological dead space.


\paragraph{Physiological Dead Space ($VD$}) is the sum of anatomical and alveolar dead space and can be estimated with the Bohr Equation:

\begin{equation}
    VD = V_t \times \frac{(P_aCO_2 - etCO_2)}{P_aCO_2}
    \label{eq:bohr}
 \end{equation}
 
$etCO_2$ is the end tidal partial pressure of $CO_2$ measured at the nose during at the end of expiration.

\paragraph{For example:} If $V_t$ is 500 $mL$, $P_aCO_2$ is 40 $mmHg$, and $etCO_2$ is 35 $mmHg$ then $VD$ is estimated as 62.5 $mL$. The units are in $mL$ because the $V_t$ is in $mL$ and it is multiplied by a ratio which is unitless since the $P_aCO_2$ and the $etCO_2$ are both in $mmHg$.

\paragraph{Question:} Figure out the $VD$ if $V_t$ is 300 $mL$, $P_aCO_2$ is 50 $mmHg$, and $etCO_2$ is 30 $mmHg$. How does this compare with the above example? 

\paragraph{Question:} Figure out the $VD$ if $V_t$ is 300 $mL$, $P_aCO_2$ is 40 $mmHg$, and $etCO_2$ is 15 $mmHg$. How does this compare with the above examples?

\subsection{Right-Left Shunt}

Another situation that can result a $V/Q$ mismatch is a right-left (RL) shunt. When alveolar dead space causes a mismatch it is because ventilated alveoli do not receive any (or enough) perfusion. A RL shunt is the opposite situation. Alveoli that do not have ventilation, or are under ventilated, receive blood flow. Because this blood flow does not exchange gases with the alveoli (because the alveoli does not have ventilation), the right sided blood (venous blood) returns unchanged and joins with the left sided blood (arterial blood). 

The simplest cause of a RL shunt is a breath hold (not breathing, apnea). With apnea the alveoli still have air. However, since the alveoli equilibrate with blood passing through they quickly, without $V_A$, have the same amount of $O_$ and $CO_2$ as venous blood (R). This results in elevated $P_aCO_2$ and reduced $P_aO_2$ that worsen with prolonged breath hold.

Pathological causes of RL shunts include anything that fills the alveoli with something other than ventilated air. For example, pulmonary edema, mucus with pneumonia, inflammation with any pulmonary inflammatory condition such as COVID (which can also cause pneumonia). Chronic pulmonary conditions such as emphysema and bronchitis increase the $RV$ which increase the number of alveoli getting blood flow but not new ventilation (RL shunt).

\subsubsection{$VD$ \& RL Shunt}

There is a bit of an inverse relationship between $VD$ and RL shunt. Across all the alveoli-capillary interaction space between both lungs and assuming ventilation and blood flow are both occurring; then any alveoli getting ventilation and not perfusion ($VD$) means there are other alveoli getting perfusion and not ventilation. An extreme - and purely academic - example would be if all ventilation went to the right lung; and all blood flow went to the left lung. In this situation the overall $V/Q$ matching is 0\%. The right lung is completely dead space ventilation $VD$; and the left lung is a complete RL shunt.


\subsection{A-a Gradient}

The A-a gradient is the drop off that occurs between the partial pressure of $O_2$ from fully ventilated alveoli ($P_AO_2$) to the partial pressure of $O_2$ in mixed arterial blood ($P_aO_2$). The $P_AO_2$, at sea level with a normal fraction of inspired oxygen ($F_iO_2 = 0.21$) is taken as 104 $mmHg$. This blood mixes with blood from other regions of the lung as it travels back to the left atrium. In addition to the venous blood from the bronchial pulmonary circulation, there is blood from normally occurring RL shunts given the normal resting $V/Q$ matching of 80\% (20\% mismatching). Therefore, we should not be surprised that the normal $P_aO_2$ is less than 104 $mmHg$ of $P_AO_2$. This difference (approximately 104 - 95 = 9 mmHg) is the A-a gradient. The width of the A-a grdient (a larger number) is associated with a more severe RL shunt. In critical care situations the A-a gradient can be elevated even when a patient has a normal blood oxygen level because the $P_AO_2$ is elevated. 
For example, if a patient at sea level is on supplemental oxygen with an $F_iO_2 = 0.6$ (60\% oxygen), then the estimated $P_AO_2$ is 300 $mmHg$, and the A-a gradient even with a normal $P_aO_2$ would be 205, indicating substantial RL shunting.

\section{Regulation}
% Left off here.....
Most of our lives breathing is automatically paced by central pattern generators (CPGs) in from the respiratory center of the brain stem. This center receives afferent signals from chemoreceptors, pulmonary stretch sensors and

Regulation of ventilation is automatic 
Automatic but voluntary control - we breath without thinking about it, but we can think about it and control it (like posture and gait).



%The regulation of blood $O_2$ and $CO_2$ includes monitoring these values with chemoreceptors, but the efferent (action) side of these regulatory pathways include variations in ventilation.


\section{Practice Connections}

\subsection{Nose vs. mouth breathing}

\subsection{Pulmonary Function Tests - Forced Expiratory Manuever}

\subsection{Ventilatory Muscle Testing}

\subsection{Ventilatory Insufficiency Patterns}

\printbibliography[heading=subbibintoc]