% !TEX root = ../notes_template.tex
\chapter{Muscle Regulation}\label{chp:regulation}

\minitoc
The regulation of muscle tension is managed by the central nervous system. This chapter introduces the basic mechanisms utilized by the CNS to fulfill this role. The CNS has one approach for muscle fiber tension regulation - it can manipulate the number of twitches per second (frequency summation). It also has one approach for whole muscle tension regulation - it can manipulate the number of muscle fibers twitching (motor unit summation). 

\vspace{5mm}

\textbf{Objectives include:}
\begin{enumerate}
    \item
    \item
    \item
    \item
    \item
\end{enumerate}\

\section{Motor Units}

\subsection{Motor Unit Excitation}

An $\alpha$-motor neuron, once excited, will release only one neurotransmitter, ACh at its axon terminal. ACh will only have one effect at the NMJ, creating excitatory micro-potentials on the motor end plate. However, in the Central Nervous System (such as the Spinal Cord) competing neurotransmitters can be released at synapses of the $\alpha$-motor neuron dendrites. The impact of these neurotransmitters can be excitatory or inhibitory. Since they occur after the synaspse they are called excitatory post synaptic potentiations (EPSPs) or inhibitory post synaptic potentiations (IPSPs). Whether, and how frequently, an $\alpha$-motor neuron sends an excitation to the muscle fibers it innervates is balanced by competition between the EPSPs and the IPSPs.

\section{Frequency Summation}

\section{Motor Unit Summation}



\section{Muscle Fiber Differentiation}

\subsection{Order of Recruitment}

\section{Regulatory Feedback}

\subsection{Muscle Spindles}

\subsection{Golgi Tendon Organs}

\section{\textit{Connections:} Electromyography}

\printbibliography[heading=subbibintoc]