% Book preface - about the project, etc.
\chapter*{Advice to the Reader}
\addcontentsline{toc}{chapter}{Advice to the Reader}

The second draft. Not the second edition. Draft. The work in front of you today is one year old. In early April 2022, nothing you read in this manuscript was available. It represents a years worth of work condensed into a few months of the year. Why? Because to work on this project requires an author  with the challenges I have on focusing to have no other demands on my time. It requires a singular focus. I tend to have a singular focus on clinical physiology from approximately mid May through August every year. That's the time frame during which I can work on this book.

What does it mean to be a second draft? 
I think Vonnegut's title for his notes and comments on writing style sum it up best: "Pity the reader." \cite{vonnegut_pity_2019} But at least it's not a first draft. Reading a first draft "is a lot like inflating a blimp with a bicycle pump. Anybody can do it. All it takes is time." \cite{vonnegut_pity_2019}

\section{Reading to Learn}

What follows are notes on reading and learning for your time in this course in particular, and the DPT program in general. These are just suggestions. Textbooks in general, and your DPT books in particular, tend to be loaded with information. Some of the information you already know, much you do not. Much of the information across various books and within a book overlaps. Some of the information is extraneous, some of it is necessary.\footnotemark{}\footnotetext{I've tried in this second edition to eliminate as much of the extraneous material as possible.} The trick to effective reading as part of your studying relies on your ability to cut through to the necessary core. Readers all come from different backgrounds and have had different experiences. Therefore, some of what is necessary is that for all; some of what is necessary is just for you. Hence, reading is a very personal adventure. Ultimately reading is a conversation between you and the writer. As an avid reader and writer, I read and write as if I were having a conversation with the writer or reader. This \textbf{dialectic} approach tends to work well for helping with studying and learning. Most of the advice on How to Read a Book being provided is from the classic text on this topic by Mortimer Adler and Charles van Doren \cite{adler_how_1942}.

\subsection{From Unconscious Incompetence towards Conscious Competence}

The first transition for learning is moving from unconscious incompetence to conscious incompetence. In other words, reading about things you do not know allows you to take the first step towards learning - you go from not knowing what you do not know (\textit{unconscious incompetence}) towards knowing what you do not know (\textit{conscious incompetence}). It is only from conscious incompetence that you can begin moving towards the next step - \textbf{conscious competence}. 

The tension associated with conscious incompetence is motivating to the active processes necessary to learn. However, that tension can also be overwhelming. It is helpful to have a well understood plan for how to get from conscious incompetence to conscious competence. Helping you with that plan is the job of your teacher. But all your teacher can do is help, it is ultimately something you need to do. The more active you are with the process, the more successful you will be. \cite{brown_make_2014}

The plan proposed in this reading is to understand first that this process is taking place; second, the format that information is being presented; and third and what it is that you need to take away from it. 

You have already accomplished the first part of the plan by being in the program and in classes - you are now aware that process is taking place! It is important to keep it in mind as you read each chapter. 

Your active process should include these components. 

\begin{itemize}
\item Read
\item Take notes on your reading
\item Discuss what you are learning with your classmates
\item Discuss what you are learning with your teacher
\end{itemize}

Keep in mind the last bullet above (\textit{Discuss what you are learning with your teacher}) happens during all and any interactions with your teacher and this includes written communication (most clearly assignments).

Cycling through this active process addresses the second and third part of the plan. That is, you will repeatedly be exposed to new information through reading. You will also repeatedly consider what it is that you need to take away from the information. This is where the notes and discussions are important parts of the plan. You will have to think about that information as you take notes on it and you should focus your notes on what you need to know about it. Discussions are to guide you - individually and as a group - through the process; affirming you are learning what you need to learn and correcting your process as needed.

\section{Reading Tips}

You do not need to read every word. The goal of reading and note-taking are grasping key concepts; the significance of information in any reading is variable. Most importantly, the significance varies by individual. What is important to you may already be understood by someone else. What is well understood and therefore less important for you, may be significant for another. It is not I've tried to fill the book with impertinent information. It's that the book needs to be a complete source of necessary information. 

\begin{flushleft}
Here are some steps to help you determine what you should read closely and what might be peripheral.
\end{flushleft}

\begin{itemize}
\item{Get an overview (skim the entire chapter, perhaps write down what seems to be the outline based on the section headings and hierarchy).}
\item{Read the summary (if one exists) and conclusions first for a big-picture view.}
\item{Read the objectives if provided.}
\item{Attempt to answer any questions or prompts if provided.}
\item{Build more elaborate notes on main topics while you read the chapter.}
\end{itemize}

Consider why you need to learn what you are learning. Understanding why you need to learn and how you will use your knowledge will greatly improve your ability to get an overview for the big picture and the main topics. Specifically, we will focus on the reasoning process in which you use knowledge. Then from thinking about that reasoning process you can ask what it is you need to know to assist that process; once you gain information and understand it you can then consider how your reasoning process has been changed - hopefully enhanced.

Practical recommendations:
\begin{itemize}
\item Make note of section titles. Chapters are broken down - build more elaborate notes on these while you read the chapter
\item Big ideas: what main ideas are reflected in the introduction, conclusion and section titles? Be sure to record all relevant details of the big ideas in the text as you read the chapter
\item Follow visual cues: main ideas will often be bolded, italicized, bulleted, set in different font sizes, color, and/or spacing. Additionally, illustrations, figures, tables, charts, diagrams, and the corresponding captions elaborate on key ideas. Use these to determine the significance of concepts, and to take notes accordingly
\item What's repeated: concepts, facts, and processes mentioned more than once in the piece are likely significant
\end{itemize}

\subsection{Taking Notes}
Your optimal style may include some combination of the following:
\begin{itemize}
\item Dating your notes, and provide a heading that describes the piece's overall content
\item Numbering the pages of your notes
\item Paraphrasing instead of writing verbatim - writing in your own words, except for formulas, definitions, and specific facts (i.e. involving dates), which should be recorded exactly as in the text
\item Using consistent abbreviations and symbols
\item Developing an ideal organizational format, like an outline, table, or note cards, depending on content
\item Leaving room in the margins for additional thoughts or questions, or better still leaving the opposite page available for later use (meaning, you take your notes on every other page and leave the opposite page for additional thoughts and questions)
\item As a final step you may want to try typing your notes, which can be used for exam-studying, once you have clarified any ambiguities
\end{itemize}

%------------------------------------------------

%------------------------------------------------

\subsection{Excerpts and Commentary from \textit{How to Read a Book} by Mortimer Adler}

You will be reading to learn and, as Adler points out, there are really two kinds of learning. One kind of learning is simply getting more information (Adler calls this becoming informed). A second kind of learning is to come to understand what you did not understand before learning (Adler calls this becoming enlightened). Here he points out the difference.
\begin{displayquote}
"Getting more information is learning, and so is coming to understand what you did not understand before. But there is an important difference between these two kinds of learning. To be informed is to know simply that something is the case. To be enlightened is to know, in addition, what it is all about: why it is the case, what its connections are with other facts, in what respects it is the same, in what respects it is different, and so forth. This distinction is familiar in terms of the differences between being able to remember something and being able to explain it. If you remember what an author says, you have learned something from reading him. If what he says is true, you have even learned something about the world. But whether it is a fact about the book or a fact about the world that you have learned, you have gained nothing but information if you have exercised only your memory. You have not been enlightened. Enlightenment is achieved only when, in addition to knowing what an author says, you know what he means and why he says it."\cite{adler_how_1942}
\end{displayquote}


The goal of reading this book is for you to learn clinical physiology from a muscle centered point of view as a step towards being able to practice physical therapy by becoming enlightened. Of course, to become enlightened you must first become informed. One of the premises of Adler is that there are levels of reading that tend to correspond to the progress from becoming informed towards becoming enlightened. 
\vspace{0.2in}
\begin{flushleft}
There are four levels of reading:     
\end{flushleft}

\begin{itemize}
\item{Elementary}
\item{Inspectional}
\item{Analytical}
\item{Syntopical}
\end{itemize}
\vspace{.2in}

\subsection{Elementary Reading}

\begin{displayquote}
The first level of reading we will call Elementary Reading. Other names might be rudimentary reading, basic reading or initial reading; any one of these terms serves to suggest that as one masters this level one passes from nonliteracy to at least beginning literacy. In mastering this level, one learns the rudiments of the art of reading, receives basic training in reading, and acquires initial reading skills.
\end{displayquote}

You are here - you have this already. Meaning, you can read. You can open a book, read it and become informed.

\subsection{Inspectional Reading}
\begin{displayquote}
The second level of reading we will call Inspectional Reading. It is characterized by its special emphasis on time. 
\end{displayquote}

Hence, another way to describe this level of reading is to say that its aim is to get the most out of a book within a given time — usually a relatively short time, and always (by definition) too short a time to get out of the book everything that can be learned.

You will have to do an inspectional reading of entire chapters. That does not mean you read the entire chapter. You skim, browse, sample the entire chapter. You are learning from that process - gaining information. You gain information about what is in the chapter, its organization and structure and some about its content. If there is a large section on risk factors and there are subheadings about risk factors you are informed about the risk factors, and there are are lots of risk factors. If there is a short section on risk factors you know that we know little about the risk factors. 

An important aspect of inspectional reading for this program will be to identify the pieces of information in the readings that you need to understand. Those sections that you need to read analytically. In many ways you already understand aspects of the readings, for instance, you understand what the author means when they list risk factors. You then read and are informed about a particular risk factor for a particular disease. Then you read a bit more, reflect on it a bit and come to understand how that particular factor puts someone at risk for that particular disease. More specifically, you understand the concept of a risk factor already. Now you come to be informed that elevated c-reactive protein seems to be a risk factor for atherosclerosis. You read some more and consider that the current pathological process of atherosclerosis involves inflammation of the internal arterial walls and that c-reactive protein is an inflammatory marker, you now have a bit of understanding about the mechanisms underlying how c-reactive protein is a risk factor for atherosclerosis, and can most likely make other connections with this understanding.

Prior to taking notes from your inspectional reading you will want to mark up your book - here are some recommendations on book markings (you can save Access Physiotherapy readings as PDFs and either print them and mark them up, or mark up the PDFs electronically):

1. UNDERLINING (or highlighting) of major points; of important statements of fact or connections important to the argument (premises or conclusions).

2. VERTICAL LINES AT THE MARGIN to emphasize a statement already underlined or to point to a passage too long to be underlined. Basically supersedes underlying and marks things that deserve an analytical reading.

3. STAR, ASTERISK, OR OTHER DOODAD AT THE MARGIN to be used sparingly, to emphasize the ten or dozen most important statements or passages in the chapter. You may want to fold a corner of each page on which you make such marks or place a slip of paper between the pages. In either case, you will be able to take the book off the shelf at any time and, by opening it to the indicated page, refresh your recollection. These are marked as priorities for analytical reading.

4. NUMBERS IN THE MARGIN to indicate a sequence of points made by the author in developing an argument - generally very helpful when its time for making notes.

5. NUMBERS OF OTHER PAGES IN THE MARGIN to indicate where else in the book the author makes the same points, or points relevant to or in contradiction of those here marked; to tie up the ideas in a book, which, though they may be separated by many pages, belong together. Many readers use the symbol \textit{Cf} to indicate the other page numbers; it means compare or refer to.

6. CIRCLING OF KEY WORDS OR PHRASES (similar function as underlining) can be used to develop a list of words you need to find the definition of on your second read through of the chapter. It is really important to come away from readings with a larger vocabulary.

7. WRITING IN THE MARGIN, OR AT THE TOP OR BOTTOM OF THE PAGE to record questions (and perhaps answers) which a passage raises in your mind; to reduce a complicated discussion to a simple statement; to record the sequence of major points right through the book. Your process may include putting these directly into a notebook, it may be helpful to do this during inspectional reading - and they are questions that you might answer during your analytical reading so your notes will not have to include the question. If your analytical reading does not answer the question, then you will certainly want to include the question in your notes for class discussion.

\subsection{Analytical Reading}
\begin{displayquote}
The third level of reading we will call Analytical Reading. It is both a more complex and a more systematic activity than either of the two levels of reading.

Analytical reading is thorough and complete reading or good reading that is the best reading you can do. If inspectional reading is the best and most complete reading that is possible given a limited time, then analytical reading is the best and most complete reading that is possible given unlimited time. The analytical reader must ask many, and organized, questions of what he is reading.
\end{displayquote}

Ok, here you see that there is one clear aspect of analytical reading that we do not have the luxury of including - texit{given unlimited time}. We do not have unlimited time. So, our strategy will be to get through the majority of the readings with inspectional reading (gaining information) and gain understanding through a bit of analytical reading, and a bit of analyzing the information through the reasoning process that you will be using information for in practice. A major component of the analytic reading will be condensing the text to graphical  models (as much as possible) that provide you the causal structure of the concepts covered in the readings since the causal structures are the most relevant to physical therapy practice.

When you do select text for analytical reading - the general process includes finding the argument:
\begin{displayquote}
a sequence of propositions, some of which give reasons for another. This logical unit is not uniquely related to any recognizable unit of writing, as terms are related to words and phrases, and propositions to sentences. An argument may be expressed in a single complicated sentence. Or it may be expressed in a number of sentences that are only part of one paragraph. Sometimes an argument may coincide with a paragraph, but it may also happen that an argument runs through several or many paragraphs.
\end{displayquote}

And you should note: 
\begin{displayquote}
There are many paragraphs in any book (or paper) that do not express an argument at all, perhaps not even part of one. They may consist of collections of sentences that detail evidence or report how the evidence has been gathered. As there are sentences that are of secondary importance, because they are merely digressions or side remarks, so also can there be paragraphs of this sort. It hardly needs to be said that they should be read rather quickly. 

Because of all this, FIND IF YOU CAN THE PARAGRAPHS IN A BOOK THAT STATE ITS IMPORTANT ARGUMENTS; BUT IF THE ARGUMENTS ARE NOT THUS EXPRESSED, YOUR TASK IS TO CONSTRUCT THEM, BY TAKING A SENTENCE FROM THIS PARAGRAPH, AND ONE FROM THAT, UNTIL YOU HAVE GATHERED TOGETHER THE SEQUENCE OF SENTENCES THAT STATE THE PROPOSITIONS THAT COMPOSE THE ARGUMENT.
\end{displayquote}

Argument example:
C-reactive protein is a marker in the blood of an inflammatory process somewhere in the body
Atherosclerosis is a disease that starts with an inflammatory process
Therefore, c-reactive protein is a risk factor for atherosclerosis

Or:
Inflammation -> c-reactive protein,
Inflammation -> atheroscrosis,
Therefore, Inflammation -> (c-reactive protein AND atherosclerosis)

Please note the form of this argument. As you will learn in the program the form of the argument says something to us about the mechanisms and there are risk factors, for sure, with different mechanisms. Here we see the c- reactive protein is really an associated risk factor, meaning it is not a risk factor that directly causes atherosclerosis. It is a risk factor that is associated with something else that directly causes atherosclerosis.

The previous example was to demonstrate how we go from inspectional to analytical reading by finding (or constructing) arguments and how that leads to understanding. As you accumulate understanding, you will have an easier time getting an understanding because of the vast interconnections in general and specifically in the DPT program. 

This is such an important part of your learning for the program we will even cheat a bit due to time constraints and attempt to construct arguments out of inspectional reading. What we really need to do if define a good term for something between inspectional and analytical reading. Perhaps something such as \textit{inspectanaltyic} reading but there is likely a better option.


\section{Practical Books}

Adler makes some insightful and helpful comments on practical books. Most of the DPT readings are practical. Yes, there is a lot of theory and yes there is some science, but all in all it is about practice. It is about action and doing. 

\begin{displayquote}
The most important thing to remember about any practical book is that it can never solve the practical problems with which it is concerned.
\end{displayquote}

A practical book cannot solve the problems with which it is concerned, because solving problems that involve knowledge requires action in the world. 

\begin{displayquote}
A theoretical book can solve its own problems. But a practical problem can only be solved by action itself. When your practical problem is how to earn a living, a book on how to make friends and influence people cannot solve it, though it may suggest things to do. Nothing short of the doing solves the problem. It is solved only by earning a living.
\end{displayquote}

To paraphrase Adler for your context: When your practical problem is how to be a physical therapist, a book on being a physical therapist cannot solve it, though it may suggest things to do. Nothing short of doing solves the problem. It is solved only by being a physical therapist.

While you read through the program, make markings, grasp key words, choose what to read analytically, make notes to gain information and understanding related to the reasoning process essential to being a physical therapist that requires content knowledge - keep the following in mind: 

1. Doing (action) takes place in a particular situation, always in the here and now and under a particular set of circumstances. You cannot act in general. The kind of practical judgment that immediately precedes action must be highly particular. Whereas books are a mix of general and particular, but your action as a DPT will always be particular. 

Here is what we rely on:

\begin{displayquote}
Practical books can, however, state more or less general rules that apply to a lot of particular situations of the same sort. Whoever tries to use such books must apply the rules to particular cases and, therefore, must exercise practical judgment in doing so. In other words, the reader himself must add something to the book to make it applicable in practice. He must add his knowledge of the particular situation and his judgment of how the rule applies to the case.
\end{displayquote}

2. A practical book may contain more than rules. It may state principles that underlie the rules and make them understandable. 

\begin{displayquote}
The principles that underlie rules are usually in themselves scientific, that is, they are items of theoretical knowledge. Taken together, they are the theory of the thing. Thus, we talk about the theory of bridge building. We mean the theoretical principles that make the rules of good procedure what they are.

Practical books thus fall into two main groups. Some, are primarily presentations of rules. Whatever other discussion they contain is for the sake of the rules. There are few great books of this sort. The other kind of practical book is primarily concerned with the principles that generate rules. Most of the great books in economics, politics, and morals are of this sort.
\end{displayquote}

Many of the readings you will read in this program present rules but are primarily concerned with the principles that generate the rules. Thus there are two sorts of arguments in such books. There are arguments that attempt to show that the rules are sound - similar to books that present rules. But since these books are also trying to teach you the principles that generate the rules, the major propositions and arguments will look like those in a purely theoretical book. The propositions will say that something is the case, and the arguments will try to show that it is so.
\begin{displayquote}
But there is an important difference between reading such a book and reading a purely theoretical one. Since the ultimate problems to be solved are practical—problems of action, in fields where men can do better or worse, an intelligent reader of such books about practical principles always reads between the lines or in the margins. He tries to see the rules that may not be expressed but that can, nevertheless, be derived from the principles. He goes further. He tries to figure out how the rules should be applied in practice.
\end{displayquote}

Understanding the principles that generate rules allow you, in practice, to appropriately apply the appropriate rule in a particular context. 
\begin{displayquote}
Unless it is so read, a practical book is not read as practical. To fail to read a practical book as practical is to read it poorly. You really do not understand it, and you certainly cannot criticize it properly in any other way. If the intelligibility of rules is to be found in principles, it is no less true that the significance of practical principles is to be found in the rules they lead to, the actions they recommend. This indicates what you must do to understand either sort of practical book. It also indicates the ultimate criteria for critical judgment. In the case of purely theoretical books, the criteria for agreement or disagreement relate to the truth of what is being said. But practical truth is different from theoretical truth. A rule of conduct is practically true on two conditions: one is that it works; the other is that its working leads you to the right end, an end you rightly desire. Suppose that the end an author thinks you should seek does not seem like the right one to you. Even though his recommendations may be practically sound, in the sense of getting you to that end, you will not agree with him ultimately. And your judgment of his book as practically true or practically false will be made accordingly. If you do not think careful and intelligent reading is worth doing, this book (that is the book on \textit{How to Read a Book}) has little practical truth for you, however sound its rules may be.
\end{displayquote}


\printbibliography[heading=subbibintoc]



