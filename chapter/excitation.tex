% !TEX root = ../notes_template.tex
\chapter{Muscle Excitation}\label{chp:excitation}

\minitoc

The excitation of a muscle fiber starts with an excited motor axon ($\alpha$-motor neuron) and ends with the binding of calcium to troponin. Binding calcium to troponin  takes a crossbridge from an inactivated to activated state. Understanding excitation requires an understanding of excitable membranes. Understanding excitable membranes requires an understanding of membrane transport such as ion channels, pumps, ligands and receptors. Understanding membrane transport connects to homeostatic mechanisms of the neuroendocrine system influencing end organs; function of sensory receptors; and pharmacodynamics.

\vspace{5mm}

\textbf{Objectives include:}
\begin{enumerate}
    \item Explain the events of excitation - activation coupling.
    \item Relate the events of excitation - activation coupling to the creation of active tension.
\item Explain the components and capabilities of an excitable membrane.
    \item Explain the events that result in sarcolemma depolarization.
   \item Explain the events that result in neuromuscular endplate depolarization.
    \item Explain the function and role of the neuroendocrine system for homeostasis.
\item Explain the basic function of sensory receptors.
    \item Explain pharmacodynamics based on actions occurring at end organ receptors
   \item Demonstrate the ability to apply basic physiology concepts such as the cell membrane, mass balance, and flow gradients to the analysis of patient/client problems related to the generation of muscle excitation, the effectiveness of the neuroendocrine system and pharmaceuticals.
\end{enumerate}

\section{Muscle Excitation}
\paragraph{Excitation}
% THis section needs work!!!
Muscle excitation is the process that couples with crossbridge activation and results in a twitch. Excitation refers to particular sequence of events that changes the state of a membrane. These events can also be called an action potential. Excitation (or an action potential (AP)) is occurs when a membrane depolarizes. To depolarize the membrane has to be in a polarized state (also called a resting membrane potential). Any membrane in the body that undergoes excitation (that is, can be excited) has two states: excited and not excited; depolarized and polarized; action potential and resting membrane potential, and is called an excitable membranes. Nerve and muscle cell membranes are called excitable membranes. Excitation is a mechanism through which the body sends and receives signals. When membrane excitation results in an action in the cell that influences cellular processes, which includes processes such as exocytosis,\footnotemark\footnotetext{Exocytosis is a form of active transport and bulk transport in which a cell transports molecules (e.g., neurotransmitters and proteins) out of the cell). From \url{https://en.wikipedia.org/wiki/Ligand}} the excitation is sending a signal. When a membrane is excited through binding of a ligand\footnotemark\footnotetext{A ligand is any molecule or atom which binds reversibly to a protein. A ligand can be an individual atom or ion. It can also be a larger and more complex molecule made from many atoms. A ligand can be natural, as an organic or inorganic molecule. A ligand can also be made synthetically, in the laboratory. From \url{https://biologydictionary.net/ligand/}} to a receptor on its surface that excitation is receiving a signal. 

\subsection{Overview from $\alpha$-Motor Neuron to Crossbridge Activation}

Skeletal muscle fiber excitation starts with excitation of an $\alpha$-motor neuron (Step 1 in Figure \ref{fig:excitation_overview}) which excites a muscle fiber at the the motor end plate by crossing the neuromuscular junction (NMJ) (Step 2 in Figure \ref{fig:excitation_overview}). Excitation of the motor end plate causes excitation of the sarcolemma, which causes excitation of the transverse tubule system (T-tubules). Excitation of T-tubules cause excitation of the sarcoplasmic reticulum (SR), which causes the release of calcium into the muscle fiber and the activation of a crossbridge (Step 3 in Figure \ref{fig:excitation_overview}) This last step (Step 3 in Figure \ref{fig:excitation_overview}) is referred to as Excitation-Activation Coupling (or Excitation-Contraction Coupling). A single excitation results in a twitch.

\begin{figure}[!ht]
    \centering
    \includegraphics[width=1\linewidth]{./figure/excitation_overview.png}
    \caption{Overview of excitation from the $\alpha$ motor neuron to crossbridge activation \footnotesize{Created with BioRender.com}}
    \label{fig:excitation_overview}
\end{figure}

\paragraph{Step 1 - The $\alpha$-Motor Neuron}
The excitation of an $\alpha$-motor neuron is regulated by the central nervous system. Excitation occurs across the membrane of its dendrites in the spinal cord and the wave of excitation propagates along the axon to its terminals at neuromuscular junctions (NMJs) by sequential excitation facilitated by voltage gated ion channels. Each muscle fiber receives only one axon terminal and therefore has one NMJ. However, the number of axon terminals for each $\alpha$-motor neuron is highly variable (See Chapter \ref{chp:regulation}. Propagation of excitation flows from the spinal cord to the NMJ by means of saltatory conduction which allows excitation to big steps across the membrane. Saltatory conduction increases the nerve conduction velocity and is made possible by the myelin sheathes (Figure \ref{fig:Motoneuron}). The membrane is exposed along the axon at nodes of Ranvier. Excitation travels by jumping from one node of Ranvier to the next.\footnotemark\footnotetext{The implications of the myelin sheath on nerve conduction velocity is detailed in Chapter \ref{chp:regulation} as an important consideration for muscle tension regulation.} 

\begin{figure}[!ht]
    \centering
    \includegraphics[width=1\linewidth]{./figure/Motoneuron.png}
    \caption{$\alpha$-motor neuron terminating at several muscle fiber motor end plates \footnotesize{Created with BioRender.com}}
    \label{fig:Motoneuron}
\end{figure}

% Stopped here on June 2nd

\paragraph{Step 2 - The Neuromuscular Junction}

At the NMJ the excited axon terminal releases the neurotransmitter acetylcholine (ACh) which crosses the NMJ synapse. Neurotransmitters, as well as other molecules that bind to receptors, are called ligands. ACh binds a ligand gated ion channels on the motor end plate. 


\paragraph{Step 3 - Excitation - Activation Coupling}
This results in the excitation of the motor end plate which spreads to the sarcolemma and then the T-tubules due to voltage gated ion channels. There is a unique unique junction between T tubules and the sarcoplasmic reticulum (SR) mediated by calcium channels. One acts as voltage sensors in the T tubules, and a second acts as calcium release channels in the SR.\footnotemark{}\footnotetext{The position of these two calcium channels differ between skeletal and cardiac muscle, which correlates with the functional differences in the control of contraction between these two types of muscle.} The implications of this unique T-tubule - SR junction is that excitation of the T-tubule quickly results in release of calcium from the SR.

%NEed to add Ca pumping back into the SR



Figure of this anatomy and all of its components?

\section{Excitation - Activation Coupling}

Excitation activation coupling (EAC) (often referred to as excitation contraction coupling (ECC)) starts with excitation of the SR and ends with the binding of calcium to troponin (activation of the crossbridge). EAC results in a twitch, and adding twitches together results in tetany.

Figure of the sarcoplasmic reticulum

% write something about latency between excitation and activation

\begin{figure}[!ht]
    \centering
    \includegraphics[width=1\linewidth]{./figure/eac-latency.png}
    \caption{EAC Latency \footnotesize{(Wikimedia Commons, CC BY 4.0, \href{https://commons.wikimedia.org/wiki/File:The_latent_period_between_the_muscle_action_potential_and_contraction.png}{EAC-Latency})}}
    \label{fig:Motoneuron}
\end{figure}

\subsection{Twitch}
An excitation that results in a activation of enough crossbridges to produce a single temporary production of measurable tension is a twitch. It’s not useful to attempt to equate this to a single depolarization, and it is unlikely that a twitch refers to a single moment of crossbridge activation. A twitch can be an experimentally manipulated event. An electrical stimulus is provided to a muscle which results in either, or a combination of, $\alpha$-motor neuron excitation and sarcolemma excitation that excites the SR and activates enough crossbridges to produces enough active tension to produce a measurable force. A twitch can also be initiated by isolated voluntary excitations of an $\alpha$-motor neuron, or isolated involuntary excitations of an $\alpha$-motor neuron or sarcolemma. A twitch of a muscle fiber, or even many muscle fibers, is a fundamental unit of active tension. However, alone, a twitch of active tension does not produce what we will consider attaining tension since the goal or attaining tension is some sort of voluntary movement or stability. The origin of these twitch excitations is covered in more detail in upcoming sections of this chapter.



\subsection{Tetany}
Repeated excitation results in repeated activation and therefore continued crossbridge activation and repeated twitches. Successive twitches create tension that summates (adds together) in a process called tetany. When the force of active tension is measured, such as during a muscle test, it is the result of many muscle fibers being in tetany.  

Add a figure of twitches becoming tetany

\section{Molecular Events of Excitation}

Membrane excitation (depolarization) requires the resting potential state (polarization). A membrane cannot depolarize unless it is first polarized. Additional detail about excitable membranes such as understanding of how the resting state is established, how this resting state is excited, and how this wave of excitation spreads, is necessary to understand muscle excitation and how various clinical situations may influence excitation.

% Moved from above - not sure what to do with this....
Signal transmission that results in cellular events (including along the membrane) it is called signal transduction. The action potential signal transduction that occurs along the surface of a nerve and muscle membrane is facilitated by the presence of voltage gated channels embedded in the membrane. Signal transduction from a membrane to another membrane is facilitated by ligand gated channels at specific places in membrane. The detailed physiology of excitable membranes such as establishing and maintaining a resting membrane potential, and mechanisms that result in action potentials are important details covered later in the chapter. 
%------------------

\subsection{Excitable Membranes}

Excitable membranes have the capability of being in two states, a resting state and an excited state. The difference between these states is the electrical potential across the membrane. In the resting state the membrane is said to be polarized. In the excited state the membrane is not polarized, more commonly referred to as depolarized. The resting membrane potential (RMP) is a particular polarized potential across the membrane and includes the inside of the cell (intracellular) being more negative than the outside of the cell (extra cellular). By convention this is considered negative polarity.\footnotemark{}\footnotetext{This is by convention because it is a choice to describe polarity relative to the intracellular space. If we choose to describe polarity relative to the extra cellular space then the polarity is positive} During excitation an action potential (AP) changes the membrane potential (MP) from negative polarity to positive polarity for a short period of time (how long the MP has positive polarity depends on the particular cell membrane).

The RMP is not a natural (physically neutral) state. We consistently expend energy to maintain the RMP against a natural tendency for the polarity to dissipate. It is helpful to understand the cell membrane before further discussion of the events that establish RMP and allow for an AP.

\subsubsection{Cell Membrane}

The cell membrane is a semipermeable phospholipid bilayer barrier that forms a boundary for the cell. Part of its semipermeability is related to combination of the bilayer being hydrophobic (detracts water) and the inside of the membrane being hydrophilic (attracts water) which limits what can freely diffuse through the membrane. And part is due to the particular transport proteins within the membrane for selective forms of active, facilitated and passive transport. The membrane separates the intracellular from the extra cellular space. Through selective transport it establishes the conditions for cellular activities within the cell and thus influences (but does not completely establish) conditions outside the cell. Cell membranes are dynamic structures that adapt in response to the intracellular and extra cellular conditions to meet the needs of the cell.

Figure of a cell membrane

\paragraph{Sodium Potassium ATPase Pumps}
$Na^+/K^+$ ATPase pumps are embedded throughout excitable membranes. These protein based molecular machines convert the energy of ATP to the movement of three $Na^+$ outside the cell and two $K^+$ inside the cell with each cycle of the pump (each use of ATP). Given the natural tendency of molecules to move based on a concentration gradient (high concentration to low concentration) the resultant higher concentration of  $Na^+$ outside and higher concentration of $K^+$ inside the cell requires the work of these active transport pumps against the concentration gradient.

Figure of pumps

Refer to figure of the cell membrane 

\paragraph{Channels \& Receptors}
Channels are proteins embedded in the cell membrane that allow passage of molecules between the intracellular and extra cellular space. Channels can be gated and therefore only allow passage under certain circumstances. 
Voltage gated channels only allow passage when the membrane potential is at certain values. They may open at a certain value and close a certain period of time later; or they may close at a certain value and open a period of time later. 
Ligand gated channels only allow passage when the channel is activated due to the binding of a ligand (a molecule that can bind). A ligand gated channel is also a receptor, or closely associated with a receptor. 
Receptors are proteins that can bind to a ligand. Some receptors are ligand gated channels (a receptor where the activity that binding produces is opening a channel gate). Other receptors perform different activities within the cell. 

Figure of receptors

\paragraph{$Na^+$ \& $K^+$ Channels}

$Na^+$ channels allow the passage of $Na^+$ with its concentration gradient and can be either ligand gated or voltage gated. $K^+$ channels allow the passage of $K^+$ with its concentration gradient and in nerve and muscle excitable membranes are voltage gated. 

Refer to figure of the cell membrane

\subsection{Establishing the Resting Membrane Potential}

In a resting (non excited) state these membranes have a resting membrane potential (RMP). The RMP is a difference in electrical polarity across the membrane. For both nerve and muscle membranes the resting polarity is negative inside the cell relative to outside the cell. 


The resting membrane potential is an equilibrium potential. It is the consequence of two physiological situations. First, the presence of large concentration gradients for $K^+$ and $Na^+$ across the plasma membrane. Second, the relative permeability of the membrane to those ions. The concentration gradients for $K^+$ and $Na^+$ are the product of the activity of the $Na^+$-$K^+$-ATPase pumps. The concentration gradient maintains a large outwardly directed $K^+$ gradient, and the large inwardly directed $Na^+$ gradient.

The relative permeability of the plasma membrane to $Na^+$ and $K^+$ reflects the open versus closed status of ion-selective membrane channels. Different membranes display different degrees of permeability to different ions. In resting conditions the membrane of nerve and muscle cells are more permeable to $K^+$. Due to the concentration gradient $K^+$ flows out of the cell. As if flows out it creates a negative potential (leaves behind net negative charge inside the cell). However, this negative potential also works to prevent the flow of $K^+$ outside of the cell. The negative potential inside the cell attracts positive ions. Since the $K^+$ is the positive ion what the cell membrane is most permeable the outward flow of $K^+$ due to the concentration gradient is decreases to no net flow of $K^+$ due to the electrically driven inward flow (pull, attraction) for $K^+$ as a positive ion by the negative potential inside the cell. The potential that this balance between outward flow of $K^+$ due to concentration, and inward flow of $K^+$ due to electric potential is called the equilibrium potential. The resting membrane potential is the equilibrium potential. Since $K^+$ has the highest permeability at rest, the resting membrane potential is primarily influenced by the equilibrium potential of $K^+$.

Under normal, healthy, conditions different cell membranes have slightly different resting membrane (equilibrium) potentials. These differences are due to differences in the permeability of the membranes to $K^+$ (or to other ions, such as $Na^+$, $Cl^-$, $Ca^{2+}$). The extra cellular concentration of these ions is held constant throughout the body and therefore under normal circumstances differences in the concentration gradients are not a major source of differences in resting membrane potentials. 


\subsection{Trigger \& Spread of Sarcolemma Excitation (AP)}

An action potential involves the opening of voltage gated $Na^+$ channels due to a change in voltage in the resting membrane potential (becoming less negative). The number of voltage gated $Na^+$ channels that open is proportional to the change in the membrane potential. This cycle of opening in response to a voltage change is self perpetuating since once a $Na^+$ channel opens $Na^+$ flows into the cell due to the $Na^+$ concentration gradient. Due to the movement (not a change in the actual gradient) the membrane potential voltage changes (becomes less negative). The point that approximately half of the voltage gated $Na^+$ channels open is considered a threshold potential. At the threshold potential the voltage gated $Na^+$ channels continue to rapidly open until the equilibrium potential approaches that of $Na^+$ (as opposed to $K^+$) since at this point the membrane is more permeable to $Na^+$. The equilibrium potential for $Na^+$ is approximately 65 mV.  Once a membrane reaches the threshold potential an action potential is said to have occurred. It is at this point that the action potential is said to be “all or none.” Either it occurs or it does not occur. Prior to the threshold potential membranes can undergo micro potential changes. These micro potentials can be depolarizing (excitatory potentials that move the membrane potential toward the threshold), or hyperpolarizing (inhibitory potentials that move the membrane potential further away from the threshold potential).

Voltage gated $Na^+$ channels have two gates. One is an activation gate and it is closed under resting conditions and opens during excitation. The second is an inactivation gate and it is open under resting conditions but closes soon after the activation opens. The inactivation gate closure is timed, it is not triggered by events outside of the molecule itself. The inactivation gate is the first step to stopping an action potential. At this point the activation gate is open and the inactivation gate is closed. To be reset and completely ready for another action potential the activation gate must close and the inactivation gate must open (these are timed events). After an action potential the period of time that a large proportion of the $Na^+$ gates not yet reset is called the \textbf{absolute refractory period}. It is the period that regardless of how large a stimulus, the membrane cannot undergo another action potential. 

%%% Need to look up the timing of these events and relate to refractory periods….

Voltage gated $K^+$ channels open but are delayed when compared to $Na^+$. When they open the permeability for $K^+$ increases and the outward flow of more $K^+$ than under resting conditions starts to return the membrane potential to its resting value. Therefore, the opening of the $K^+$ channels is the second step to stopping an action potential. A short time after opening the $K^+$ will close (also a timed event, not triggered by outside events). During the time that the $Na^+$ gates are reset and the $K^+$ gate remains opened the membrane is said to be in a \textbf{relative refractory period}. It is the period that a normal stimulus the membrane will not achieve an action potential, but with a larger than normal stimulus an action potential is possible. 

Spread of excitation occurs because when one area of a membrane depolarizes during an action potential the change in membrane potential at nearby sections of the membrane starts the self perpetuating process of opening voltage gated $Na^+$ channels. This spread occurs in one direction (meaning it does not result in cyclic repeating action potentials of the entire membrane) because of the refractory periods. If a membrane is visualized as a set of dominoes and the resting potential is when the dominoes are upright, and an action potential is when a domino falls, then the refractory period is the period when the domino is still fallen. Because of the fallen domino on one side of the next falling domino, the dominos fall in one direction away from the starting stimulus. (Section needs work - and a figure)

\section{$\alpha$-motor Neuron Excitation \& the Neuromuscular Junction}

The beginning of muscle fiber excitation is motor end plate excitation. The motor end plate is a specialized are of the sarcolemma that has receptors, specifically ACh receptors that are also ligand gated $Na^+$ channels. These ligand gated $Na^+$ channels open when ACh binds to them which changes the membrane potential of the motor end plate. If there are just a few ACh receptor gates open then there may be an excitatory potential (also called a excitatory post synaptic potential (EPSP) since at this point the motor end plate is after (post) the synapse from the $\alpha$-motor neuron). However, if enough ACh receptor gates open and the voltage changes enough to open voltage gated channels, and the membrane potential reaches its threshold value then an action potential will occur in the motor end plate. This excitation will then spread along the entire sarcolemma down to the the T-tubules, resulting in the release of $Ca^{2+}$ from the SR. $Ca^{2+}$ in the muscle fiber will change the state of crossbridges from inactive to active. This single excitation may produce a twitch (a measurable generation of force from active tension) as long as the $Ca^{2+}$ released results in enough active crossbridges. 

\paragraph{Excitation to Regulation}

Many events have just been described for one excitation, particularly since one excitation is hardly enough to create a twitch, and a twitch is not enough to produce meaningful movement. A more efficient system of excitation-activation coupling exists in the cardiac muscle. However, in the skeletal muscle the efficiency of creating tetany from one excitation is sacrificed for the sake of regulation of muscle tension. Since creating tetany requires many action potentials, then the amount of tetany can be regulated by the number of action potentials. The regulation of muscle fibers is the topic for Chapter \ref{chp:regulation}.

\section{\textit{Clinical Physiology Connections}}

The concepts of excitable membranes, receptors, ligand channels and signal transduction are pervasive throughout clinical physiology. In this final section of the chapter the connection is made between these topics to neuroendocrine, sensory receptors and pharmacodynamics.

\subsection{Neuroendocrine}

The neuroendocrine system regulates several critical physiological systems and coordinates cellular functions throughout the body. Examples include mobilizing the body for fight or flight reactions by getting all cells in the body ready to meet demands that require energy mobilization at the expense of other maintanence activities. For example, release of hormones for fight or flight will signal liver cells to hold off on using energy to repair the cell membrane and instead use energy to convert glycogen (a stored fuel) into glucose (a usable fuel). Amazingly the neuroendocrine system can alert cells all over the body to do activities that each of those cells can do for the fight or flight situation, even though those activities are varied between the cells. While the liver cells are releasing glucose, the heart cells are prepared to use more glucose and are working more frequently (higher rate) and harder (more pressure). The differentiation of tasks completed by these cells is due to the specialization of the cells, and due to the receptors around the body that are responding to the same ligand. Most cells in teh body have a ligand receptor for epinephrine (a major catecholamine of the fight or flight response). Epinephrine is also called adrenaline. Receptors for epinephrine are therefore called adrenergic receptors. The adrendergic receptors on all cells have a similar binding location for epinephrine, but the receptor itself transducts different signals into the cell depending on the receptor and depending on the cell. 

\subsection{Sensory Receptors}

Sensory receptors are specialized receptors at the terminal end of the axon of a sensory neuron. They differ from ligand and voltage gated receptors in the signals that provokes their excitation. In general, the signal the provokes excitation in a sensory neuron is whatever phenomenon or sensation that sensory neuron is associated. If the sensory neuron is excited by changes in temperature then it is a temperature sensory neuron; if by changes in tension, it is a tension sensory receptor; if by vibrations, it is a vibration (as in the ear) sensory receptor; if by photons, it is a light sensory receptor (as in the retina); if by certain non specifically chemicals or ions (such as hydrogen for pH) they are chemoreceptors; if by changes in pressure they are baroreceptors. Some sensory receptors terminate in areas of the brain that allow perceptions of the signal in a way that gives them additional meaning. For example, some chemoreceptors proceed to the brain and produce pain signals, once in the brain a complicated cascade of connections are made that can, over time, change from being protective to themselves harmful to movement. Other chemoreceptors that detect an increase in carbon dioxide proceed to areas in the nervous system to produce unconscious reflex responses (changing respiratory rate) and also to the brain (a general state of anxiety and air hunger). Some receptors don’t go to the brain at all, but interact at areas in the nervous  system where responses are automatic (baroreceptors as part of the autonomic nervous system to regulate blood pressure). Since baroreceptors don’t go to the brain at all we do not perceive what our blood pressure is directly - making elevated blood pressure a “silent killer.” Most physiological  homeostasis regulatory networks involve a sensory receptor of some type that contributes to the monitoring of a homeostatic variable. 

\subsubsection{Sensory Receptor Dynamics}
As receptors the sensory receptors (all of them) are subjects to the same dynamics discussed as part of the cell membrane and pharmacodynamics. When a sensory receptor undergoes up regulation or down regulation there are then changes to when the sensory receptors signal their respective neurons. With upregulation of chemoreceptors for carbon dioxide an individual will tend to have a higher respiratory rate and maintain a lower carbon dioxide level since their response to carbon dioxide is increased. The upregulation then continues to propagate itself. Since it results in lower carbon dioxide levels, the receptors remain upregulated in an attempt to be able to respond to lower carbon dioxide levels. A balance point is eventually achieved. But this new balance point may be enough to result in what is converted altered homeostasis.

\subsubsection{Neuromuscular Sensory Receptors} 
There are several important neuromuscular sensory receptors that sense signals from the muscle, musculotendonous junction and/or the tendon. Signals for tension (such as golgi tendon organs), signals for changes in length (such as the muscle spindle), and signals for the chemical constituents of the extra cellular fluid surrounding muscle fibers (chemoreceptors). These receptors and their role in muscle function will be covered in upcoming chapters on regulation, energetics and microcirculation.

\subsection{Pharmacodynamics}

Pharmacodynamics considers the biochemical and physiologic effects of drugs. The biochemical effects are completely related to the drug as a ligand. 

\section{Summary \& Next Steps}


% Figures

\begin{figure}[!ht]
    \centering
    \includegraphics[width=1\linewidth]{./figure/Myofibril_Structure.png}
    \caption{Myofibril Structure \footnotesize{(Created with Biorender.com)}}
    \label{fig:Myofibril_Structure}
\end{figure}


\begin{figure}[!ht]
    \centering
    \includegraphics[width=1\linewidth]{./figure/eac-latency.png}
    \caption{$\alpha$-motor neuron terminating at several muscle fiber motor end plates \footnotesize{(Wikimedia Commons, CC BY 4.0, \href{https://commons.wikimedia.org/wiki/File:Motoneuron.png}{Motoneuron})}}
    \label{fig:Motoneuron}
\end{figure}

\printbibliography[heading=subbibintoc]



