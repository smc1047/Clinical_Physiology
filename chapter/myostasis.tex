% !TEX root = ../notes_template.tex
\chapter{Myostasis}\label{chp:myostasis}
Updated on \today
\minitoc
This chapter introduces readers to the physical stress theory \cite{mueller_tissue_2002}. It extends this theory with new insights into the hypertrophy, isotrophy and atrophy signal transduction pathways. It also covers the molecular epigentic basis for muscle memory and the emerging understanding of the gut microbiome - muscle axis. Adaptation (responsiveness to training) and plasticity (ability to adapt) based on genetic factors, and in response to interventions that do not cause injury are considered.

% Include basics of immunological function

\vspace{5mm}

\textbf{Objectives include:}
\begin{enumerate}
    \item
    \item
    \item
    \item
    \item
\end{enumerate}

\subsection{Muscle Integrity}
Muscle Integrity focuses on muscle fibers being muscle fibers, capable of generating tension. Muscle fibers exist to execute the act of generating tension so integrity is about the parts and capabilities that make a muscle fiber a muscle fiber. Muscle Integrity considers both ECF in support of, and movement influencing, muscle fibers. ECF supports the long term act of creating muscle tension by providing the necessary resources for the muscle to maintain its integrity (it continues to be a muscle and therefore to be able to do the act of creating tension). Movement is also involved (the arrow from movement to muscle in Figure \ref{fig:muscle_centered_approach}) because movement, as you probably know, provides (or does not provide) a stimulus for the muscle fiber. Muscle cells create movement and movement creates stress and strain on muscle fibers that act as triggers for myostasis (muscle fibers continue to be, which we will call isotrophy) or to grow (hypertrophy) and when not needed to reduce their need for resources (atrophy). Isotrophy, hypertrophy and atrophy occur on a spectrum covered in Part III Muscle Integrity. Muscle Integrity includes the cellular activity occurring within muscle fibers, and is supported by the ECF and stimulated by movement.

Metrics of Muscle Integrity include the ability to continue to do what a muscle cell does (persevere) which occurs through maintaining the various parts, capabilities and interactions. A measurement of integrity could be the cross section area of a muscle, or the lean mass of muscle, as indicators of how well the muscle (as a muscle collection of muscle fibers) is being maintained.

\section{Muscle Development}

\section{Physical Stress Theory}

\section{Hypertrophy - Isotrophy - Atrophy Spectrum}

%How an activity influences muscle fatigue is based on the overload. The overload training concept is the stimulus for training related muscle adaptations. Training adaptations often occur in response to training fatigue in an attempt to build system capacity to avoid future fatigue. The relationship between fatigue and training adaptations is provided in Chapter \ref{chp:myostasis} on Myostasis.

\subsection{Signal Transduction Pathways}

\subsection{Transcription - Translation}

\subsection{Molecular Muscle Memory}

\subsection{Gut Microbiome - Muscle Axis}

\section{Adaptation \& Plasticity}

\subsection{Training Responses}

\subsection{Hypoxia}

\subsection{Blood Flow Restriction}

\printbibliography[heading=subbibintoc]