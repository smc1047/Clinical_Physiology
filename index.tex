\input{./preamble.tex}
% \input{./glossaries.tex}
\input{./macros.tex}

\begin{document}

\title{manuscript in progress \\~\\ {\bf{\huge{Clinical Physiology:}}  \\ a muscle centered approach}}

\author{\textbf{Authors:} \\ Sean Collins \\ Bog Sniezek \\}

\date{Updated on \today}
\maketitle

%--copyright--------------------------------------------------
\pagebreak
\thispagestyle{empty}

{\small
Copyright \copyright 2023 Sean Collins

\vspace{0.2in}

\begin{flushleft}
Sean Collins, PT, ScD       \\
Professor of Clinical Inquiry        \\
Doctor of Physical Therapy Program \\
Plymouth State University \\
17 High Street, MSC68 \\
Plymouth, NH 03264 \\
smcollins1@plymouth.edu 
\end{flushleft}

\vspace{0.2in}
The original form of this book is \LaTeX\ source code.  Compiling this \LaTeX\ source has the effect of generating a device-independent representation of a textbook, which can be converted to other formats and printed. However, this also means there will be an occasional typo with larger than expected, or strange, consequences. If this is noticed - please let the author know at the address above. 
The \LaTeX\ source code can be found on \href{https://github.com/scollinspt/Clinical_Physiology}{ GitHub at this Link.}

\newpage
\section*{About the book}
Clinical Physiology: A muscle centered approach is a manuscript (book) in progress. The draft you are reading was released \today.

The project emerged during the authors' conversations about Unifying Systems Theory (UST) \cite{kahlen_perception_2017}. UST provided the system framework for the primary author (Collins) to consider how physical therapists should learn clinical physiology as a foundational knowledge for practice. Our conclusion was a muscle centered approach.

\vspace{0.2in}

A muscle centered approach puts the muscle fiber (cell) at the center of inquiry. Part I establishes some preliminaries to put the book and the muscle fiber in context. The context of the book is the reader's learning. The context of the muscle fiber is in the wider area of inquiry regarding movement.  Part II focuses on what a muscle fiber does as a system. Muscle fibers create tension. The process of creating tension is an inward looking process. We dig deep into the cellular, mechanical and chemical mechanisms of the muscle fiber. Part III focuses on the necessary supporting systems for the muscle fiber. Each muscle fiber receives and releases material from its surrounding environment (extra cellular fluid). Several physiological systems integrate to ensure the stability of that environment. Part IV focuses on the muscle molecular biology. Specifically, how the muscle continues to fulfill its role of creating tension without being too large of a burden on the physiological support systems. Topics include myostasis, atrophy, hypertrophy, and repair. Part V includes a variety of integrated exercises. These topics have been selected to highlight the need for physiological coordination.

\vspace{5mm}

\section*{Clinical Physiology in a DPT curriculum}
A DPT program has the challenge of educating individuals from diverse backgrounds in the core set of knowledge to become physical therapists. Clinical Physiology is foundational knowledge. The book fits best as a first term course, or a first course on human physiology.\footnotemark{}\footnotetext{The book assumes the reader has taken pre requisite courses such as anatomy \& physiology, physics, and chemistry.} It is not an exercise physiology text, though it has a lot of content that would be found in an exercise physiology text. It is not a medical physiology text, though it has a lot of content that would be found in a medical physiology text. The book includes foundational knowledge of physiology for a physical therapist. That knowledge falls in a previously unmet textbook space that combines exercise physiology and medical physiology. 

\section*{Contributors}

 \subsection*{Sean Collins, PT, ScD}
Sean Collins is a physical therapist with a doctor of science in ergonomics and epidemiology from the UMass Lowell. He was professor of physical therapy and biomedical engineering at UMass Lowell for 18 years prior to relocating to Plymouth State University (PSU) as the founding director of the Doctor of Physical Therapy (DPT) program where is currently a professor. This project is possible because of the PSU DPT curriculum, and the generous sabbatical support from the PSU administration and DPT faculty colleagues. He is the primary editor for the project and author of most chapters.

\subsection*{Bog Sniezek}
Bog Sniezek is a systems scientist (more to come). He is the supporting editor and founder of the systems science approach being utilized (Unified Systems Theory).

\subsection*{Tom Sniezek, PT, DPT}
Tom Sniezek is a physical therapist, graduate of the DPT program at UMass Lowell and currently practices physical therapy in Ohio. (More on role to come)

\subsection*{Nathaniel Mailloux, PT, DPT}
Nathaniel Mailloux is a physical therapist, graduate of the DPT program at Plymouth State University and currently practices physical therapy in New York. He is the primary author for the Integrative Exercise on Fick's Equation.

\subsection*{Kyle Coffey, PT, DPT}
Bio coming soon. He is the primary author for the Integrative Exercise on Exercise Prescription; and co-author for the chapters in the section on Muscle Integrity.

\subsection*{Kelly Legacy, PT, DPT}
Clinical Assistant Professor \& Director of Clinical Education at Plymouth State University. Dr. Legacy teaches the Exercise Prescription and Nutrition course in the DPT program. She is the primary author of the chapter on Digestion, Absorption and Metabolism.

} % end small

%---end--copyright--------------------------------------------------------


\setcounter{tocdepth}{1}
\setcounter{minitocdepth}{1} 

%\begin{multicols}{2}
    \dominitoc% Initialization
    \adjustmtc[2]% chp number shift
    \tableofcontents
    \label{toc-contents}
%\end{multicols}

%  \listoffigures
%	\listoftables
	
%\begin{multicols}{2}
% \listoftheorems[ignoreall,show={theorem}]
%\end{multicols}
%\renewcommand{\listtheoremname}{List of \textit{Connection} Topics}
%\begin{multicols}{2}
%	\listoftheorems[ignoreall,show={definition}]
%\end{multicols}

	%\printglossaries
	%bib2gls
	% \printunsrtglossaries % print all types
	%\printunsrtglossary[type={abbreviations},title=List of \textit{Connection} Topics,style=listgroup]
	% \printunsrtglossary[type={abbreviations},title=List of Abbreviations,style=listhypergroup] % doesn't work
%	\printunsrtglossary[type={symbols},title=List of Symbols,style=listgroup]
	% \printunsrtglossary % main entry

%%%%%%%%%%%%%%%Content%%%%%%%%%%%%%%%
% \mainmatter % separat the number of toc and mainmatter

%\frontmatter

%\input{}
%\chapter*{Contributors}
\addcontentsline{toc}{chapter}{Contributors}

\subsection*{Sean Collins}
Sean Collins is a physical therapist with a doctor of science in ergonomics and epidemiology from the UMass Lowell. He was professor of physical therapy and biomedical engineering at UMass Lowell for 18 years prior to relocating to Plymouth State University (PSU) as the founding director of the Doctor of Physical Therapy (DPT) program where is currently a professor. This project is possible because of the PSU DPT curriculum, and the generous sabbatical support from the PSU administration and DPT faculty colleagues. He is the primary editor for the project and author of several chapters.

\subsection*{Bog Sniezek}
Bog Sniezek is a systems scientist (more to come). He is the supporting editor and founder of the systems science approach being utilized (Unified Systems Theory).

\subsection*{Tom Sniezek}
Tom Sniezek is a physical therapist, graduate of the DPT program at UMass Lowell and currently practices physical therapy in Ohio. (More on role to come)

\subsection*{Nathaniel Mailloux}
Nathaniel Mailloux is a physical therapist, graduate of the DPT program at Plymouth State University and currently practices physical therapy in New York. He is the primary author for the Integrative Exercise on Fick's Equation.

\subsection*{Kyle Coffey}
Bio coming soon. He is the primary author for the Integrative Exercise on Exercise Prescription; and co-author for the chapters in the section on Muscle Integrity.

\printbibliography[heading=subbibintoc]
%\input{./chapter/preface.tex}


%\mainmatter
\part{Preliminaries}

% Book preface - about the project, etc.
\chapter*{Advice to the Reader}
\addcontentsline{toc}{chapter}{Advice to the Reader}

The second draft. Not the second edition. Draft. The work in front of you today is one year old. In early April 2022, nothing you read in this manuscript was available. It represents a years worth of work condensed into a few months of the year. Why? Because to work on this project requires an author  with the challenges I have on focusing to have no other demands on my time. It requires a singular focus. I tend to have a singular focus on clinical physiology from approximately mid May through August every year. That's the time frame during which I can work on this book.

What does it mean to be a second draft? 
I think Vonnegut's title for his notes and comments on writing style sum it up best: "Pity the reader." \cite{vonnegut_pity_2019} But at least it's not a first draft. Reading a first draft "is a lot like inflating a blimp with a bicycle pump. Anybody can do it. All it takes is time." \cite{vonnegut_pity_2019}

\section{Reading to Learn}

What follows are notes on reading and learning for your time in this course in particular, and the DPT program in general. These are just suggestions. Textbooks in general, and your DPT books in particular, tend to be loaded with information. Some of the information you already know, much you do not. Much of the information across various books and within a book overlaps. Some of the information is extraneous, some of it is necessary.\footnotemark{}\footnotetext{I've tried in this second edition to eliminate as much of the extraneous material as possible.} The trick to effective reading as part of your studying relies on your ability to cut through to the necessary core. Readers all come from different backgrounds and have had different experiences. Therefore, some of what is necessary is that for all; some of what is necessary is just for you. Hence, reading is a very personal adventure. Ultimately reading is a conversation between you and the writer. As an avid reader and writer, I read and write as if I were having a conversation with the writer or reader. This \textbf{dialectic} approach tends to work well for helping with studying and learning. Most of the advice on How to Read a Book being provided is from the classic text on this topic by Mortimer Adler and Charles van Doren \cite{adler_how_1942}.

\subsection{From Unconscious Incompetence towards Conscious Competence}

The first transition for learning is moving from unconscious incompetence to conscious incompetence. In other words, reading about things you do not know allows you to take the first step towards learning - you go from not knowing what you do not know (\textit{unconscious incompetence}) towards knowing what you do not know (\textit{conscious incompetence}). It is only from conscious incompetence that you can begin moving towards the next step - \textbf{conscious competence}. 

The tension associated with conscious incompetence is motivating to the active processes necessary to learn. However, that tension can also be overwhelming. It is helpful to have a well understood plan for how to get from conscious incompetence to conscious competence. Helping you with that plan is the job of your teacher. But all your teacher can do is help, it is ultimately something you need to do. The more active you are with the process, the more successful you will be. \cite{brown_make_2014}

The plan proposed in this reading is to understand first that this process is taking place; second, the format that information is being presented; and third and what it is that you need to take away from it. 

You have already accomplished the first part of the plan by being in the program and in classes - you are now aware that process is taking place! It is important to keep it in mind as you read each chapter. 

Your active process should include these components. 

\begin{itemize}
\item Read
\item Take notes on your reading
\item Discuss what you are learning with your classmates
\item Discuss what you are learning with your teacher
\end{itemize}

Keep in mind the last bullet above (\textit{Discuss what you are learning with your teacher}) happens during all and any interactions with your teacher and this includes written communication (most clearly assignments).

Cycling through this active process addresses the second and third part of the plan. That is, you will repeatedly be exposed to new information through reading. You will also repeatedly consider what it is that you need to take away from the information. This is where the notes and discussions are important parts of the plan. You will have to think about that information as you take notes on it and you should focus your notes on what you need to know about it. Discussions are to guide you - individually and as a group - through the process; affirming you are learning what you need to learn and correcting your process as needed.

\section{Reading Tips}

You do not need to read every word. The goal of reading and note-taking are grasping key concepts; the significance of information in any reading is variable. Most importantly, the significance varies by individual. What is important to you may already be understood by someone else. What is well understood and therefore less important for you, may be significant for another. It is not I've tried to fill the book with impertinent information. It's that the book needs to be a complete source of necessary information. 

\begin{flushleft}
Here are some steps to help you determine what you should read closely and what might be peripheral.
\end{flushleft}

\begin{itemize}
\item{Get an overview (skim the entire chapter, perhaps write down what seems to be the outline based on the section headings and hierarchy).}
\item{Read the summary (if one exists) and conclusions first for a big-picture view.}
\item{Read the objectives if provided.}
\item{Attempt to answer any questions or prompts if provided.}
\item{Build more elaborate notes on main topics while you read the chapter.}
\end{itemize}

Consider why you need to learn what you are learning. Understanding why you need to learn and how you will use your knowledge will greatly improve your ability to get an overview for the big picture and the main topics. Specifically, we will focus on the reasoning process in which you use knowledge. Then from thinking about that reasoning process you can ask what it is you need to know to assist that process; once you gain information and understand it you can then consider how your reasoning process has been changed - hopefully enhanced.

Practical recommendations:
\begin{itemize}
\item Make note of section titles. Chapters are broken down - build more elaborate notes on these while you read the chapter
\item Big ideas: what main ideas are reflected in the introduction, conclusion and section titles? Be sure to record all relevant details of the big ideas in the text as you read the chapter
\item Follow visual cues: main ideas will often be bolded, italicized, bulleted, set in different font sizes, color, and/or spacing. Additionally, illustrations, figures, tables, charts, diagrams, and the corresponding captions elaborate on key ideas. Use these to determine the significance of concepts, and to take notes accordingly
\item What's repeated: concepts, facts, and processes mentioned more than once in the piece are likely significant
\end{itemize}

\subsection{Taking Notes}
Your optimal style may include some combination of the following:
\begin{itemize}
\item Dating your notes, and provide a heading that describes the piece's overall content
\item Numbering the pages of your notes
\item Paraphrasing instead of writing verbatim - writing in your own words, except for formulas, definitions, and specific facts (i.e. involving dates), which should be recorded exactly as in the text
\item Using consistent abbreviations and symbols
\item Developing an ideal organizational format, like an outline, table, or note cards, depending on content
\item Leaving room in the margins for additional thoughts or questions, or better still leaving the opposite page available for later use (meaning, you take your notes on every other page and leave the opposite page for additional thoughts and questions)
\item As a final step you may want to try typing your notes, which can be used for exam-studying, once you have clarified any ambiguities
\end{itemize}

%------------------------------------------------

%------------------------------------------------

\subsection{Excerpts and Commentary from \textit{How to Read a Book} by Mortimer Adler}

You will be reading to learn and, as Adler points out, there are really two kinds of learning. One kind of learning is simply getting more information (Adler calls this becoming informed). A second kind of learning is to come to understand what you did not understand before learning (Adler calls this becoming enlightened). Here he points out the difference.
\begin{displayquote}
"Getting more information is learning, and so is coming to understand what you did not understand before. But there is an important difference between these two kinds of learning. To be informed is to know simply that something is the case. To be enlightened is to know, in addition, what it is all about: why it is the case, what its connections are with other facts, in what respects it is the same, in what respects it is different, and so forth. This distinction is familiar in terms of the differences between being able to remember something and being able to explain it. If you remember what an author says, you have learned something from reading him. If what he says is true, you have even learned something about the world. But whether it is a fact about the book or a fact about the world that you have learned, you have gained nothing but information if you have exercised only your memory. You have not been enlightened. Enlightenment is achieved only when, in addition to knowing what an author says, you know what he means and why he says it."\cite{adler_how_1942}
\end{displayquote}


The goal of reading this book is for you to learn clinical physiology from a muscle centered point of view as a step towards being able to practice physical therapy by becoming enlightened. Of course, to become enlightened you must first become informed. One of the premises of Adler is that there are levels of reading that tend to correspond to the progress from becoming informed towards becoming enlightened. 
\vspace{0.2in}
\begin{flushleft}
There are four levels of reading:     
\end{flushleft}

\begin{itemize}
\item{Elementary}
\item{Inspectional}
\item{Analytical}
\item{Syntopical}
\end{itemize}
\vspace{.2in}

\subsection{Elementary Reading}

\begin{displayquote}
The first level of reading we will call Elementary Reading. Other names might be rudimentary reading, basic reading or initial reading; any one of these terms serves to suggest that as one masters this level one passes from nonliteracy to at least beginning literacy. In mastering this level, one learns the rudiments of the art of reading, receives basic training in reading, and acquires initial reading skills.
\end{displayquote}

You are here - you have this already. Meaning, you can read. You can open a book, read it and become informed.

\subsection{Inspectional Reading}
\begin{displayquote}
The second level of reading we will call Inspectional Reading. It is characterized by its special emphasis on time. 
\end{displayquote}

Hence, another way to describe this level of reading is to say that its aim is to get the most out of a book within a given time — usually a relatively short time, and always (by definition) too short a time to get out of the book everything that can be learned.

You will have to do an inspectional reading of entire chapters. That does not mean you read the entire chapter. You skim, browse, sample the entire chapter. You are learning from that process - gaining information. You gain information about what is in the chapter, its organization and structure and some about its content. If there is a large section on risk factors and there are subheadings about risk factors you are informed about the risk factors, and there are are lots of risk factors. If there is a short section on risk factors you know that we know little about the risk factors. 

An important aspect of inspectional reading for this program will be to identify the pieces of information in the readings that you need to understand. Those sections that you need to read analytically. In many ways you already understand aspects of the readings, for instance, you understand what the author means when they list risk factors. You then read and are informed about a particular risk factor for a particular disease. Then you read a bit more, reflect on it a bit and come to understand how that particular factor puts someone at risk for that particular disease. More specifically, you understand the concept of a risk factor already. Now you come to be informed that elevated c-reactive protein seems to be a risk factor for atherosclerosis. You read some more and consider that the current pathological process of atherosclerosis involves inflammation of the internal arterial walls and that c-reactive protein is an inflammatory marker, you now have a bit of understanding about the mechanisms underlying how c-reactive protein is a risk factor for atherosclerosis, and can most likely make other connections with this understanding.

Prior to taking notes from your inspectional reading you will want to mark up your book - here are some recommendations on book markings (you can save Access Physiotherapy readings as PDFs and either print them and mark them up, or mark up the PDFs electronically):

1. UNDERLINING (or highlighting) of major points; of important statements of fact or connections important to the argument (premises or conclusions).

2. VERTICAL LINES AT THE MARGIN to emphasize a statement already underlined or to point to a passage too long to be underlined. Basically supersedes underlying and marks things that deserve an analytical reading.

3. STAR, ASTERISK, OR OTHER DOODAD AT THE MARGIN to be used sparingly, to emphasize the ten or dozen most important statements or passages in the chapter. You may want to fold a corner of each page on which you make such marks or place a slip of paper between the pages. In either case, you will be able to take the book off the shelf at any time and, by opening it to the indicated page, refresh your recollection. These are marked as priorities for analytical reading.

4. NUMBERS IN THE MARGIN to indicate a sequence of points made by the author in developing an argument - generally very helpful when its time for making notes.

5. NUMBERS OF OTHER PAGES IN THE MARGIN to indicate where else in the book the author makes the same points, or points relevant to or in contradiction of those here marked; to tie up the ideas in a book, which, though they may be separated by many pages, belong together. Many readers use the symbol \textit{Cf} to indicate the other page numbers; it means compare or refer to.

6. CIRCLING OF KEY WORDS OR PHRASES (similar function as underlining) can be used to develop a list of words you need to find the definition of on your second read through of the chapter. It is really important to come away from readings with a larger vocabulary.

7. WRITING IN THE MARGIN, OR AT THE TOP OR BOTTOM OF THE PAGE to record questions (and perhaps answers) which a passage raises in your mind; to reduce a complicated discussion to a simple statement; to record the sequence of major points right through the book. Your process may include putting these directly into a notebook, it may be helpful to do this during inspectional reading - and they are questions that you might answer during your analytical reading so your notes will not have to include the question. If your analytical reading does not answer the question, then you will certainly want to include the question in your notes for class discussion.

\subsection{Analytical Reading}
\begin{displayquote}
The third level of reading we will call Analytical Reading. It is both a more complex and a more systematic activity than either of the two levels of reading.

Analytical reading is thorough and complete reading or good reading that is the best reading you can do. If inspectional reading is the best and most complete reading that is possible given a limited time, then analytical reading is the best and most complete reading that is possible given unlimited time. The analytical reader must ask many, and organized, questions of what he is reading.
\end{displayquote}

Ok, here you see that there is one clear aspect of analytical reading that we do not have the luxury of including - texit{given unlimited time}. We do not have unlimited time. So, our strategy will be to get through the majority of the readings with inspectional reading (gaining information) and gain understanding through a bit of analytical reading, and a bit of analyzing the information through the reasoning process that you will be using information for in practice. A major component of the analytic reading will be condensing the text to graphical  models (as much as possible) that provide you the causal structure of the concepts covered in the readings since the causal structures are the most relevant to physical therapy practice.

When you do select text for analytical reading - the general process includes finding the argument:
\begin{displayquote}
a sequence of propositions, some of which give reasons for another. This logical unit is not uniquely related to any recognizable unit of writing, as terms are related to words and phrases, and propositions to sentences. An argument may be expressed in a single complicated sentence. Or it may be expressed in a number of sentences that are only part of one paragraph. Sometimes an argument may coincide with a paragraph, but it may also happen that an argument runs through several or many paragraphs.
\end{displayquote}

And you should note: 
\begin{displayquote}
There are many paragraphs in any book (or paper) that do not express an argument at all, perhaps not even part of one. They may consist of collections of sentences that detail evidence or report how the evidence has been gathered. As there are sentences that are of secondary importance, because they are merely digressions or side remarks, so also can there be paragraphs of this sort. It hardly needs to be said that they should be read rather quickly. 

Because of all this, FIND IF YOU CAN THE PARAGRAPHS IN A BOOK THAT STATE ITS IMPORTANT ARGUMENTS; BUT IF THE ARGUMENTS ARE NOT THUS EXPRESSED, YOUR TASK IS TO CONSTRUCT THEM, BY TAKING A SENTENCE FROM THIS PARAGRAPH, AND ONE FROM THAT, UNTIL YOU HAVE GATHERED TOGETHER THE SEQUENCE OF SENTENCES THAT STATE THE PROPOSITIONS THAT COMPOSE THE ARGUMENT.
\end{displayquote}

Argument example:
C-reactive protein is a marker in the blood of an inflammatory process somewhere in the body
Atherosclerosis is a disease that starts with an inflammatory process
Therefore, c-reactive protein is a risk factor for atherosclerosis

Or:
Inflammation -> c-reactive protein,
Inflammation -> atheroscrosis,
Therefore, Inflammation -> (c-reactive protein AND atherosclerosis)

Please note the form of this argument. As you will learn in the program the form of the argument says something to us about the mechanisms and there are risk factors, for sure, with different mechanisms. Here we see the c- reactive protein is really an associated risk factor, meaning it is not a risk factor that directly causes atherosclerosis. It is a risk factor that is associated with something else that directly causes atherosclerosis.

The previous example was to demonstrate how we go from inspectional to analytical reading by finding (or constructing) arguments and how that leads to understanding. As you accumulate understanding, you will have an easier time getting an understanding because of the vast interconnections in general and specifically in the DPT program. 

This is such an important part of your learning for the program we will even cheat a bit due to time constraints and attempt to construct arguments out of inspectional reading. What we really need to do if define a good term for something between inspectional and analytical reading. Perhaps something such as \textit{inspectanaltyic} reading but there is likely a better option.


\section{Practical Books}

Adler makes some insightful and helpful comments on practical books. Most of the DPT readings are practical. Yes, there is a lot of theory and yes there is some science, but all in all it is about practice. It is about action and doing. 

\begin{displayquote}
The most important thing to remember about any practical book is that it can never solve the practical problems with which it is concerned.
\end{displayquote}

A practical book cannot solve the problems with which it is concerned, because solving problems that involve knowledge requires action in the world. 

\begin{displayquote}
A theoretical book can solve its own problems. But a practical problem can only be solved by action itself. When your practical problem is how to earn a living, a book on how to make friends and influence people cannot solve it, though it may suggest things to do. Nothing short of the doing solves the problem. It is solved only by earning a living.
\end{displayquote}

To paraphrase Adler for your context: When your practical problem is how to be a physical therapist, a book on being a physical therapist cannot solve it, though it may suggest things to do. Nothing short of doing solves the problem. It is solved only by being a physical therapist.

While you read through the program, make markings, grasp key words, choose what to read analytically, make notes to gain information and understanding related to the reasoning process essential to being a physical therapist that requires content knowledge - keep the following in mind: 

1. Doing (action) takes place in a particular situation, always in the here and now and under a particular set of circumstances. You cannot act in general. The kind of practical judgment that immediately precedes action must be highly particular. Whereas books are a mix of general and particular, but your action as a DPT will always be particular. 

Here is what we rely on:

\begin{displayquote}
Practical books can, however, state more or less general rules that apply to a lot of particular situations of the same sort. Whoever tries to use such books must apply the rules to particular cases and, therefore, must exercise practical judgment in doing so. In other words, the reader himself must add something to the book to make it applicable in practice. He must add his knowledge of the particular situation and his judgment of how the rule applies to the case.
\end{displayquote}

2. A practical book may contain more than rules. It may state principles that underlie the rules and make them understandable. 

\begin{displayquote}
The principles that underlie rules are usually in themselves scientific, that is, they are items of theoretical knowledge. Taken together, they are the theory of the thing. Thus, we talk about the theory of bridge building. We mean the theoretical principles that make the rules of good procedure what they are.

Practical books thus fall into two main groups. Some, are primarily presentations of rules. Whatever other discussion they contain is for the sake of the rules. There are few great books of this sort. The other kind of practical book is primarily concerned with the principles that generate rules. Most of the great books in economics, politics, and morals are of this sort.
\end{displayquote}

Many of the readings you will read in this program present rules but are primarily concerned with the principles that generate the rules. Thus there are two sorts of arguments in such books. There are arguments that attempt to show that the rules are sound - similar to books that present rules. But since these books are also trying to teach you the principles that generate the rules, the major propositions and arguments will look like those in a purely theoretical book. The propositions will say that something is the case, and the arguments will try to show that it is so.
\begin{displayquote}
But there is an important difference between reading such a book and reading a purely theoretical one. Since the ultimate problems to be solved are practical—problems of action, in fields where men can do better or worse, an intelligent reader of such books about practical principles always reads between the lines or in the margins. He tries to see the rules that may not be expressed but that can, nevertheless, be derived from the principles. He goes further. He tries to figure out how the rules should be applied in practice.
\end{displayquote}

Understanding the principles that generate rules allow you, in practice, to appropriately apply the appropriate rule in a particular context. 
\begin{displayquote}
Unless it is so read, a practical book is not read as practical. To fail to read a practical book as practical is to read it poorly. You really do not understand it, and you certainly cannot criticize it properly in any other way. If the intelligibility of rules is to be found in principles, it is no less true that the significance of practical principles is to be found in the rules they lead to, the actions they recommend. This indicates what you must do to understand either sort of practical book. It also indicates the ultimate criteria for critical judgment. In the case of purely theoretical books, the criteria for agreement or disagreement relate to the truth of what is being said. But practical truth is different from theoretical truth. A rule of conduct is practically true on two conditions: one is that it works; the other is that its working leads you to the right end, an end you rightly desire. Suppose that the end an author thinks you should seek does not seem like the right one to you. Even though his recommendations may be practically sound, in the sense of getting you to that end, you will not agree with him ultimately. And your judgment of his book as practically true or practically false will be made accordingly. If you do not think careful and intelligent reading is worth doing, this book (that is the book on \textit{How to Read a Book}) has little practical truth for you, however sound its rules may be.
\end{displayquote}


\printbibliography[heading=subbibintoc]




% Introduction to the core concepts of physiology and the book
\chapter*{Introduction}
\addcontentsline{toc}{chapter}{Introduction}
Updated on \today
\minitoc


This chapter introduces basic concepts for knowing and applying clinical physiology in physical therapy practice. First we consider what we mean by clinical physiology. Second we make sure all readers have the same understanding of the basic concepts of physiology. Along the way we introduce some of the broader aspects of the approach we are taking related to clinical epistemology and Unifying Systems Theory. Clinical epistemology is bound to the use of models to summarize, test and apply clinical knowledge. Unifying Systems Theory keeps us focused on the muscle fiber act of tensioning, and organizes our presentation of the quality attributes of that act such as fidelity, efficacy and integrity, and the ubiquity of fractal recursion in physiological systems. 

\vspace{5mm}

\textbf{Objectives include:}
\begin{enumerate}
    \item Explain a muscle centered approach to clinical physiology for the practice of physical therapy
    \item Provide an example of a model that is useful in physical therapy practice.
    \item Explain the basic concepts of physiology
    \item Explain how whole system function and capacity relates to multi system function and capacity. 
    \item Provide an example of how each basic concept of physiology applies to the analysis of patient/client problems.
    \item Explain the hierarchy of adaptation and adaptability from genetic, epigenetic, anatomic, physiologic, behavioral and cultural and its relevance in physiological adaptation.
    \item Explain how the International Classification of Function (ICF) clinical physiology relates to the hierarchy of adaptation and muscle centered approach..
\end{enumerate}

\subsection{A note about objectives}

Each chapter has objectives. They include specific behavioral terms related to the expectation for the objective. There is an ordered relationship between the behavioral terms that represents increasing and nested expectations. To define is to understand the words being used, to know what they represent. Whether you can define something can be tested with a straightforward question such as identifying the correct definition from a set of definitions; whether you can write the definition; respond true or false to a proposed definition for a term; or identify the correct term from a set of terms when presented with a definition. To explain something goes beyond defining. It also implies that you can define the required terms. To explain\footnotemark{}\footnotetext{Explain is the act associated with understanding. If you understand something you can explain it. Objectives are typically written as acts - what the person learning can do if they have achieved the objective. Since understanding is observed as an act of explaining, we will use the objective "explain" when there is something you must "understand". How do you demonstrate that you understand? You explain whatever it is that you understand.} you need to simultaneously hold a set of definitions together including how they relate to one another. You can explain a model by demonstrating that you know all the definitions involved, and describe how they relate to one another, and can communicate that explanation. Whether you can explain something can be tested by asking about several aspects of what you are to explain or asking you to infer which part is missing; or by asking about one part and asking you to infer the other part. The point is, to explain goes beyond defining and has an expectation of definition. But to define does not require you to explain. To evaluate something requires you to make judgements and to infer beyond the thing you are asked to evaluate. To evaluate implies that you can define and explain the thing (or concept) you are asked to evaluate. Being tested for whether you can evaluate can include asking you to infer to or from something that has not been directly covered or discussed, going beyond the thing (or concept)\footnotemark{}\footnotetext{By a "thing" we intend a particular thing, perhaps a particular object, whereas by a "concept" we intend a universal that will contain many particulars. For example, there is a muscle fiber as a thing, meaning this particular muscle fiber and there is the concept of muscle fiber, meaning anything that has the properties of being a muscle fiber, hence universal and conceptual. This idea of the particular (concrete) and the universal (abstract) is an important concept in and of itself for your practice as physical therapists. You have both a particular patient, perhaps Sean, and you have universal patient, perhaps everyone with heart failure.} you are asked to evaluate to consider its implications in different situations. Providing an example is an objective based on a specific task and is considered a form of asking you to evaluate since it goes beyond a particular thing (or concept) to a totally new thing (or concept). Asking you to provide an example of a model requires you to know the concept of a model well enough that you can use your background knowledge to come up with something you're already familiar with that fits what know about models. That level of knowledge requires you to at least be able to evaluate a model because you have essentially evaluated your background knowledge and determined that it is a model. The chapter objective above that asks you to provide an example of a model that is useful to physical therapy practice assumes you will be able to define, explain and evaluate the concept of a model, and that you have background knowledge about physical therapy practice that you can use to evaluate models used in practice to consider what those models are and whether they are useful. 

\section{Clinical Physiology}

Physiology is a foundational science for physical therapy. Clinical physiology focuses on physiology relevant to understand health and "un" health (injury, illness, disorder, dysfunction, disequilibrium). This is a clinical physiology text for physical therapists. Since physical therapists are movement specialists, and since the physiology of muscles is integral to the movement system, our approach to clinical physiology is a muscle centered approach.  A muscle fiber\footnotemark{}\footnotetext{Muscle fibers are muscle cells, and muscle cells are muscle fibers. We do our best throughout the book to refer to muscle fibers, but if we refer to a muscle cell please know that this equates to saying muscle fiber. Similarly, when we talk about the general biological properties of cells, please know that these properties are true for muscle fibers} is a system, and muscle fibers are the structural unit of muscle. A system is whole and acts (works towards a common objective). A system has agency (ability to act to satisfy values). 

%Correct (fidelity), complete (integrity), concise (efficacy)

The act of a muscle fiber is tensioning (act of creating tension). As a system that acts by creating tension muscles are correct, complete and concise. A muscle fiber correctly acts to create tension, it is complete (has what it needs for that act), and it is concise in creating tension. A muscle fiber interacts with other systems and is supported in order to satisfy its act. Much of clinical physiology for a physical therapist considers the act of muscle tensioning, what and how a muscle fiber interacts with other systems, and what support is required for the muscle fiber to act.

When a muscle fiber acts to create tension it attains and sustains tension for movement. To satisfy this act muscle fibers interact and are supported. Therefore interactions and support are an important part of the book. Muscles get a lot of their support from the the extra cellular fluid (ECF), and maintain their ability to create tension through cellular and structural integrity.

Clinical physiology for physical therapy with considers the three interactions depicted in Figure \ref{fig:muscle_centered_approach}.  The first interaction considers the relationship between muscle cells and movement, second between movement and extra cellular fluid, and third between extra cellular fluid and muscle cells. These interactions are all support relationships. The first is an undeniable part of the movement specialist’s domain of knowledge and an area that most physical therapy students have little problem understanding how important it is for them to understand. The second and third relationship are less obvious and deserve additional explanation. The following sections explain these supporting interactions.

\begin{figure}[!ht]
    \centering
    \includegraphics[width=1\linewidth]{./figure/muscle_centered_approach.png}
    \caption{A Muscle Centered Approach \footnotesize{(Created with BioRender.com)}}
    \label{fig:muscle_centered_approach}
\end{figure}

\subsection{Muscle Fibers and Movement}

Muscle fibers are necessary but not sufficient causes for human movements. Muscles cause movement by creating tension. Active and passive tension and a whole host of details are being skipped here (such as parts, capabilities, roles and interactions) combine to create movement.

Muscle fibers create tension but cannot do this alone.  First, muscle fibers must come together as a muscle and combine their ability to create tension in order to create movement. They create tension with interaction to attachments, mostly, to bones\footnotemark{}\footnotetext{Mostly to bones because sometimes muscles attach to something other than bones such as the eye muscles or facial muscles that attach to skin to create facial expressions, or muscles that attach to themselves.} that form joints that are capable of movement. The process of coming together, and creating tension and transmitting tension to bones is covered in the next two chapters (Fundamentals and Tension). Other aspects of movement such as the mechanics and kinematics of the musculoskeletal system, or the coordination, control and learning of the neuromotor and behavioral systems, are not covered in this book. To create tension muscles need to possess the capabilities of being excited (provoked into creating active tension), being regulated to create the right amount of tension, and converting energy. Excitation, Regulation and Energetics are covered in Part II, Muscle Fidelity and Efficacy. 

\subsubsection{Muscle Fidelity \& Efficacy}

Muscle Fidelity focuses on attaining tension. The parts and capabilities of muscle fiber that allow it to attain tension when needed. Metrics of fidelity assess how well tension is attained such as measures of force.

Muscle Efficacy focuses on sustaining and transforming tension. How well does muscle sustain tension once attained, and how well does it transform tension to its intended use, typically in interaction with something outside of or beyond muscle itself. Metrics of efficacy assess how effectively (including efficiently) muscle fulfils its purpose and includes whether the muscle can sustain (continue to generate a tension) and persevere (continue to repeat the process of sustaining tension). Efficacy metrics consider time along with force, commonly referred to as endurance, or force and energetics (efficiency). 

\subsection{Extra-Cellular Fluid and Muscle Fibers}

The capability to create tension requires support. Support is received from the surroundings of the muscle fiber, the extra cellular fluid (ECF). ECF is supported and tightly regulated through the activity of several physiological systems. In this sense, as depicted in Figure \ref{fig:muscle_centered_approach}, the ECF causes (in a supportive role) muscle tension.\footnotemark{}\footnotetext{It is admittedly much more complicated than this and the use of the term causes can be challenged, but the assertion is that without the proper ECF muscle active tension cannot occur. It is true that muscle passive tension can occur with a passive stretch of the muscle but such a passive stretch is typically the product of another system putting energy into the muscle to create the passive stretch and that other system putting energy into the muscle often relies in some way on ECF.} When a muscle fiber creates tension it utilizes and exchanges resources with the ECF, thus muscle cells thus rely on and contribute to ECF. These systems that refresh and regulate the ECF are the focus Part III, Muscle Support and include the circulatory, renal, pulmonary, gastrointestinal, metabolic (liver and pancreas), and neuroendocrine systems and physical laws upon which physiological function is based (diffusion and osmosis, mass balance, flow based on gradients to name a few).

\subsection{Movement and Extra-Cellular Fluid}

The ECF contains and sustains the correct materials for supporting and supplying muscle fibers. The ECF and all the systems that support it are also supported by the ECF. The ultimate source of materials in the ECF is movement (Figure \ref{fig:muscle_centered_approach}). It may be the movement of the heart, the movement of the arterioles that regulate blood flow, the movement of the breathing muscles that move air into and out of the lungs, the movement of limbs to get food and water, the movement of the jaw and esophagus and gastric tube to digest and move food and water for absorption, the movement of the colon for waste products. These have all been examples of movements caused by muscle tension in support of the ECF that is ultimately supporting the muscle fibers causing the muscle tension.


\subsection{Muscle Integrity}
Muscle Integrity focuses on muscle fibers being muscle fibers, capable of generating tension. Muscle fibers exist to execute the act of generating tension so integrity is about the parts and capabilities that make a muscle fiber a muscle fiber. Muscle Integrity considers both ECF in support of, and movement influencing, muscle fibers. ECF supports the long term act of creating muscle tension by providing the necessary resources for the muscle to maintain its integrity (it continues to be a muscle and therefore to be able to do the act of creating tension). Movement is also involved (the arrow from movement to muscle in Figure \ref{fig:muscle_centered_approach}) because movement, as you probably know, provides (or does not provide) a stimulus for the muscle fiber. Muscle cells create movement and movement creates stress and strain on muscle fibers that act as triggers for myostasis (muscle fibers continue to be, which we will call isotrophy) or to grow (hypertrophy) and when not needed to reduce their need for resources (atrophy). Isotrophy, hypertrophy and atrophy occur on a spectrum covered in Part III Muscle Integrity. Muscle Integrity includes the cellular activity occurring within muscle fibers, and is supported by the ECF and stimulated by movement.

Metrics of Muscle Integrity include the ability to continue to do what a muscle cell does (persevere) which occurs through maintaining the various parts, capabilities and interactions. A measurement of integrity could be the cross section area of a muscle, or the lean mass of muscle, as indicators of how well the muscle (as a muscle collection of muscle fibers) is being maintained.

\subsubsection{Fractal Recursion}
The cycle depicted in Figure \ref{fig:muscle_centered_approach}, ECF causes muscle tension causes movement causes ECF, and so on, is called fractal recursion. There are many examples of fractal recursion in physiology. A muscle fiber is supported by the ECF, which is supported by the micro-circulation, which is supported by blood flow, which is supported by blood volume, which is supported by drinking, which is supported by the movements of acquiring fluid to drink, which is supported by muscles. There are numerous supported:supporting relationships nested within physiology.

%An example of fractal recursion is the Supported/Supporting pattern. The pattern repeats at any scale up or down. Perhaps in this example, start with Tensioning (Supported) / Muscle (supporting). Then we take Muscle-integrity (Supported) / ECF (Supporting). The other concept in play here is the notion of federation toward unification. So Movement as an example (Unifying objective) can be supported at the first level by bone (Counterpoising), muscle (Tensioning), tendon (Anchoring) and cartilage (Bearing). Using fractal recursion to differentiate further, muscle tensioning (Supported) is the unifying objective and ECF (Supporting) facilitates the sustaining acts of "Fueling" and "Wasting".

\subsection{Models}

Figure \ref{fig:muscle_centered_approach} constitutes a model. Models are simplified abstractions of reality. This model does not include all the details. It is not supposed to include all the details. If it included all the details it would be overwhelming. Hashing out just some of the detail requires the rest of this book. Hashing out more of the details includes most of your education. Hashing out all of the details is not currently possible and requires more than one lifetime.\footnotemark{}\footnotetext{The lifetime of many scientists has already been spent on hashing out the details that we currently know. Some may say that all the details is computationally intractable considering the depth of mechanisms at the molecular level.} Hashing out the details we continue to create and consider additional models, over and over again, models of models of models of models (fractal recursion) as we successively approximate something useful to understand and knowledge to base practice.

There are numerous models throughout this book. With each model we consider a quote by George Box: “All models are wrong some models are useful.” The muscle centered approach is our first step. Three variables and interrelationships provide a model of clinical physiology as a guide to reasoning for physical therapists in the context of injuries, conditions, and diseases. This is a useful model. There are certainly going to be situations that don’t fit the model since not all details are contained in the model. But for the task of learning clinical physiology to practice physical therapy, it's useful. 

\paragraph{More about models}

We build models in order to summarize, test and apply our knowledge \cite{collins_synthesis_2018}. Models come in many forms - graphic, causal, logical, mathematical, computational, even stories are examples of models. There are many models about muscles utilized to summarize, test and apply knowledge about muscles. For example, the sliding filament theory is a model about how muscles develop active tension featured in Chapter \ref{chp:tension} on Tension. The sliding filament theory was developed based on striated appearance (overlapping filaments) and the observation that length of muscle influenced the overlap of striations and was related to how much tension could generated when excited (length - tension relationship). The inference proceeds, if these filaments slide past each other to generate tension then overlap changes with length and should influence how much tension can be generated. So the sliding filament model proposed the length-tension relationship as a hypothesis. It then received experimental support when the length - tension relationship was observed. The muscle acted as expected if the sliding filament model was true. Based on the sliding filament model, and based on the length - tension curve (behavior predicted by the model), there are joint positions to test a muscles ability to generate maximum active tension (force). This model of muscle, sliding filaments and length - tension, is applied in practice. It is a model of muscle that is true for all striated muscle (including cardiac muscle, striated but differentiated from skeletal muscle).

This book and your education and practice is full of models that summarize, test and apply knowledge of muscles and physiological systems that support them. Just remember, models are abstractions of reality, not reality. "All models are wrong, but some models are useful" (George Box). Compared to reality all models are limited. Always ask We must whether the model being applied is useful - useful to test what we think we know, useful to summarize what we know, and useful to utilize what we know.

\subsection{Clinical Physiology Summary}

Clinical physiology from a muscle centered approach starts with a simple model. Based on this model if a PT considers muscles as a cause of movement but doesn't consider the factors that influence the ECF that is critical to the excitation, regulation, energetics\footnotemark{}\footnotetext{The convention we will utilize is that word "energetic" is an adjective, something is energetic, marked by vigor or effect, related to energy. However, adding an 's' to the end of energetic, energetics changes it to a verb, the process of converting energy and should not be confused as a plural form of the adjective energetic.} and integrity of the muscle fibers then they are ignoring a critical component of the supported:supporting relationship. If a PT considers how the ECF influences muscle fibers and thus movement and does not consider how continued movement influences the ECF they may miss the opportunity to stop repeated hospital admissions.\footnotemark{}\footnotetext{Such as readmissions due to repeated breakdowns in a person that goes home, cannot move enough to sustain hydration or nutrition or ventilation and returns to the hospital dehydrated, malnourished and with pneumonia \cite{collins_heart_2015}} Considering these three supported:supporting fractal recursive variables is a first step to thinking about clinical physiology with a muscle centered approach for the purpose of practicing physical therapy.

\section{Basic Concepts of  Physiology}

The Basic Concepts of Physiology provide a unifying framework that we apply throughout the book. They are a summary of the pre-requisites for reading this book, which is aimed at physical therapy students (graduate level) who are expected to have already be familiar with the basics of physical, chemical and biological sciences. In this section we consider each of the basic concepts provide insights of how they are useful to the physical therapist. We consider how the basic concepts of physiology are applied to the analysis of patient/client problems and are a foundation for understanding altered physiological states (pathophysiology).

\subsection{Causality}
Living organisms have causal mechanisms whose functions are explainable by a description of the cause-effect relationships that are present. Much debate exists about what constitutes a causal mechanism and relationship, and identifying as well as specifying causal relationships and models (to understand mechanisms) are an important part of physical therapy practice \cite{collins_synthesis_2018}. Accepting that these causal mechanisms are explainable is a philosophical assumption, just because many cause-effect relationships are explainable we still must deal with what is called the problem of induction. We attempt to solve, or at least manage, the problem of induction through rigorous manipulation and observation (experiment) or systematic observations (epidemiology) and subjecting these observations to statistical inference. Regardless of of what we know, or can know, about causal relationships and mechanisms based on what we observe, is the assumption that underlying any effect we observe there is a cause, or a set of inter-related causes. This is an essential aspect of causality as a basic concept of physiology and we will be applying this basic concept every step of the way.
\paragraph{}
Causality is applied to the analysis of patient/client problems through the very fact that if we observe an effect we infer some underlying cause. If there is an alteration in blood pressure, glucose or sodium we infer something is the cause of those alterations. Inference from effects to causes is called abduction, and when there are multiple possible causes from a set of effects, is called inference to the best explanation. The statistical inference used to calculate the probabilities of inferences from causes to effects is called Bayesian inference and it is the foundation of any diagnostic process. 
\paragraph{}
Muscles generate tension. Tension is an effect. Part I considers the causes of that tension. If muscles generate too much or too little tension (an effect that is noticed as muscle stiffness or muscle weakness) then we consider the causes of the tension and what may be wrong, why are they not currently causing the right amount of tension? 

\subsection{Cells}

\subsubsection{Cell Theory}

Cell Theory states that all cells making up an organism have the same DNA. Cells are considered the basic unit of life. All of the specialized functions we cover in this book are based on differentiation and the subsequence specialization of cells throughout the body.   
\paragraph{}
Cell Theory is applied to the analysis of patient/client problems through the observation of the health and well being of cells as an indication of whether there are problems, first with the cell, and second as a possible abnormal cause. For example, any cells that divide at a rapid rate (which itself is the result of DNA expression specific to the specialization of the cell), such as skin cells, have a greater chance of DNA mutation during cell division. A DNA mutation during skin cell division can result in a variety of benign or malignant cancerous cells that appear on the surface of the skin. The presence of a mole, particularly one that is not circular or has rough edges, is a possible sign of altered DNA of those cells creating cancerous cells. A benefit to having a high rate of cell division is ability of these cells to replaced dead or damaged cells resulting in faster healing and complete repair as opposed to an adaptive repair.

\paragraph{}
Extending the concept of the rate of cell division further is the general principle that areas of the body with lower rates of cell division (chondrocytes, cells that spawn cartilagenous material), or sometimes so low to be observably absent (some nervous system cells), are known to have slower or absent rates of healing and repair. When we consider osteoarthritis as a "wear and tear" of the joints we are saying that the rate of wear and tear exceeds the rate of healing and repair, which are determined in large part by the rate of cellular division in those specialized cells. Whether a supplement such as chondroiten (a molecule found in cartilage) helps treat the symptoms of osteoarthritis is dependent on whether the cause of the imbalance between damage and repair has something to do with not having enough chondroiten, or whether chondroiten does not just provide material, but is coupled in some way with increasing the repair process.
 
\subsubsection{Cell Membrane}

The boundary of a cell is delimited by a semi permeable plasma membrane, the cell membrane. In a muscle fiber it is the sarcolemma. The sarcolemma is a complex structure that determines what substances enter or leave the muscle fiber. It is the boundary between the inside of the cell (cytoplasm, for a muscle fiber the sarcoplasm) and the outside of the cell (ECF). The sarcolemma is essential for cell signaling, transport and other processes. The proteins (receptors, channels, pumps) embedded in the sarcolemma influence how permeable the membrane is to certain substances, and thus how quickly those substances can enter (or exit) the cell. Many of the adaptations to muscle fibers to exercise training are related to changes in the proteins of the sarcolemma that allow certain muscle acts to occur more rapidly - such as excitation for example. To increase the rate of excitation the sarcolemma can include additional sodium/potassium pumps, more sodium and potassium channels that allow faster excitation cycling (discussed in much more detail in Chapter \ref{chp:excitation} on Muscle Excitation).

\paragraph{}
The cell membrane is applied to the analysis of patient/client problems in a rather fundamental way. Whether a cell is a cell includes whether it has a boundary. If damage to the cell membrane is complete, the cell no longer exists, it is dead. The assessment of cardiac cell death (infarction) includes whether certain proteins that are found in the cardiac muscle are in the blood. The reasoning is that for those proteins to be in the blood, they had to escape the cardiac muscle and that can only happen if the cell membrane is irreversibly damaged, and if the cell membrane is irreversibly damaged then the cell is dead. Therefore, if there is Troponin I or T in the blood, or Creatine Kinase - Myocardial Band (CK-MB) in the blood, then some sort of cardiac muscle fiber infarct has likely occurred.

\subsubsection{Cell-Cell Communication}

This may seem like a nit but what if you inverted the order/emphasis of this statement. The expression of a function by a federating (multi-cellular organism) requires the ability to coordinate individual action typically utilizing the most concise means. Just a thought.

The function of any multi-cellular organism requires the ability to coordinate individual action typically using the most concise means. This can be achieved by cells passing information to one another through "cell-cell communication". These communication processes include endocrine (hormones delivered system wide in the blood which alert all muscle fibers receiving blood flow that have a receptor within their sarcolemma) and neural signaling (which tends to more specifically target particular muscle fibers). 

\paragraph{}
Cell-cell communication is applied to the analysis of patient/client problems in any situation where endocrine or nervous system interactions with the cells may be abnormal. The release of insulin signals to cells that they should increase their glucose uptake. If cell-cell communication is impaired, as with Type II Diabetes, and the insulin receptors on the cell do not adequately receive that communication then cells will not uptake more glucose in response to insulin and blood glucose values will increase. One possible cause of an elevated blood glucose is a problem with cell-cell communication of the insulin receptors not responding to insulin. 

\subsection{Genes to proteins}
Genes (DNA) code for the synthesis of proteins (muscle filaments and enzymes are proteins). This is not all the genome\footnotemark{}\footnotetext{All the genes together} or the epigenome\footnotemark{}\footnotetext{Above the genome, the meta genome}, does, but it is an important part of muscle cell differentiation, specialization and myostasis. The genes in a particular cell that are expressed determine the functions of that cell (i.e. make it a muscle fiber and not, say, a liver cell), and additionally can make it a particular type of muscle fiber (i.e. a fast or slow twitching muscle fiber). Genes to proteins are an important component of Part III: Muscle Integrity.

\paragraph{}
Genes to proteins is applied to the analysis of patient/client problems whenever we consider the impact of cancer (mutations in DNA during division that then influence the proteins being created) on cellular function. Genes to proteins is also the fundamental process by which muscle fibers are maintained (myostasis - isotrophy, staying the same) or adapt (myostasis - hypertrophy or atrophy). Mechanisms generating tension in the muscle fiber are dependent on proteins, and the proteins are produced during the translation and transcription of DNA. 

\subsection{Energy}
The life of the organism requires the constant expenditure of energy. The acquisition, transformation, and transportation of energy are essential functions. Experimental approaches to understand muscle are focused on muscle fibers as energy conversion machines, taking chemical energy and converting it into mechanical energy \cite{woledge_energetic_1985}. 
The earliest of these studies measured work and heat based on the physical law of the conservation of energy. Energy is neither created or destroyed. Energy is transformed from one form to another. When a muscle fiber breaks down adenosine triphosphate (ATP) to generate tension, as discussed in Chapter \ref{chp:energetics} on Muscle Energetics), the energy of that chemical bond is transformed to mechanical work (force x distance) or as heat \cite{hill_heat_1938}.

\paragraph{}
Energy is applied to the analysis of patient/client problems since life is dependent on the constant expenditure of energy, and therefore requires ongoing acquisition, transformation, and transportation. If muscle fibers lack the fidelity and efficacy to allow movements that enable the acquisition of energy (consumption of nutrients), then muscle will subsequently lack integrity (atrophy, loss of muscle mass) due to the organisms continual need to acquire energy. Muscle integrity is sacrificed in situations of malnutrition or starvation, even when that malnutrition or starvation is caused by the inability to move. 

\subsection{Mass Balance}
The quantity of ”stuff” in any system, or in a compartment of a system, is determined by the inputs into the system and the outputs from that system or compartment. This concept is based on the law of conversation of mass. When someone consumes 50 grams of carbohydrate (CHO) those 50 grams are then part of them and will be processed (digested, metabolized and either eliminated, stored or otherwise utilized). 

\paragraph{}
Mass Balance is applied to the analysis of patient/client problems any time we consider the amount of some physiological variable, and this is ubiquitous in practice. Blood volume of all of the components that make up blood volume must be there in just the right mass balance for the ECF to have just the right mass balance for the muscle fibers to have just the right mass balance. When blood work is performed in a lab for a patient it is a general check on whether mass balance is being maintained on a whole host of components of blood volume.

\subsection{Flow Down Gradients}
The transport of “stuff” (ions, molecules, nutrients, blood, and gas) is a central process at all levels of organization in the organism, and a simple model (Ohm’s Law) describes such transport. 

Ohm's Law describes the relationship between current (flow of electricity), voltage and resistance:
\begin{equation} 
\label{ohms_eq}
Current = \frac{Voltage}{Resistance}
\caption{Ohm's Law}
\end{equation}

For fluid transport, such as blood through vessels and air through the airways we consider:
\begin{equation} 
\label{flow_eq}
Flow = \frac{Pressure}{Resistance}
\caption{Ohm's Law applied to Flow}
\end{equation}

For diffusion across membranes we consider:
\begin{equation} 
\label{diffusion_eq}
Diffusion = \frac{Concentration}{Resistance}
\caption{Ohm's Law applied to Diffusion}
\end{equation}

In all of these examples, which come up frequently through the book, current, flow and diffusion are proportional to the voltage, pressure or concentration gradient (numerator), and inversely proportional to the resistance (denominator). This is an essential concept. This model can be expanded for additional context and explanation. For example, considering the factors that influence resistance of blood flow (expanding the denominator) provides additional context and explanation for the regulation of cardiac output. 

\paragraph{}
Flow Down Gradients is applied to the analysis of patient/client problems any time we are considering pressure (blood pressure, oxygen partial pressure, etc) or concentrations (sodium concentration in the ECF). Disturbances in these pressures or concentrations fundamentally disturbs the flow that is necessary for normal physiological function.

\subsection{Homeostasis}
The internal environment of an organism is actively maintained constant by the feedback function of cells, tissues, and organs organized into primarily negative feedback systems.\\
Important points about homeostasis:
\begin{itemize}
    \item Not everything is regulated (i.e. heart rate)
    \item Homeostasis is not an on/off switch
    \item ”Relatively constant” – meaning there can be acceptable variation
    \item Set points can change
    \item There is a hierarchy of homeostatic regulation
\end{itemize}

The hierarchy of homeostatic regulation includes the fact that cells manage their internal environment (homeostasis), which includes taking and giving molecules and water to the ECF, the ECF is kept in homeostasis via exchange  with the blood volume (intra-vascular ECF) and homeostasis of the blood volume (water, cells, electrolytes, nutrients, etc) is exchanging with organ systems (lungs, liver, pancreas, gut). A simple graphic model of glucose homeostasis is shown in Figure \ref{fig:glucose_homeo} and shows the interactive relationship between glucose, glucagon and insulin.

\begin{figure}[!ht]
    \centering
    \includegraphics[width=1\linewidth]{./figure/glucose_homeo.png}
    \caption{Simple Model of Glucose Homeostasis that does not consider the organs or cellular mechanisms underlying the homeostatic relationship between the three components. In this system, glucose is being regulated. \footnotesize{(Created with BioRender.com)}}
    \label{fig:glucose_homeo}
\end{figure}

Figure \ref{fig:complete_glucose_homeo} A more complete model of glucose regulation starts by including the multiple paths by which glucose can be regulated and the role of metabolic organs such as the liver and the endocrine pancreas.

\begin{figure}[!ht]
    \centering
    \includegraphics[width=1\linewidth]{./figure/complete_glucose_homeo.png}
    \caption{Model of Glucose Homeostasis that includes the organs involved in this homeostatic system \footnotesize{(Created with BioRender.com)}}
    \label{fig:complete_glucose_homeo}
\end{figure}

\paragraph{}
Homeostasis is routinely applied to the analysis of patient/client problems. Many diseases, conditions and syndromes, regardless of their underlying cause, results in some sort of disturbance to homeostasis; and thus any disturbance in homeostasis is ultimately analyzed for the possible causes.

\subsection{Interdependence}
Cells, tissues, organs, organ systems interact with one another (are dependent on the function of one another) to sustain life. Interdependence is includes nested and fractal recursive hierarchies of homeostatic regulation (supported:supporting relationships). The function and capacity of the "whole system" (the word for health comes from the word for whole) is related to the interdependent relationship between the parts, capabilities and attributes of each of the multiple systems acting together in supporting and supported roles. 

\paragraph{}
Interdependence is applied to the analysis of patient/client problems during the process of differential diagnosis. This process includes asking the "inference to the best explanation" question - "What are all the things that could cause this problem?" Given the interdependence of the whole system (the whole person), what contributes a list of potential causes for the observed effects?

\subsection{Structure - Function}
The function of a cell, tissue, or organ is determined by its form. Structure and function (from the molecular level to the organ system level) are intrinsically related to each other. Structure influences function (right now) and function influences structure (eventually). Much of Part II Muscle Fidelity and Efficacy is how structure influences structure right now; and Part III Muscle Integrity includes how function influences structure eventually (hypertrophy and atrophy).

\paragraph{}
Structure - Function is applied to the analysis of patient/client problems in the process of diagnosis of pathoanatomical conditions and also is the foundation for pathokinesiological interventions. If function is impaired, we consider the structure that may result in such an impaired function. For example, if there is an deviation in the gait pattern (function), we may consider the length of the hip flexors (structure impacts function). Also, if there is a functional movement deviation, prolonged sitting (function), we may teach corrective exercises to prevent future changes in structure (shortened hip flexors) that become dysfunctions (function impacts structure eventually).

\subsection{Hierarchy of Adaptation}

There is a hierarchy of adaptation and adaptability working within and around physiological systems. Adaptation refers to changes that occur to the system so that it optimizes fidelity, efficacy and integrity when completing its act. A muscle fiber act is creating tension. Adaptation refers to changes that occur in a muscle fiber to meet the needs of applying tension (fidelity), transforming the muscle fiber to meet the needs of applying tension in a more efficient way (efficacy), and changes to the muscle fiber that allow it to continue to be a muscle fiber (integrity). Adaptability refers to the ability to adapt. The hierarchy includes genetic, epigenetic, anatomic, physiologic, behavioral and cultural changes. Cultural changes being a mechanism of passing behavioral adaptation from one generation to another. This entire hierarchy is involved through the life span and has relevance to physiological adaptation. Very clearly all of those that are within connect directly to physiological adaptation (genetic, epigenetic, anatomic, physiologic). Behaviors are enabled by our physiology, but then also influence our physiology (our physiology allows us to eat sugar, and then our eating of sugar influences our physiology). And the culture around us gives rise to acceptable, and not acceptable, behaviors. 

\paragraph{International Classification of Function (ICF)}
Physical therapists incorporate the hierarchy of adaptation with the World Health Organization's International Classification of Function (ICF), see Figure \ref{fig:icf}. The ICF is a model that describes how physiological systems contribute (as body systems and functions) and interact with functioning (or dysfunction or impairments). The ICF model focuses on the bidirectional interactions between bodily functions (and structures), activities (movements and behaviors) and participation (behaviors and interactions). It also considers to health conditions that impact functions, activities and participation along with external and personal factors. 

\begin{figure}[!ht]
    \centering
    \includegraphics[width=1\linewidth]{./figure/ICF.png}
    \caption{International Classification of Function (ICF) \footnotesize{(Created with BioRender.com)}}
    \label{fig:icf}
\end{figure}

The ICF is a useful model for physical therapy practice. It is useful for understanding physiological interdependence, structure - function relationships, and the hierarchy of adaptation. The middle of the ICF is \textit{Activity}, which is primarily referring to movement, making the ICF a movement centered approach. The model considers what is necessary for movement, body structure and functions; and why we do movements, to participate. In this way the ICF overlaps with our muscle centered approach as depicted in Figure \ref{fig:muscle_centered_approach_icf}

\begin{figure}[!ht]
    \centering
    \includegraphics[width=1\linewidth]{./figure/muscle_centered_approach_icf.png}
    \caption{Muscle centered approach overlap with the ICF) \footnotesize{(Created with BioRender.com)}}
    \label{fig:muscle_centered_approach_icf}
\end{figure}

\section{Chapter Summary \& Next Steps}

Clinical physiology with a muscle centered approach for a physical therapist is based on the relationship between muscle fibers, movement and extra-cellular fluid. The muscle centered approach considers muscle fibers as crucially necessary, but not sufficient, causes of movement. Movement is a complex multi-system physiological function and capacity through its contribution to the ECF (Figure \ref{fig:muscle_centered_approach}). Movement is critical to the health and well being of the body given that movement is critical to the ECF, and the ECF is critical to all cells in the body. The muscle centered approach is a \textit{middle out} approach to understanding the complexity and hierarchy of physiology for the physical therapist as a movement specialist. By \textit{middle out} we refer to the entire approach as a model of clinical physiology that looks upward in scale (movement as a behavior within a culture) and looking downward in scale (to the genes).

\paragraph{Next Steps}
Part I is focused on Muscle Fidelity \& Efficacy, what it takes for muscle to attain, sustain and persevere in the act of tensioning. Part II is focused on Muscle Support, what it takes to support muscle fibers in the act of attaining, sustaining and persevering in the act of tensioning. Part III focuses on Muscle Integrity, how do muscle fibers persevere and maintain under steady state and variable conditions and in the face of threats to integrity (damage, injury and illness).  

\printbibliography[heading=subbibintoc]
% !TEX root = ../notes_template.tex
\chapter{Fundamentals}\label{chp:fundamentals}
Updated on \today
\minitoc

This chapter covers fundamentals of muscle structure and function. It defines terms and explains concepts used through the book. It is expected that anyone reading this book has a background that includes Anatomy \& Physiology. Skim or even skip sections that you have already mastered. The book attempts to balance the depth and breadth of clinical physiology with a muscle centered approach for physical therapy. This means the book is not the deepest nor the broadest coverage of each topic. There are areas that you have gone deeper, but we may take a broader perspective. There are areas where you are comfortable with the breadth, but we may take a deeper perspective. Physical therapists balance depth and breadth (look down, look up and beyond) during examination, evaluation and interventions. The PT may consider the molecular effect of calcium for muscle contraction, its role within the supportive matrix of bone, its absorption and distribution through the body, and whether someone is physically capable of independently obtaining and preparing meals that provide sufficient calcium in their diet. The terms - deeper and broader - are relative.

\vspace{5mm}

\textbf{Objectives include:}
\begin{enumerate}
   \item Explain muscle \textit{in situ} .
   \item Explain muscle fidelity and efficacy. 
    \item Describe and explain the importance of the macroscopic muscle scaffold including connective tissue, tendons and bony attachments.
    \item Explain what is meant by attaining and sustaining tension.
    \item Explain the difference between active and passive tension.
    \item Describe and explain the implications of muscle pennation.
    \item Explain the contexts that result in $\Delta$ velocity during active tension (concentric, eccentric, isometric).
    \item Explain roles of muscles such as agonist, antagonist, synergists and stabilizers.
    \item Explain the relationship between of muscle active tension, muscle force, muscle torque and movement.
    \item Provide of an example of a free body diagram including the relevant forces to generate torques and which relationships between the torques produce different contraction types and movements.
\end{enumerate}

\section{Muscle Fibers and Muscles}
Through the book we take a muscle centered approach. Most of the time what we mean is a muscle fiber centered approach. As a clinical physiology book our emphasis is on the cellular events occurring in the muscle fiber to fulfill its act of tensioning; the fidelity, efficacy and integrity of that act; and how that act is supported by integrated physiological function. We cannot escape the reality that while muscle fibers are the basic unit of the muscle system, each muscle fiber constitutes one cell of an entire muscle. It is the muscle as a structural and functional entity that physical therapists may think about more often, whether the biceps femoris muscle is too tight or too weak, for example. We cannot lose sight of the fact that as a physical therapist you also need to be able to recognize and investigate and relate what is happening in that muscle to what is happening in its cells, and how the body is supporting those cells. The muscle includes muscle fibers and connective tissue. A single muscle fiber, alone, has a limited functional role in the moving human. It is in the collective action of muscle fibers, as a muscle, that muscle fibers play an integral functional role in the moving human. To be organized and to act as an effective collective, muscle fibers are bound together by connective tissue, and through connective tissue are attached to bones or other anatomical structures. 

\paragraph{Muscles \textit{in situ}}
The muscles that you know (biceps femoris, deltoid, gastrocnemius, extensor digitorum longus, etc) are classified and identified (named) by shape and location. Shape is determined by how the muscle fibers are connected together with connective tissue. Variations in shape occur primarily from variations in how the muscle fibers are connected together and arranged. Similar to how you can construct many different shapes with a set of similarly shaped bricks. The location of a muscle is determined by what it interacts with based on its tendon attachments. The muscles you learn in anatomy and consider as part of a physical exam when palpating and testing are muscles that are identified in their original shape and in their original location. This is the muscle \textit{in situ}.\footnotemark{}footnotetext{\textit{in situ}, adverb or adjective, meaning in original position}

The act of a muscle fiber is to create tension. Creating tension occurs in the context of the muscle fibers connected together, and in interaction with other structures through tendon connections, \textit{in situ}. Throughout the book we will center on the muscle fiber as the basic tension creating system. However, we cannot lose contact with muscle \textit{in situ} so we need to know how muscle fibers come together as muscles, and how they transmit tension to bones, and how the muscle \textit{in situ}  impacts the muscle fiber.

\section{Muscle Fidelity \& Efficacy and the Act of Tension}

The muscle fiber acts to create tension and collectively muscle fibers transmit this tension to muscles \textit{in situ}. Muscle fidelity refers to the ability of, and how well, the muscle fiber and muscle attain tension. Attain tension simply means creating tension. How much tension can be attained is often measured as force. Muscle fidelity is a quality referring to the internal capabilities and parts that allow muscle fibers and muscles to attain tension. We consider muscle fidelity at the level of the muscle fiber and at the level of the muscle. 

Muscle efficacy is a quality of how well muscle sustains and transforms tension to its intended use. Sustains tension simply means keeping a certain amount of tension in the muscle. Sustaining tension typically includes consideration of how much tension has been attained, and then integrating over time. It is measured as how long an attained amount of tension is sustained. What this means is that you must consider what has been attained in order to consider whether it can be sustained. You must attain to sustain. Muscle fibers and muscles attempt to fulfill the act of tension as efficiently as possible, and by attaining tension efficiently tension can be sustained. Efficacy is held in balance with fidelity and we can consider the efficacy of fidelity, and the fidelity of efficacy, but that need not slow us down at this point. For now, efficacy refers to how efficiently tension can be created which influences sustaining tension and this is balanced with how well tension is created, which influences attaining tension. The balance between fidelity and efficacy is notable in later chapters when we consider differentiation of muscle fibers. Some muscle fibers are optimized to attain tension at the expense of sustaining tension, and others are optimized to sustain tension at the expense of attaining tension. 

\subsection{Tension}

The act of muscle fibers and muscles is to create tension. For a definition of tension we can turn to physics: 

\begin{displayquote}
In physics, tension is described as the pulling force transmitted axially by the means of a string, a cable, chain, or similar object, or by each end of a rod, truss member, or similar three-dimensional object; tension might also be described as the action-reaction pair of forces acting at each end of said elements. Tension is the opposite of compression.\footnotemark{} \footnotetext{\url{https://en.wikipedia.org/wiki/Tension_(physics)}}
\end{displayquote}

Tension creates a pulling force between two objects. For a muscle fiber those objects are connective tissues connected to other muscle fibers serially and in parallel, or to connective tissues that connect to tendons. For a muscle those objects are bones, with tendons forming a transitional bridge between muscle fibers and bone. An important feature of the physical concept of tension is that it is related to a pulling on two objects. We can talk about pulling a rope between two objects (there is a force involved). If the objects continue to move closer to one another (to approximate each other) and the slack rope remains in place then the rope may start to compress and limit further approximation of the objects, but that is not tension, it is compression. The act of a muscle fiber is tension, not compression.\footnotemark{} \footnotetext{There may be particular muscles that compress at certain joint ranges of motion (ROM). For example, passive elbow or knee flexion can be said to have a "soft end feel" because the slackened flexion muscles are compressed which limit further flexion. A soft end feel occurs when soft tissue compression limits further ROM.}

\section{Connective Tissue Scaffold}

Building a muscle from muscle fibers requires an intricate connective tissue scaffold that connects and shapes muscle fibers. This scaffold also functions to transmit tension between muscle fibers and tendons.  This connective tissue binds muscle fibers into what we see and define as a muscle. Three connective tissue compartments connect and organize muscle fibers so that the tension created by and transmitted to the muscle fiber interacts as required with the muscle \textit{in situ}. Figure \ref{fig:muscle_scaffold.jpg} shows the hierarchical structural relationship the endomysium, perimysium and epimysium. The \textbf{endomysium} is thin layer of connective tissue that surrounds each muscle fiber. The endomysium should not be confused with the sarcolemma which is a cell membrane and also surrounds each muscle fiber. The \textbf{perimysium} is a thicker layer of connective tissue that surrounds groups of muscle fibers and forms a \textbf{fascicle}. The muscle fibers in a fascicle are arranged in parallel (and thus influences a muscle's cross sectional area (thickness). The arrangement of multiple fascicles influences both the muscle length (if fascicles are arranged in series) and pennation (if fascicles are arranged at an angle to the overall muscle's line of tension through its tendons. The \textbf{epimysium} surrounds all the fascicles to form a muscle belly (readily observed on visual observation). A muscle belly may or may not be distinguished from a muscle. This depends on context. For example, the deltoid muscle is known to have separate parts (anterior, medial and posterior), each of these parts has it's own epimysium and is thus a separate belly. The point is that the epimysium does not necessarily surround the muscle as you know it and identify it (i.e. if you identified the deltoid you'd be identifying three separate muscle bellies that are each encased in an epimysium connective tissue wrap). There are additional connective tissue wraps that do surround muscles as you know them (superficial to the epimysium), and there are additional connective tissues that surround ever more superficial layers that result in muscles being bound to one another (i.e. fascia and fascial trains), but these levels of connective tissue encasement is not covered in this book. A reminder of the fascia layer and fascial trains is easy to experience with the difference you feel between a hamstring stretch with ankle dorsiflexion vs. plantar flexion.

\begin{figure}[!ht]
    \centering
    \includegraphics[width=1\linewidth]{./figure/muscle_scaffold.jpg}
    \caption{Connective Tissue Scaffold of Muscle \footnotesize{(Public Domain figure from \href{https://commons.wikimedia.org/wiki/File:Illu_muscle_structure.jpg}{Wikimedia Commons})}}
    \label{fig:muscle_scaffold}
\end{figure}

\subsection{Roles of the Connective Tissue Scaffold}

The endomysium, perimysium, epimysium connective tissue scaffold have a role in connecting, containing and arranging muscle fibers. They are also fundamental in the transmission of tension and to and through tendons to their attachments \cite{turrina_muscular_2013}; and to secure the physical location and approximation of the neurovascular supply to the muscle fibers.

\paragraph{Transmit tension}

Each muscle fiber transmits tension to one another and through the connective tissue scaffold. To transmit tension the connective tissue must securely, and with minimal elasticity, connect each muscle fiber to each other and to the tendon. The endomysium and perimysium tend to be thinner and contain both collagen and reticular fibers. Whereas the epimysium tends to have a greater volume of collagen (less elastic). 

The transmission of tension from muscle fiber to tendon to bone includes two zones of transition, muscle-to-tendon, and tendon-to-bone. The muscle-to-tendon zone is referred to as the \textbf{musculotendonous junction (MTJ)}. The MTJ is where muscle fibers start to diminish and muscle connective tissue, primarily epimysium starts to increase and meet the collagenous tendon tissue. From the muscle fibers toward the tendon the MTJ includes multi directional tendon projections gradually reorienting to being longitudinally oriented tendon tissue which allows a uniform transmission of tension. You can think of the multidirectional projections as part of the MTJ's role of integrating and summing the muscle fiber tension to be focalized in the tendon \cite{knudsen_human_2015}. Tendons consist of dense connective tissue that allows them to absorb and transmit substantial tension. Tenocytes (cells that make tendons) account for about 20\% of tendon volume, and the extracellular matrix (ECM) accounts for about 80\% of tendon volume \cite{kjaer_role_2004}. 

The tendon-to-bone zone is referred to as the tendon-bone attachment. The formation of an attachment between tendon and bone starts during embryonic development. During musculoskeletal system assembly, the tendon–bone attachment  forms a structure that is mechanically complex given the need to transfer tension between two materials that differ greatly in stiffness (tendon being non mineralized connective tissue, and bone being mineralized connective tissue).    

\paragraph{Secure the neurovascular supply}
The connective tissue scaffold secures the neurovascular supply to muscle. By secure we mean literally makes sure that the neurovascular supply stays were it is supposed to stay. By entering through epimysium and perimysium connection tissue the neurovascular structures (nerves, arteries, veins) are secure in their path regardless of what length a muscle belly may take. They are then embedded in the endomysium in close proximity to the muscle fiber with which they  interact regardless of the length of the muscle fiber. The vascular supply is delivered by arteries and arterioles, and removed by veins and venules, that provide blood flow that penetrates the epimysium and perimysium to reach the muscle fibers. Capillaries then surround each muscle fiber embedded and anchored in the endomysium to interact and exchange mass with the muscle extra-cellular fluid. Innervation of muscle for both excitation and regulation by the nervous system comes from branches of peripheral nerves that penetrate the epimysium and perimysium and terminate in and are anchored by the endomysium at the required location for each muscle fiber (at the neuromuscular end plate and further discussed in Chapter 3 on Muscle Excitation).


\section{Muscle Mechanics}

Chapter 2 on Tension focuses on the mechanisms internal to a muscle fiber for generating tension. Our inclusion of muscle fiber mechanics here is a simple first pass to serve the purpose of considering some basic topics of muscle mechanics such as passive and active tension, pennation, and the velocity of $\Delta$ length\footnotemark{}\footnotetext{$\Delta$ length refers to changes in length} during active tension.

\subsection{Active \& Passive Tension}

Tension created by a muscle fiber can be either active or passive. Active tension is generated when the muscle fiber transform chemical into mechanical energy. The mechanical energy of active tension pulls the ends of a muscle fiber together, it attempts to shorten the muscle fiber and through transmission of that tension to the connective tissue scaffold it attempts to shorten the entire muscle (bring the tendon attachments together). Passive tension is generated when another force attempts to lengthen a muscle fiber. When another muscle, or some other force attempts to pull the tendon attachments apart, which transmits tension to the muscle fibers and they lengthen without attempting to transform chemical energy into mechanical energy. 

\subsection{Pennation - Fascicle Arrangement}

Fascicle arrangement includes whether the fascicles are organized (mostly) parallel to the overall tendon line of pull, or whether there is an angle between the fascicle and the overall tendon line of pull. Figure \ref{fig:pennation} shows an arbitrary muscle with a line of pull of the whole muscle along the overall tendon line in red $F_{wm}$; and the line of pull of the muscle fibers bundled in fascicles at angle $\alpha$ in orange $F_{fiber}$. When there is an angle, such as $\alpha$, the muscle is described as being pennnated. Another way to say this is that pennation exists when there is an angle between the line of pull of the fascicles and the line of pull of the tendons.\footnotemark{}\footnotetext{Note, if the sum of all $\alpha = 0$ from one attachment to the other attachment then the muscle is not pennated. For example even in long muscles such as the biceps brachi there is, at any point along the muscle, an $\alpha$, but from one end to the other they cancel each other out because at either end they are angled in the opposite direction.} Note from Figure \ref{fig:pennation} that the line of pull of the tendons is usually the vector sum of the lines of pull of the fascicles. With pennation there is a vector component of the muscle fiber that acts in line with the tendon pull, and there is a vector component that does not work in line with the tendon pull. Typically the vectors that act in line with the tendon pull sum together; whereas the vectors that do not act in line of the tendon pull cancel each other out.\footnotemark{}\footnotetext{There are situations where the vector components not in line with the tendon line of pull do not cancel out, however in such instanes there is usually another structure that creates the forces to properly counter balance those forces}  Fascicle arrangement is - in large part - one of the primary features that distinguishes various classifications of muscles; and the unique fascicle arrangement, along with shape, even within a classification can help distinguish particular muscles from one another.


\begin{figure}[!ht]
    \centering
    \includegraphics[width=1\linewidth]{./figure/pennation.png}
    \caption{Pennation of Muscle Fibers \footnotesize{(Public Domain figure from \href{https://commons.wikimedia.org/wiki/File:Pennation_angle_of_fibers_in_pennate_muscle.png}{Wikimedia Commons})}}
    \label{fig:pennation}
\end{figure}

Fascicle arrangement is a muscle \textit{in situ} factor that influences the velocity of shortening of the muscle per unit of shortening of a muscle fiber; and the muscle force produced, per unit of force of a muscle fiber. Overall it is not a modifiable aspect of a muscle, meaning there are no training (stretching, pulling, needling, strengthening) programs that alter a particular muscles pennation status (whether it is pennated or not). Though there is evidence of training influencing, in small degrees, the angle of pennation \cite{cuthbert_effect_2020}.

\paragraph{Velocity of Shortening}
 
If two muscles, one pennated and one not pennated, had a set of homogeneous muscle fibers that all shortened at the same velocity during active tension with no resistance. The muscle without pennation would shorten with a higher velocity. This occurs because the shortening of each fiber is summed to the whole muscle. If each fiber shortens to 50\% of its length in one second, then the muscle as a whole would shorten to 50\% of its length in one second. However, in a pennated muscle, when each fiber shortens to 50\% of its length in one second, the muscle as a whole shortens at about 50\% $\times \cos(\alpha)$ (recall that $\alpha =$ angle of pennation). For example, if $\alpha = 35^\circ$ then with all muscle fibers shortening 50\% in 1 second there would be about 40\% in 1 second.

\paragraph{Force of Active Tension}

For reasons you may already know, and that we will cover in Chapter 2 on Tension, the amount of force that a muscle can develop during maximal active tensioning is proportional to the cross sectional area of the muscle. The force of the whole muscle is proportional to the number of parallel muscle fibers that are actively generating tension. Note the stipulation that these fibers must be parallel. Adding muscle fibers in series (which make a muscle longer but not thicker) do not contribute to the force generated by the entire muscle (at its ends) because each muscle fiber, indeed each of many elemental units within the muscle fiber, pull from both ends when generating active tension and thus some of the force that they generate, in series, is cancelled out. In Figure \ref{fig:Sacromeres_series_parrallel} there are three fibers in series (A) and in parallel (B). When in series the forces from 1 and 2 cancel each other out and the force of 3 is transmitted to the the ends. When in parallel the forces of the three sum. Note that the muscle fibers in parallel increase the cross sectional area.

\begin{figure}[!ht]
    \centering
    \includegraphics[width=1\linewidth]{./figure/Sarcomeres_series_parrallel.png}
    \caption{Muscle Fibers in Series \& Parallel \footnotesize{(Created with Biorender.com}}
    \label{fig:Sacromeres_series_parrallel}
\end{figure}


Pennation tends to increase the cross sectional area (the number of muscle fibers in parallel). The increase in cross sectional area is enough to compensate for any loss in force associated with the vector sum. Similar to the loss of velocity in a pennated muscles, not all of the muscle fiber tension is transmitted to the line of pull of the tendon and thus is lost. But, given the additional fibers in parallel it makes up for the loss due to the vector sum. For example, if a muscle fiber can generate 1 unit of force and we have 50 fibers in parallel the muscle as a whole can generate 50 units of force. If we can get 100 fibers in parallel with a pennated arrangement of those fibers with an angle of pull that transmits 70\% of the force to the line of pull of the tendon then the muscle can generate 70 units of force. So, despite a slight loss of efficacy at the level of the muscle fiber (less tension transmits to the line of pull of the tendon), there is a gain in fidelity at the level of the muscle (generates more tension). Whereas muscles that are not pennated (fusiform or strap muscles) tend to have higher velocities of shortening during unresisted active tension and lower maximal forces during resisted active tension.

\paragraph{Overall Effect of Pennation}
The overall effect of pennation tends to be a decrease in velocity of shortening during unresisted active tension and an increase in maximal force during resisted active tension. Keep in mind that these are these are not variations that occur due to muscle adaptations to training or other modalities (pennation or not pennation). There is some evidence that angles of pennation can be modified slightly, and there is certainly evidence that all muscles can change their cross sectional area regardless of their pennation status. As you learn the anatomy of various muscles in your physical therapy education you should consider what that muscle tends to be used for and how whether that muscle is, and if so, how it is, pennated. It is also important to consider how muscles tend to work together in groups and how groups of muscles tend to complement one another in their synergistic roles.

\subsection{$\Delta$ Length During Active Tension}

Let’s consider $\Delta$ length a change in length of a muscle, \footnotemark{}\footnotetext{For now we will not consider whether there is a change in muscle fiber length or the dynamics that are occurring within the muscle fiber.} and that negative $\Delta$ length indicates a muscle shortens, positive $\Delta$ length indicates a muscle is lengthening and zero $\Delta$ length occurs when a muscle does not change length. We can now name various “contraction” types (as they are commonly known) based on the $\Delta$ length during active tension. 

\begin{table}[h!]
\centering
\begin{tabular}{||c c c ||} 
 \hline
 $\Delta$ Length & Muscle & Contraction Type \\ [0.5ex] 
 \hline\hline
 Negative & Shortens & Concentric \\
 Zero & Does not change & Isometric \\[1ex] 
 Positive & Lengthens & Eccentric \\[1ex] 
 \hline
\end{tabular}
\caption{Classification of $\Delta$ Length During Active Tension}
\label{table:contraction_types}
\end{table}


An important consideration to the concept known as contraction types is that anything other than a concentric contraction includes other counteracting forces. A muscle outside of its connected interaction space (not \textit{in situ}) will shorten (negative $\Delta$ length) with active tension.\footnotemark{} \footnotetext{Note that we are here assuming a full tetanic sequence of excitation contraction coupling as will be covered in upcoming chapters.} Therefore, the contingencies of 0 $\Delta$ length or positive $\Delta$ length are dependent on what is occurring in the system which includes all of the other interactions by other muscles and other external forces acting on the joint(s) that the muscle we are considering on it acting. And since the forces acting on joints create joint movement based on the torque being created, we can say that whether a muscle has a concentric, isometric or eccentric contraction depends on the balance between torque it is creating and the torque being created by all other forces acting at the joint.

\subsubsection{Example of how torque influences muscle contractions during active tension}

Figure \ref{fig:free_body_diagram} is a free body diagram (a simplified model) of the forces and related torque during the nordic hamstring exercise (NHE). During this exercise the subject is on their knees with their ankles supported so that the lower limb (below the knee) cannot move. The subject keeps their hips straight and slowly shifts their body mass forward. Once the body mass (black arrow) is anterior to the axis of rotation of the knee it creates a knee extension torque (black circular arrow). The subject allows themselves to not fall forward to quickly but usually reaches a point where they can no longer slow their descent and they fall to the floor. While they are attempting to slow their descent to the floor they are using their hamstring muscles which create a force through active tension (red arrow), which then creates a knee flexion torque (red circular arrow). The exercise can be done a variety of ways. In one scenario the subject can stop and hold themselves in one position. In another scenario, after holding themselves they can use the force of hamstring active tension to bring them upright again. 

\begin{figure}[!ht]
    \centering
    \includegraphics[width=1\linewidth]{./figure/free_body_diagram.png}
    \caption{Free Body Diagram of the Nordic Hamstring Exercise \footnotesize{(Created with Biorender.com}}
    \label{fig:free_body_diagram}
\end{figure}

There is a relationship between the force of the body mass ($F_{bm}$) and the extension torque created by the body mass ($T_{bm}$), just as there is a relationship between the force of the muscle ($F_{m}$) and the flexion torque created by the muscle ($T_{m}$). The force varies based on how much tension is created. The movement varies based on the relationship between the two opposing torques as seen in Table \ref{table:NHE}. It is important to note that the movement and the contraction types are dependent on the relationship of the torques, and the torques are related to both the force generated by the tension of the muscle, the body weight and their respective perpendicular distances.\footnotemark{}\footnotetext{This book will not spend much time at this level of analysis. However, if you are not comfortable with torque and vectors from physics and other courses you have had it is a worthwhile to refresh your memory now.}

\begin{table}[h!]
\centering
\begin{tabular}{||c c c c c||} 
 \hline
 Torque Relationship & $\Delta$ Length & Muscle & Contraction Type & Movement\\ [0.5ex] 
 \hline\hline
$T_m > T_{bm}$ & Negative & Shortens & Concentric & Knee Flexion \\
$T_m = T_{bm}$ & Zero & Does not change & Isometric & No Movement\\[1ex] 
$T_m < T_{bm}$ & Positive & Lengthens & Eccentric & Knee Extension \\[1ex] 
 \hline
\end{tabular}
\caption{$\Delta$ Length During Active Tension Based on Torque Relationships}
\label{table:NHE}
\end{table}

\section{Muscle Roles}

\begin{itemize}
\item Agonist: Muscle creating tension for the intention of the movement.
\item Antagonist: Muscle that creates the opposite movement of the agonist.
\item Synergists: Muscles that work together and combine their tension to create a movement. We can talk about synergist agonists and synergist antagonists, though the former is typically the intention when just referring to synergists.
\item Stabilizer: Muscles that provide tension to neutralize certain motions toward the purpose of a movement
\end{itemize}

To define a muscle role we must consider context. Context for muscle roles are typically movements. Movements include all the motions that are occurring, the directions of those motions and the intention of the mover doing the movement. For example, if the movement is the downward phase of a squat one of the motions is knee flexion. The intention of the mover is knee flexion (they intend to move downward). The knee extensor muscles are involved as agonists even though knee is flexing. Just because the motion is knee flexion, the agonists are not the knee flexors.



\section{Summary \& Next Step}

This chapter covered some of the fundamentals of muscle function such as fidelity and efficacy and how they related to attaining and sustaining tension. We discussed how the connective tissue scaffold allows muscle fiber tension to transmit to the muscle tendon and bony attachments for the purpose of whole muscle tension, including what it means to consider a muscle \textit{in situ}, and how that same scaffold provides an anchor for the neuromuscular supply and the overall shape of a muscle, including fascicle arrangement. Fascicle arrangement influences how the muscle \textit{in situ} force and length is related to muscle fiber force and $\Delta$ length. And how not only a particular muscle’s active tension, but other forces, have an integral influence the $\Delta$ length of a muscle during active tension and therefore contraction type. Finally we classified four muscle roles, agonist, antagonist, synergist and stabilizer. The next chapter covers the muscle fiber and considers its structure and the mechanisms involved in attaining and sustaining both active and passive tension.

\printbibliography[heading=subbibintoc]




\part{Muscle}

% !TEX root = ../notes_template.tex
\chapter{Muscle Tension}\label{chp:tension}

\minitoc

The goal of this chapter is to describe how muscle fibers generate both active and passive tension. Passive tension occurs when the muscle fibers resist lengthening and can also be referred to as stretch. Active tension is what we think of when we use the term muscle contraction. The muscle fibers create active tension which generates a force that shortens (or attempts to shorten) the fiber.  The fundamental difference between passive and active tension is whether the muscle is converting chemical energy from the form adenosine triphosphate (ATP) to mechanical energy at the crossbridge site between the proteins actin and myosin. During active tension the muscle is actively converting chemical energy to mechanical energy at this crossbridge site, whereas during passive tension the muscle is not actively converting chemical energy to mechanical energy at this crossbridge site, but is still resisting a unidirectional change in length. A unidirectional change in length we mean a muscle only generates passive tension in one direction, it resists lengthening. The muscle does not generate passive tension to resist shortening. The important point here is that the difference between active tension and passive tension is not whether the muscle is shortening or lengthening. The difference is also not whether there is positive or negative $\Delta$ length. The difference is whether the muscle fiber is converting chemical energy to mechanical energy. The term we’ll use for this is activated or activation, not contracted or contraction since the term contraction tends to imply a change in length. 

\vspace{5mm}

\textbf{Objectives include:}
\begin{enumerate}
    \item Describe the structures that generate tension.
    \item Explain the role of passive tension in muscle function.
    \item Describe the sliding filament theory of active tension.
    \item Explain the role of active tension in muscle function.
    \item Explain the underlying mechanisms and implications of the length-tension and force-velocity relationships.
   \item Apply the concept of muscle tension to the analysis of patient/client problems related to the generation of muscle force.
\end{enumerate}

\section{Anatomy of Tension}

The anatomy of tension focuses on the structures that are (mostly \footnotemark{}\footnotetext{First, the focus is on the muscle fiber not the muscle \textit{in situ}. But even if we take the muscle \textit{in situ} and consider the passive tension of the connective tissues (endomysium, perimysium, epimysium) and the tendon attachments, most of the tension from a muscle still emerges from the muscle fiber structures}) responsible for generating tension within the muscle fiber. Our focus on the muscle fiber is based on the clinical physiology focus. Cells are the basic unit of life and referring back to the Introduction, there are three basic principles of physiology on the topic of cells (cell theory, cell membrane, cell-to-cell communication). Even though in this chapter we identify parts of the muscle fiber that are required for tension (and hence are more basic than the muscle fiber for tension), in subsequent chapters it will become more clear how muscle fibers are really the basic “whole” unit for tension.

The parts of a muscle fiber that generate tension are located and organized into the sarcomere. It is important to understand the sarcomere within a hierarchical organization of the full muscle tension structure. Below we proceed through this hierarchical organization top down and then bottom up. The reader should be comfortable going in either direction and should pay particular attention to the middle position of muscle fibers, and how there are combinations of parallel and series combinations of structures. 

\paragraph{}
Hierarchical organization of tension structure, proceeding top down: 
\begin{itemize}
\item Muscles are built by fascicles. 
\item Fascicles are built by muscle fibers. 
\item Muscle fibers are built by myofibrils in parallel. 
\item Myofibrils are built by sarcomeres in series. 
\item Sarcomeres are built by myofilaments and structural proteins.
\end{itemize}

\paragraph{} 
Hierarchical organization of tension structure, proceeding bottom up: 
\begin{itemize}
\item Myofilaments and structural proteins build a sarcomere. 
\item Sarcomeres in series built a myofibril. 
\item Myofibrils in parallel build a muscle fiber. 
\item Muscle fibers in parallel build a fascicle. 
\item Fascicles in parallel and series build a muscle.
\end{itemize}

\begin{figure}[!ht]
    \centering
    \includegraphics[width=1\linewidth]{./figure/Myofibril_Structure.png}
    \caption{Myofibril Structure \footnotesize{(Created with Biorender.com)}}
    \label{fig:Myofibril_Structure}
\end{figure}

\paragraph{Muscle Fibers} As can be seen in the hierarchical organization above and in Figure \ref{fig:Myofibril_Structure} muscle fibers include a set of myofibrils arranged in parallel. The boundary of a muscle fiber includes the endomysium and sarcolemma (cell membrane of the muscle fiber). Within the muscle fiber the myofibrils are surrounded by cellular components with critical roles to play to the the tensioning act. Components such as extensions of the sarcolemma terminating at sacroplasmic reticula which are critical during the process of excitation and regulation (covered in Chapters 3 and 4); the mitochrondria which are critical for energetics (covered in Chapter 5); and nuclei and associated protein manufacturing components such as the Golgi apparatus and ribosomes which are critical for Muscle Integrity (covered in Part III). For now we are focused on the muscle fiber components that generate tension. For the process of generating tension, the myofibrils are the functional building block of muscle fibers, and sarcomeres are the functional building blocks of a myofibril.



\subsection{Anatomy of a Sarcomere}
Figure \ref{fig:Sarcomere_Structure} also provides a detailed look at the tensioning proteins of the sarcomere that will be the focus of the rest of this chapter. The sarcomere includes all of the proteins between two Z-discs. These proteins are all actors in the act of passive tension (resistance to positive $\Delta$ length) and active tension through the cross-bridge cycle (sliding filament model).

\begin{figure}[!ht]
    \centering
    \includegraphics[width=1\linewidth]{./figure/Sarcomere_Structure.png}
    \caption{Sarcomere Structure: Primary Protein "Actors" \footnotesize{(Created with Biorender.com)}}
    \label{fig:Sarcomere_Structure}
\end{figure}

\paragraph{Primary Protein "Actors" and their Primary and Secondary Roles}
\begin{itemize}
\item Actin - primary role in active tension; secondary role in passive tension
\item Myosin - primary role in active tension; secondary role in passive tension
\item Troponin - Tropomyosin - primary role in active tension; secondary role in passive tension
\item Titin - primary role in passive tension; secondary role in passive tension
\end{itemize}

Passive tension is created when the Z-discs are pulled apart from one another. Active tension is created in an attempt to pull the Z-discs closer together. For our purposes knowing the precise structure of each of these proteins is not necessary,\footnotemark{}\footnotetext{As long as you recognize that their structure is of great importance. As with all proteins the structure is the foundation of function. Proteins form molecular machines that have very precise functions based on their very precise structure. Anything that alters the structure of a protein alters its function. Variations in pH and temperature that exceed certain boundary conditions, for example, have the capability of changing protein structure.} our focus is on the general structure of the proteins that allow the protein to play its role in tensioning. The arrangement of these proteins with one another within the sarcomere is important. For our purposes in this book everything you need to know about the structure and arrangement is based on the roles they play in the upcoming sections.

\subsection{Structure and Arrangement of the Sarcomere During Changes in Length}

The primary structural change to the sarcomere at different lengths is the distance between the Z-discs (Figure \ref{fig:Sarcomere_Lengths}). The majority of length changes in a muscle propagate down to, and emerge up from, changes in the distance between Z-discs of the sarcomeres.\footnotemark{}\footnotetext{The stress - strain relationships of tendons indicate that a tendon can experience about a 3\% change in length prior to damaging the tendon. This amounts to up to a 6\% change in muscle length when accounting for tendon at each of two attachments. Whether this contributes to a muscle change in length is very contextual. During lengthening it would depend on how much tension is being produced (stress on the tendon), and during shortening it would depend on how much lengthening occurred in the tendon prior to shortening since the tendons do not produce active tension.} Increasing the distance between Z-discs reduces the interaction space between the actin and myosin and, once at a certain length, starts to pull on the elastic structure of titin (unwinding of titin). Decreasing the distance between Z-discs will result in increasing the interaction space between actin and myosin, and recoil of titin to a point; followed by crowding of potential actin myosin crossbridges (decreasing interaction space), and compression of titin. The implications of these structural changes are discussed in the sections on Active and Passive Tension.

\begin{figure}[!ht]
    \centering
    \includegraphics[width=1\linewidth]{./figure/Sarcomere_Lengths.png}
    \caption{Sarcomere Lengths. A. Shortened; B. Relaxed / resting; C. Lengthened \footnotesize{(Created with Biorender.com)}}
    \label{fig:Sarcomere_Lengths}
\end{figure}

\section{Passive Tension}
Passive tension is created when the Z-discs are pulled apart from one another. During positive $\Delta$ length the muscle fibers increase in length and the Z-discs of sarcomeres are pulled apart. The amount of passive tension developed (measured as force that resists lengthening) is a function of the relative length of the muscle this is primarily due to the relative length of titin. This relationship is the passive component of what is called the length - tension relationship (Figure \ref{fig:passive_lt}) (Figure \ref{fig:passive_lt}).

\begin{figure}[!ht]
    \centering
    \includegraphics[width=1\linewidth]{./figure/passive_lt.png}
    \caption{Passive Component of the Length Tension Curve \footnotesize{(Created with Biorender.com)}}
    \label{fig:passive_lt}
\end{figure}

\paragraph{Muscle Tone}
At any given moment in the “inactivated” muscle there are some attachments between actin and myosin. The number of actin and myosin attachments influences the passive tension developed at various lengths. This is referred to as muscle tone. Muscle tone is the amount of tension in a muscle when it is not activated, that is when it is not creating active tension (and hence is essentially passive tension). Muscle tone is also dependent on the length of the sarcomere since that length influences the number of possible actin myosin attachments. While related to changes in length, the presence of these actin myosin attachments in an inactivated muscle is primarily related to the underlying reasons for the attachments and will be covered in later chapters. For now simply keep in mind that muscle tone is directly proportional to passive tension.

\subsection{Passive Tension Role In Movement}
Passive tension can have a fundamental role in movement, and alterations in passive tension can therefore have a fundamental role in altered movement (kinesiopathology). When passive tension is developed there is a force developed by the muscle undergoing passive tension that creates a torque across the joint formed by the bones the muscle attaches. For example, when the hamstrings are lengthened and generate passive tension, that passive tension starts to create a flexion torque at the knee. If the hamstrings are then activated (start to generate active tension) then the total tension created is the sum of the active and the passive tension, which together create the force that creates the flexion torque at the knee. Muscles involved in gait (walking) can cycle through periods of passive tension and then use that passive tension to supplement active tension. For example, during running the calf muscles are lengthened while the foot is on the ground which increases passive tension. When it is time for that foot to propel the body the torque produced by the calf muscles is a combination of passive and active tension. In this situation, additional passive tension reduces the need for active tension and thus increases muscle efficacy (less energy is required, and therefore the muscle can sustain tensioning for a longer period of time).

\section{Active Tension}

Active tension is created when the muscle fiber transforms the chemical energy stored in adenosine tri-phosphate (ATP) into mechanical energy that allows myosin to actively slides actin from one actin-myosin crossbridge to another. This process is activated at the level of the muscle fiber, but all of the events involved directly in this energy transformation occur in the sarcomere. The result is a tension that pulls (or attempts to pull) the Z-discs of the sarcomere closer together.

\subsection{Sliding Filament Model}
The sliding filament model of active tension (also referred to as the sliding filament theory of muscle contraction) is well understood at the level we need to understand it.\footnotemark{}\footnotetext{There remains active investigation into the sliding filament process and the details of crossbridge kinetics. All models are wrong, some models are useful. The intricacy of events in reality is likely more complicated than represented by this model. The intricacy, combined with how quickly they occur, makes a complete understanding challenging. But what we know about it is useful for our purposes. We use this model to explain many phenomenon throughout the book.} 
\paragraph{}
Figure \ref{fig:Sliding_Filament} depicts the steps in the process of creating active tension. The steps are also enumerated below.

\begin{figure}[!ht]
    \centering
    \includegraphics[width=1\linewidth]{./figure/Sliding_Filament.png}
    \caption{Sliding Filament Model of Active Tension \footnotesize{(Created with Biorender.com)}}
    \label{fig:Sliding_Filament}
\end{figure}

\begin{enumerate}
\item ADP-bound myosin head is energized with potential energy (from the release of a phosphate from ATP) and is ready to bind to actin. While the myosin head is energized, it is inactivated.
\item In the presence of calcium, calcium binds to troponin, moving tropomyosin and exposing actin binding sites. In crossbridge kinetics terminology this is the change in state of the crossbridge from inactivated to activated.
\item The bound myosin rotates its head (releasing potential mechanical energy), producing a 'power stroke'. The crossbridge remains activated.
\item ATP molecule binds to the myosin head and changes its shape allowing it to detach from actin (state change from activated to inactivated).
\item Actin and myosin are detached and not energized (inactivated)
\end{enumerate}

\subsubsection{Crossbridge Kinetics}
The current understanding of the molecular basis for muscle contraction comes from Huxley’s (1957) kinetic model of the cyclic interaction between actin myosin crossbridges \cite{huxley_muscle_1957}. Active tension behavior of a sarcomere (such as the force - velocity relationship described below) does not originate from the behavior of individual crossbridges, but from the collective action of all activated crossbridges in the sarcomere as they asynchronously go through the energy (ATP) dependent cycle described in the sliding filament model. In the crossbridge kinetics model the sarcomere has, at any given length, a number of possible crossbridges.  The number of possible crossbridges is the crossbridge interaction space. Each possible crossbridge has two states, activated and inactivated. The active tension of a sarcomere is directly proportional to the number of crossbridges in the activated state which is influenced by both the concentration of calcium in the sarcomere and the length of the sarcomere. The sliding filament model describes one cycle of crossbridge kinetics (one cycle of inactivation to activation to inactivation).

\paragraph{Introduction to Twitch}
A twitch is the fundamental unit of active tension. It is a spike in active tension generated when crossbridges are activated from one excitation of a muscle fiber. A twitch can be considered at the level of the sarcomere, myofibril, fiber and muscle. A twitch \textit{in situ} is perceptible but not functional. If many muscle fibers twitch at the same time it may be experienced as a spasm, though not long lasting. Sustained spasms (myotonia) involve more than a twitch. A twitch can be thought to equate to one cycle of crossbridge kinetics. The force of a twitch is directly proportional to the number of crossbridges in the activated state, meaning if there are more crossbridges cycling into the activated state there is more tension. Experimentally, all sarcomere twitches will tend to produce the same tension when performed at the same length and when separated by enough time. A myofibril twitches can have small variations in tension depending on how many of its sarcomeres have crossbridge cycling from the stimulus provided. Muscle fiber twitches can vary greatly in the amount of tension during a twitch because there is greater potential for variation in the number of sarcomeres involved in crossbridge cycling. Experimentally the number of sarcomeres involved in cross bridge cycling will depend on the amount of excitation stimulus provided. 

\paragraph{Introduction to Tetany}
When a sarcomere is repeatedly excited the resultant twitches (spikes in tension) from each excitation start to fuse with each other and the spikes in tension from each twitch get smooth while the overall tension increases. This transition to a smooth rise and even maintenance of tension from a sarcomere is tetany. The greater the number of excitations the greater the tension developed in tetany because the result is a greater number of crossbridges in the activated state at any given moment across time. The amount of tension developed reflects the ability to attain tension, and the period of time tetany can last during repeated excitations is the ability to sustain tension.
In Chapter \ref{chp:excitation} the concepts of twitch and tetany are the bridge between muscle excitation and crossbridge kinetics. In Chapter \ref{chp:regulation}they are used to connect muscle regulation to excitation. 

\subsection{Active Component of the Length Tension Relationship}
Since the sarcomere crossbridge interaction space varies with its length \footnotemark{}\footnotetext{Distance between the Z-discs} the number of activated crossbridges also varies with its length. Therefore, under conditions of maximal activation (high calcium concentrations), the active tension varies with sarcomere length.  The crossbridge interaction space decreases as the distance between the Z-discs increases or decreases from an ideal distance. In a circular we then define the ideal distance between the Z-discs (sarcomere length) as the length that maximizes the crossbridge interaction space. 

The active component of the length tension relationship was originally measured using isolated muscle fibers under isometric conditions at different lengths. However, we do tend to consider the impact of the underlying phenomenon during activation that includes dynamic changes in length. As the distance between the Z-discs decreases, such as during negative $\Delta$ length (concentric) or when muscle activation begins while in a shortened length, the actin-myosin interaction space is decreased because of the actin binding sites increase in concentration with a limited number of myosin heads (near the M-line) which reduces the number of potential crossbridge sites. This reduces the active tension. As the distance between the Z-discs increases, such as during positive $\Delta$ length (eccentric) or when muscle activation begins while in a lengthened length, the actin-myosin interaction space is decreased because parts of actin are not in proximity with myosin which reduces the number of crossbridge sites. This reduces the active tension.

The active component of the length tension relationship is depicted in Figure \ref{fig:active_lt}.\footnotemark{}\footnotetext{The depiction of the actin myosin crossbridges in this particular rendering includes overlap of the actin during the shortened range. Whether there is such overlap vs. crowding of sites near the M-line is still an open discussion, or at least as far as the author is aware. It's another example that all models are wrong and some models are useful.}

\begin{figure}[!ht]
    \centering
    \includegraphics[width=1\linewidth]{./figure/active_lt.jpg}
    \caption{Active Component of the Length Tension Curve \footnotesize{(OpenStax, CC BY 4.0, \href{https://creativecommons.org/licenses/by/4.0}{via Wikimedia Commons})}}
    \label{fig:active_lt}
\end{figure}


\subsection{Length-Tension Relationship (Active \& Passive Components)}

The full length tension relationship of a sarcomere, and a myofibril, is depicted in Figure \ref{fig:lt}. It is important to keep in mind that this relationship is based on experiments of myofibrils removed from their anatomical locations and maximally activated under isometric (zero $\Delta$ length) conditions. The length tension relationship of any particular muscle \textit{in situ} may vary from this general relationship based on a number of factors such as the normal resting position of the muscle with anatomical attachments, the pennation arrangement, whether the conditions of maximal activation are met (rarely), whether the conditions of zero $\Delta$ velocity is met (rarely during functional movement), or whether the muscle crosses more than one joint. Two of these deserve special mention. 

\begin{figure}[!ht]
    \centering
    \includegraphics[width=1\linewidth]{./figure/lt.jpg}
    \caption{Full Length Tension Curve with Total Tension \footnotesize{(Public Domain figure from \href{https://commons.wikimedia.org/wiki/File:Lengthtension.jpg}{Wikimedia Commons})}}
    \label{fig:lt}
\end{figure}


\paragraph{Manual Muscle Testing} During manual muscle testing (MMT) physical therapists are taught about the ideal joint range of motion (ROM) for testing. These ROMs are typically positions that put the muscle at a length intended to maximize tension based on the active component of the length tension relationship. 

\paragraph{Active \& Passive Insufficiency & Sufficiency}
A special situation that arises in muscles that have attachments crossing more than one joint is that the ROM of both joints influences the muscle length. Since muscle length influences both active and passive tension the position or movement of both joints influences active and passive tension.
Active insufficiency occurs when the active tension a muscle creates is impacted by the position or movement of the joints (at least 2) that is crosses. For example, if you attempt to extend the hip and flex the knee at the same time using active tension your ability to generate active tension will decline quickly due to the fact that the hamstring sarcomere Z-discs are getting closer and actin-myosin interaction space is decreasing.
Passive insufficiency occurs when the passive tension a muscle creates is impacted by the position or movement of the joints (at least 2) that it crosses. For example, if you attempt to flex the hip and extend the knee at the time time, passive tension develops due to the hamstring sarcomere Z-discs getting further apart and titin is resisting further changes to its length.\footnotemark{}\footnotetext{Please note that the focus here on titin is for simplicity. Other connective tissues within the muscle contribute to this passive tension. It is also possible that extra-muscular connective tissue such as fascia can contribute to the passive tension developed during passive insufficiency since these movements tend to not serve critical movement functions. They tend to be movements contrary to what would be considered critical movement functions.}
Active and passive sufficiency is the principle (possibly new to this book) that most functional movements tend to avoid active and passive insufficiency. Most functional movements tend to use multi-joint muscles in a way that optimize their tension generating capabilities based on the length tension relationship. For example, during squatting and climbing, the hamstring sarcomere length does not change much since these movements couple hip flexion with knee flexion (one direction of movement), and hip extension with knee extension (the other direction of movement).

\subsection{Necessary Molecules}
Most of the molecules involved in the sliding filament model are protein actors in the sarcomere. However, there are two molecules that make important appearances and deserve special mention at this point as a preface to where we are heading as we progress toward connecting muscle physiology to clinical physiology.

\paragraph{} 
Calcium($Ca^{2+}$) enters the scene to get things started. It triggers the events that lead to the crossbridge state change from inactivated to activated. It is the signaling molecule utilized to activate the process of the sliding filaments for active tension. It is stored in an organelle in close proximity to the sarcolemma called the sarcoplasmic reticulum. $Ca^{2+}$ enters the cell once the sarcolemma and sarcoplasmic reticulum is excited and is an important part of Chapter 3 on Excitation. Calcium is the connecting molecule in the process called “excitation - contraction coupling” which we call “excitation - activation coupling”. As you can imagine, having sufficient $Ca^{2+}$ available for the muscle fiber is essential. The processes supporting sufficient $Ca^{2+}$ concentrations is covered in Part II: Muscle Support and involves the neuroendocrine and gastrointestinal systems. 

\paragraph{}
ATP provides the chemical energy that is transformed into mechanical energy by the molecular motors within the sarcomere. ATP is kept in limited quantities inside the muscle cell. It is used to energize cellular processes that require energy (do not confuse cellular processes with chemical reactions). Examples of cellular process include transcription and translation or a pump that moves ions across a membrane to form a gradient. Chemical reactions are transformations that proceed bidirectionally with one direction usually favored by mass balance and enzyme catalysts. Since ATP is stored in limited quantities (enough for seconds of myosin energizing), the muscle fiber must be able to convert other forms of energy (usually chemical energy) into ATP. That is the primary topic of Chapter 5 on Muscle Energetics.


\section{Force \& Velocity}

Throughout the next few chapters we develop an understanding of force (as the result of tension) and velocity. Our understanding of the relationship between them will develop from two perspectives: the Force - Velocity (FV) and the Velocity - Force (VF) Relationships. In this chapter we introduce the concept and these two perspectives (FV \& VF), and consider what our current understanding about sarcomeres contributes to that understanding.

By force we are referring to force being measured that is the result of muscle tension (total tension). By velocity we are referring to changes in length over time. $\Delta$ length and velocity are related: 

\begin{equation} \label{eq_velocity}
velocity \ = \frac{\Delta \ length}{time}
\end{equation}

 Once we consider velocity we are considering changes in length, which makes the force velocity relationships different from length-tension relationships. Length tension relationships are experimentally observed under conditions of isometric (zero $\Delta$ length) conditions and is theoretically extended to dynamic (non zero $\Delta$ length) conditions. Force velocity relationships must be experimentally observed under dynamic (non zero $\Delta$ length) conditions. Since velocity involves changes in length, the force velocity relationships must include considerations related to the implications of the length tension relationship. Even though the length tension relationship is observed under conditions of isometric activation, what the length tension relationship represents (changes in sarcomere structure that result in changes in passive elastic and active tension) is occurring during a change in sarcomere length.

\paragraph{Movement Energetic Perspective}
Before proceeding with the FV and VF relationships it is important to point out that these relationships can be differentiated from a third perspective - the movement energetic perspective. Keeping in mind that movements are the sum of torques acting across joints with a larger imbalance between torques produce a “faster” movement, and keeping in mind that torques are generated by force which is generated by active muscle tension, and that active muscle tension is proportional to the number of activated crossbridges, and that the number of activated crossbridges is proportional to the amount of ATP being utilized, we can conclude that “faster” movements utilize more ATP (that is, they are more energetic). The movement energetic perspective is discussed in Chapter 5 on Muscle Energetics. In summary, it is why running at a faster pace makes you breath heavier than a slower pace; and why certain running paces (sprinting) are not possible for very long. \footnotemark{}\footnotetext{We are keeping the movement energetic perspective separate from the FV and VF relationships because it is not traditionally considered when discussing these relationships.} 

\subsection{Sarcomere Arrangement}

The FV and VF relationships can be considered at the level of the sarcomere, myofibril, muscle fiber or whole muscle. It is useful to review the arrangement of the sarcomeres across this hierarchy since the force generated and the velocity of shortening are both related to whether sarcomeres are arranged serially or in parallel. Tension within a sarcomere is transmitted to the myofibril, and sarcomeres in a myofibril are arranged serially.  Myofibrils are arranged in parallel (some may be serial)in a muscle fiber. And muscle fibers are arranged in parallel (some may be serial) within a muscle fascicle. And muscle fascicles are arranged in parallel within the muscle.

\paragraph{Sarcomeres in series - higher maximal velocity} 

A myofibril consists of sarcomeres arranged in series. As explained in Chapter 1 on Fundamentals, this means the change in length of a myofibril is the sum of  the change in length of the sarcomeres. Increasing the number of sarcomeres in a myofibril does not change the amount of tension that can be created since they are in series and this does not change the cross sectional area. Therefore, a myobril with more sarcomeres serially connected (longer) can achieve higher maximal velocity of shortening. If a sarcomere can shorten $1 \mu m$ in a millisecond, then a myofibril of 1000 sarcomeres shortens 1 mm and a myofibril of 500 sarcomeres shortens at 0.5 mm in 1 microsecond.


\paragraph{Sarcomeres in parallel - higher maximal force}
A muscle fiber consists of myofibrils arranged in series and in parallel. This means that the change of length of the muscle fiber is proportional to the change of length of myofibrils in series (which is proportional to the number of sarcomeres organized in series). Myofibrils are also arranged in parallel. The number of myofibrils in parallel influence the cross sectional area of the muscle. The tension generated increases proportionally with the cross sectional area of the muscle which can be achieved by increasing the number of myofibrils, and thus sarcomeres, in parallel. If a sarcomere generates 1 unit of tension then 10 sarcomeres serially connected also create one unit of tension (See Figure \ref{fig:Sacromeres_series_parrallel}. However, if those 10 sarcomeres are connected in parallel then they will together create 10 units of tension.

\subsection{Force - Velocity \& Velocity - Force Relationships}

\paragraph{Velocity - Force Relationship}
We consider the VF Relationship under the condition of no resistance. In the VF relationship force is a function of velocity. The muscle is not resisted \footnotemark\footnotetext{By anything of any significance beyond the resistance of joint friction, limb mass and air resistance. The point is that, in this situation, velocity is not impacted by the necessity for the muscle to generate enough active tension to overcome the force of the resistance.} and we consider the force that can be achieved with varying levels of velocity. We plot the independent (manipulated) variable on the X axis, and as the dependent (observed) variable we plot force on the Y axis. We consider the VF relationship in Chapter 4 on Muscle Regulation since the phenomenon involves variations in muscle fiber types (heterogeneity between muscle fibers).

\paragraph{Force - Velocity Relationship}
We consider the FV Relationship under the condition of resistance. In the FV relationship velocity is a function of force. The relationship between force and shortening velocity is visualized by plotting the velocity of a shortening muscle as a function of the load (or force) pulling on the muscle \cite{seow_molecular_2022}. The muscle is resisted and we consider the velocity that can be achieved with varying levels of resisted muscle force. We plot the independent (manipulated) variable on the X axis, and the dependent (observed) variable velocity on the Y axis. FV has an inversely proportional hyperbolic relationship and was first described by Hill \cite{seow_hills_2013}. Because power is the product of force and velocity it has a parabolic relationship which achieves its peak at neither the extreme of velocity or force \cite{seow_hills_2013}. 

\subsubsection{Force Velocity Relationship during Shortening (Concentric)}
Figure \ref{fig:fv_shortening_1} shows the FV relationship during concentric (shortening) activation with power. It is a consistent finding that peak power occurs between 20 and 30\% of peak force.

\begin{figure}[!ht]
    \centering
    \includegraphics[width=1\linewidth]{./figure/fv_shortening_1.jpg}
    \caption{Force Velocity Curve and Power Relationship during Shortening \footnotesize{Creative Commons Attribution License (CC BY) Figure from \cite{seow_hills_2013}}}
    \label{fig:fv_shortening_1}
\end{figure}

Figure \ref{fig:fv_shortening_2} includes five FV and power curves to demonstrate the impact of sarcomere arrangement. Readers should familiarize themselves with the various curves and make sure they understand how they relate to sarcomere arrangement.

\begin{figure}[!ht]
    \centering
    \includegraphics[width=1\linewidth]{./figure/fv_shortening_2.jpg}
    \caption{Impact of Sarcomere Arrangement on the FV Relationship during Shortening. \footnotesize{Creative Commons Attribution License (CC BY) Figure from \cite{seow_molecular_2022}}}
    \label{fig:fv_shortening_2}
\end{figure}

\paragraph{}
There are two factors contributing to the FV relationship during concentric activation. First, is the fact that a concentric activation requires the muscle generate more torque than the resistance (load). When someone generates just enough torque, they generate a minimal (non zero) velocity. As load is reduced, the maximal active tension of the muscle creates an abundance of torque and this results in greater velocity. 

Second, under the condition of attempting to move as fast as possible, as the load on and the force generated by a muscle increases, there is an increase in the velocity. The drop in force is due to the drop in load, which then allows for a higher velocity. However, it is also true that at the higher velocity there is a lower capability for generating force (the VF relationship). This is due to two factors that we can trace back to crossbridge kinetics. A, with increased velocity of shortening the crossbridge interaction space decreases rapidly which brings the sarcomeres into a shorter length and reduces the tension they can create (i.e. length tension relationship). B, with increased velocity of changes in length there is less time for crossbridges to be in their activated state, which means at any moment there is a lower number of possible crossbridges in the activated state. 

\paragraph{Experience of the FV relationship during concentric activation}
The FV relationship during concentric activation is something people can easily relate to and have experienced. There is a minimal velocity associated with any activity people perform with maximal resistance (force). Note that minimal velocity varies between activities.  For example, the minimal velocity of a dead lift is lower than the minimal velocity of a power clean. 


\subsubsection{Force Velocity Relationship during Lengthening (Eccentric)}

The FV relationship for eccentric (lengthening) activation is depicted by plotting velocity on the X-axis and force on the Y-axis since it requires a change in the direction (sign) of velocity that occurs as you go from concentric to eccentric activation. We will see this same axis flip when we consider the VF relationship in Chapter 4.

Figure \ref{fig:force_velocity_lengthening} shows the relationship between force and velocity during both shortening (right side of Y axis, positive velocity) and lengthening (left side of the Y axis, negative velocity). There is a rapid increase in force at the initial transition point. The force (load, resistance) is now overwhelming, that is exceeding, the capacity of the sarcomere to generate force. But it is just exceeding which engages the passive elements of the sarcomere (titin) to contribute to force and has not yet resulted in rapid loss of the crossbridge interaction space or time for crossbridge activation. The force that can be generated by the muscle (in opposition to the external force (load, resistance)) quickly plateaus once titin has been fully engaged and the increased velocity diminishes time for crossbridge activation and the lengthening of the muscle results in less crossbridge interaction space (again, due to the length tension relationship). 


\begin{figure}[!ht]
    \centering
    \includegraphics[width=1\linewidth]{./figure/force_velocity_lengthening.png}
    \caption{Force Velocity Curve during Shortening \footnotesize{(Created with Biorender.com}}
    \label{fig:force_velocity_lengthening}
\end{figure}


\paragraph{Experience of the FV relationship during eccentric activation}

The FV relationship during eccentric activation is something people can easily relate to and have experienced. Very simply, as the load someone attempts to hold  (but cannot actually hold) increases, the velocity of movement (muscles get longer) also increases.

\section{Summary \& Next Step}

In this chapter we have covered the micro anatomy of muscle fibers at sufficient depth to understand the mechanisms behind several concepts relevant for physical therapy practice such as the length tension and force velocity relationships. The sarcomere is the functional unit of the muscle cell due to its repeated appearance in the muscle.  You should now understand the parts and capabilities of muscle fibers at the level of the sarcomere and how these can generate active tension by converting chemical energy into mechanical energy, or passive tension by resisting change to length. Active tension involves a sequence of events repeated over and over. The sequence begins with excitation of the muscle fiber. Therefore, our next step is to consider excitation.

\printbibliography[heading=subbibintoc]

% !TEX root = ../notes_template.tex
\chapter{Muscle Excitation}\label{chp:excitation}
Updated on \today

\minitoc

This chapter covers muscle excitation, the molecular mechanisms that allow excitation, and related clinical physiology connections. 

Muscle fiber excitation starts with an excited motor axon ($\alpha$-motor neuron) and ends with the binding of calcium to troponin. Binding calcium to troponin allows a crossbridge to go from an inactivated to an activated state. The final step in muscle excitation is the coupling between excitation and activation (Excitation - Activation Coupling). 

The underlying molecular mechanisms of the muscle excitation pathway are fundamental to many areas of clinical physiology. The mechanisms include the characteristics and actions of excitable membranes. Excitable membrane characteristics include ion channels, pumps and receptors. Actions include using transport to establish a resting membrane potential and the ability to have an action potential (excitation). 

Understanding excitable membranes creates Clinical Physiology Connections opportunities in three areas: 1. system wide regulatory function of the neuroendocrine system; 2. function of sensory receptors; and 3. pharmacodynamics.

\vspace{5mm}

\textbf{Objectives include:}
\begin{enumerate}
    \item Explain the events of muscle excitation and excitation - activation coupling.
    \item Relate the events of excitation - activation coupling to the creation of active tension.
\item Explain the components and capabilities of an excitable membrane.
    \item Explain the events that result in sarcolemma depolarization.
   \item Explain the events that result in neuromuscular endplate depolarization.
   \item Explain the possible result of hypothetical ("what if") situations on the excitation and activation of a muscle fiber (for example, What if calcium was depleted and less was being released from the SR?; What if ACh was not broken down quickly at the NMJ?).
    \item Explain the function and role of the neuroendocrine system for homeostasis.
\item Explain the basic function of sensory receptors.
    \item Explain pharmacodynamics based on actions occurring at end organ receptors
   \item Demonstrate the ability to apply basic physiology concepts such as the cell membrane, mass balance, and flow gradients to the analysis of patient/client problems related to the generation of muscle excitation, the effectiveness of the neuroendocrine system and pharmaceuticals.
\end{enumerate}

\section{Muscle Fiber Excitation} % Draft of this section as of June 3, 2022 - keep as is for now
\paragraph{Excitation}
Muscle fiber excitation is the signal that initiates crossbridge activation and results in a twitch. Excitation is a state of the membrane. Based on the concept of excitation a muscle fiber exists in one of two states, resting or excitation. Excitation occurs when a membrane depolarizes and the process of depolarizing is called an action potential (AP). The resting state of the membrane is polarized and is the resting membrane potential (RMP). Cellular membranes that can undergo excitation are excitable membranes, and include nerve and muscle cell membranes. 

\paragraph{Excitation as Signal Transduction}
Excitation is a mechanism for signal transduction (sending and receiving signals). One example is when membrane excitation results in cellular actions that influence cellular processes, such as exocytosis at the terminal axon at the neuromuscular junction.\footnotemark\footnotetext{Exocytosis is a form of active transport and bulk transport in which a cell transports molecules (e.g., neurotransmitters and proteins) out of the cell). From \url{https://en.wikipedia.org/wiki/Ligand}}. A second example is when a membrane is excited through binding of a ligand\footnotemark\footnotetext{A ligand is any molecule or atom which binds reversibly to a protein. A ligand can be an individual atom or ion. It can also be a larger and more complex molecule made from many atoms. A ligand can be natural, as an organic or inorganic molecule. A ligand can also be made synthetically, in the laboratory. From \url{https://biologydictionary.net/ligand/}} to a receptor on its surface which can then excite the surface, direct another signal into the cell (or both). 

\subsection{Muscle Fiber Excitation Overview}
\paragraph{}

Skeletal muscle fiber excitation starts with excitation of an $\alpha$-motor neuron (Step 1 in Figure \ref{fig:excitation_overview}) which crosses the neuromuscular junction (NMJ) and excites the motor end plate (Step 2 in Figure \ref{fig:excitation_overview}). Excitation of the motor end plate causes excitation of the sarcolemma, which causes excitation of the transverse tubule system (T-tubules). Excitation of T-tubules cause excitation of the sarcoplasmic reticulum (SR), which causes the release of calcium into the muscle fiber and the activation of available crossbridges (Step 3 in Figure \ref{fig:excitation_overview}). Excitation of the SR and release of $Ca^{2+}$ is referred to as Excitation-Activation Coupling (or Excitation-Contraction Coupling). A single excitation results in a twitch (the fundamental unit of active tension).

\begin{figure}[!ht]
    \centering
    \includegraphics[width=1\linewidth]{./figure/excitation_overview.png}
    \caption{Overview of excitation from the $\alpha$ motor neuron to crossbridge activation \footnotesize{Created with BioRender.com}}
    \label{fig:excitation_overview}
\end{figure}

\subsection{Step 1 - The $\alpha$-Motor Neuron}
Excitation of an $\alpha$-motor neuron is regulated by the central nervous system and occurs at the membrane of its dendrites in the spinal cord. A wave of excitation travels along the axon to its terminal branches at neuromuscular junctions (NMJs) by sequential excitation facilitated by voltage gated ion channels. Each muscle fiber has one NMJ and receives one axon terminal. However, each $\alpha$-motor neuron has a variable number of terminal branches (Chapter \ref{chp:regulation}).  Excitation travels from the spinal cord to the NMJ by means of saltatory conduction which allows excitation to take big steps as it travels along the axon membrane. Saltatory conduction increases the nerve conduction velocity and is made possible by the myelin sheathes (Figure \ref{fig:Motoneuron}). The membrane is exposed along the axon at nodes of Ranvier. Excitation travels by jumping from one node of Ranvier to the next.\footnotemark\footnotetext{The implications of the myelin sheath on nerve conduction velocity is detailed in Chapter \ref{chp:regulation} as an important consideration for muscle tension regulation.} 

\begin{figure}[!ht]
    \centering
    \includegraphics[width=1\linewidth]{./figure/Motoneuron.png}
    \caption{$\alpha$-motor neuron terminating at several muscle fiber motor end plates \footnotesize{Created with BioRender.com}}
    \label{fig:Motoneuron}
\end{figure}

\subsection{Step 2 - Neuromuscular Junction - Motor End Plate - Sarcolemma Excitation}

Membrane excitation at the axon terminal releases the neurotransmitter acetylcholine (ACh) due to the action of voltage gated $Ca^{2+}$ channels signaling exocytosis of vesicles filled with ACh. ACh crosses the NMJ synapse (Figure \ref{fig:NMJ} and excites the motor end plate by binding to receptors and opening ligand gated $Na^+$ ion channels. For tight regulation of muscle activation at the NMJ, ACh is quickly broken down by cholinesterase (an enzyme synthesized by the muscle fiber that hydrolyzes ACh) and free choline molecules are transported into the axon terminal (endocytosis) to be resynthesized. However, when high concentrations of ACh accumulate at the NMJ (due to a high frequency of axon excitations) there are two processes that can result in more rapid excitation of the motor end plate.\footnotemark\footnotetext{"More rapid" means generating the motor end plate excitation faster to allow for the higher frequency of excitation, not that the motor end plate ends up having a higher frequency of excitation than the axon. Such decoupling of $\alpha$-motor axon excitation and motor end plate excitation would create additional challenges to motor control.} Temporal summation of receptors refers to repeated stimulation of one receptor by ACh; and spatial summation of receptors refers to stimulation of a larger number of motor end plate receptors. Temporal summation is limited by the amount of ACh, which is influenced by the release of ACh from the axon terminal and the breakdown of ACh in the NMJ. Spatial summation is also limited by the amount of ACh, but it is also limited by the number of receptors on the motor end plate, and is therefore adaptable over time through cellular regulatory processes (making more receptors (up regulating), or reducing the number of receptors (down regulating).

% axon terminal voltage gated Ca+ channels are inhibited by high magnesium ($Mg^{2+}$) concentrations. - not sure why I had that random fact in the middle of the paragraph - but I'm not ready to get rid of it yet because if I can expand on it and make come clinical connection it could be worthwhile.

\begin{figure}[!ht]
    \centering
    \includegraphics[width=1\linewidth]{./figure/NMJ.png}
    \caption{NMJ Excitation of the Motor End Plate \footnotesize{Created with BioRender.com}}
    \label{fig:NMJ}
\end{figure}

Excitation of the motor end plate spreads to the sarcolemma due to the opening of voltage gated ion channels on the sarcolemma. Sarcolemma excitation creates a wave similar to that in the axon (but without myelin) that spreads due to opening of voltage gated ion channels. The sarcolemma surrounds the fiber and penetrates the fiber and circles myofibrils through the intricate T-tubule system. T-tubules that encircle myofibrils form a triad with a SR on each side (See Figure \ref{fig:T-tubule}).

% Import the T-Tubule - SR triad image - CC from wikipedia

\begin{figure}[!ht]
    \centering
    \includegraphics[width=1\linewidth]{./figure/T-tubule.jpg}
    \caption{T-Tubule Encircling a Myofibril \& Forming a Triad with Sarcoplasmic Reticulum  \footnotesize{(OpenStax, CC BY 4.0, \href{https://creativecommons.org/licenses/by/4.0}{via Wikimedia Commons})}}
    \label{fig:T-tubule}
\end{figure}

\subsection{Step 3 - Excitation - Activation Coupling (EAC)}
Excitation of the T-tubules excites the sarcoplasmic reticulum (SR). There are unique junctions between T tubules and the SR mediated by $Ca^{2+}$ channels. One acts as a voltage sensor in the T tubules, and a second acts as $Ca^{2+}$ release channels in the SR.\footnotemark{}\footnotetext{The position of these two calcium channels differ between skeletal and cardiac muscle, which correlates with the functional differences in the control of contraction between these two types of muscle.} The functional implication of this unique T-tubule - SR junction is that excitation of the T-tubule quickly results in a combined excitation of the SR along with the release of $Ca^{2+}$ into the sarcoplasm\footnotemark\footnotetext{Sarcoplasm is the muscle fiber equivalent of the cytoplasm, the intracellular region that is in the cell but outside of any organelles (such as the SR, nucleus, mitochondria).} in close proximity of troponin binding sites for crossbridge activation (Figure \ref{fig:SR}).

\begin{figure}[!ht]
    \centering
    \includegraphics[width=1\linewidth]{./figure/SR.png}
    \caption{Sarcoplasmic Reticulum as the site of Excitation-Activation Coupling \footnotesize{Created with BioRender.com}}
    \label{fig:SR}
\end{figure}

\paragraph{Excitation \& EAC Latency}

Figure \ref{fig:eac-latency} represents each major phase of excitation and EAC from the $\alpha$-motor neuron to development of a twitch. The latency period time scale is in milliseconds (ms, 1/1000 of a second). The latency period of each excitation and EAC forms an upper boundary on the number of possible twitches per second. However, this boundary is determined by the latency of the excitation portion of Figure \ref{fig:eac-latency} which amounts to a few milliseconds. The fact that the twitch lasts for an estimated 100 ms means that excitation (a few milliseconds) can occur at a frequency that results in twitch summation since each twitch takes about 100 milliseconds they can sum together to produce tetany.

\begin{figure}[!ht]
    \centering
    \includegraphics[width=1\linewidth]{./figure/eac-latency.png}
    \caption{EAC Latency \footnotesize{(Wikimedia Commons, CC BY 4.0, \href{https://commons.wikimedia.org/wiki/File:The_latent_period_between_the_muscle_action_potential_and_contraction.png}{EAC-Latency})}}
    \label{fig:eac-latency}
\end{figure}

\subsubsection{Twitch to Tetany}
Each excitation activates enough crossbridges to produce a twitch. A twitch is the fundamental unit of active tension. However, a single twitch of active tension is not enough tension when the goal of tension is some sort of voluntary movement or stability. The rise and rate of a twitch depends on the amount and spread of $Ca^{2+}$ released, the available troponin sites, the available crossbridges to activate as well as the amount of elasticity in the series. One mechanism of muscle fatigue is a reduction in the release of $Ca^{2+}$ from the SR. The rate of $Ca^{2+}$ release from the SR can be increased in experimental preparations with caffeine. Low concentrations of caffeine bind to the voltage gated $Ca^{2+}$ channels at the junction of the T-tubule SR and increase $Ca^{2+}$ release during otherwise normal activation. However, high concentrations result in contracture without the need for the normal excitation signal. 

%A twitch can be an experimentally manipulated event. An electrical stimulus is provided to a muscle which results in either, or a combination of, $\alpha$-motor neuron excitation and sarcolemma excitation that excites the SR and activates enough crossbridges to produces enough active tension to produce a measurable force. A twitch can also be initiated by isolated (single) excitation of an $\alpha$-motor neuron, or isolated excitation of an $\alpha$-motor neuron or sarcolemma. 


\paragraph{Tetany}
Repeated excitation results in repeated activation and therefore continued crossbridge activation and repeated twitches (Figure \ref{fig:tetany}). Successive twitches create tension summation and is called tetany. When the force of active tension is measured, such as during a muscle test, it is the result of many muscle fibers being in tetany. The use of twitch frequency to create tetany and regulate active tension is detailed in Chapter \ref{chp:regulation}. 

\begin{figure}[!ht]
    \centering
    \includegraphics[width=1\linewidth]{./figure/tetany.png}
    \caption{EAC Latency \footnotesize{Modified with BioRender.com from \href{https://commons.wikimedia.org/wiki/File:Twitch_vs_unfused_tetanus_vs_fused_tetanus.png}{Wikimedia Commons, CC BY 4.0})}}
    \label{fig:tetany}
\end{figure}

\subsubsection{Removal of Ca from Sarcoplasm}
Removal of $Ca^{2+}$ from the sarcoplasm requires active transport with $Ca^{2+}$ ATPase. This pump is dependent on $Mg^{2+}$ and moves 2 $Ca^{2+}$ into the SR while moving one $H^+$ out at the cost of 1 ATP. Sarcoplasm levels of $Ca^{2+}$ can quickly return to resting levels which is necessary for rapid muscle relaxation and therefore muscle tension regulation and coordination. Situations that result in a delay in the removal of $Ca^{2+}$ from the sarcoplasm can delay muscle relaxation producing extra twitches or spasms (involuntary tetany). Since $Ca^{2+}$ ATPase is dependent on $Mg^{2+}$, a deficiency in $Mg^{2+}$ can create such a situation.

% Start getting read to transition to a more detailed description of excitation

\subsection{Return to Resting (Non excitation) State - RMP}
The process of returning to the resting state (RMP) is not explicitly discussed in the above sequence of membrane excitation events. It is implied, or taken for granted, once we started discussing the excitation of a twitch to the excitation of tetany (the summation of many twitches). Excitation of tetany requires that the axon, axon terminal, motor end plate, sarcolemma and T-tubule return to a resting (non excited) state. Notice, there is no excitation summation, only twitch summation.

\paragraph{}
Excitation occurs quickly because the resting state includes potential energy for excitation. The RMP is a polarized membrane with large ion concentration gradients (just waiting to run downhill with their gradients). This potential energy comes at a cost of spending energy (ATP) to pump ions against a concentration gradient. The RMP enables rapid excitation because ions are waiting with potential energy due to their concentration gradient to move quickly across the membrane (depolarize). 

\paragraph{}
The RMP also enables rapid return to the resting state with potential energy. The movement of $K^+$ and $Cl^-$ ions with their concentration gradient can quickly end excitation enabling the membrane to be quickly returned to RMP for another excitation. However the long term stability of the RMP relies on processes that depend on ion pumps to sustain concentration gradients for a lifetime.\footnotemark\footnotetext{Brain and heart death are situations where the cells of the brain and heart no longer return to the resting state, they no longer repolarize to the RMP, and therefore they can no longer undergo excitation.}
A deeper understanding of excitation and return to RMP, and clinical connections, requires a molecular and cellular understanding of excitable membranes.

% Second lap on excitation - not every step, and more detailed

\section{Excitable Membranes}

An excitable membrane is polarized in its resting state, which is the resting membrane potential (RMP). An excitable membrane can be excited (experience excitation) when the membrane loses its polarity, which is called depolarization. Depolarization is what occurs during an action potential (AP). What makes the membrane able to be depolarized is the fact that it is polarized. What makes the membrane capable of an AP is the fact that it has a RMP. Please excuse the repetition. But since these terms are all used rather interchangeably it is important that the reader develop a familiarity with the concepts with the exchange of language. The use of the various terms in the book is simply a reflection of the unfortunate situation of various terms being used across fields, papers, and other books.

For emphasis, a membrane cannot depolarize unless it is polarized. To understand muscle excitation and regulation\footnotemark\footnotetext{muscles are regulated through variations in excitation} requires understanding how the RMP is established, how the RMP depolarizes with an action potential (AP), and how the AP propagates (spreads as a wave of excitation). Understanding these concepts also opens the door to understanding several additional topics in clinical physiology. 



% Old but not ready to delete so commented out....
%Signal transmission that results in cellular events (including along the membrane) it is called signal transduction. The action potential signal transduction that occurs along the surface of a nerve and muscle membrane is facilitated by the presence of voltage gated channels embedded in the membrane. Signal transduction from a membrane to another membrane is facilitated by ligand gated channels at specific places in membrane. The detailed physiology of excitable membranes such as establishing and maintaining a resting membrane potential, and mechanisms that result in action potentials are important details covered later in the chapter. 
%------------------
\subsection{Resting Membrane Potential}

The RMP requires several particular characteristics and actions of the cell membrane, and in the case of a muscle fiber, the sarcolemma.

\subsubsection{Sarcolemma}

%Semi-permeability arises in part because of the bilayer being hydrophobic (detracts water) and the inside of the membrane being hydrophilic (attracts water) which limits what can freely diffuse through the membrane (for example oxygen and carbon dioxide can freely pass through the the sarcolemma). 

The sarcolemma is a semipermeable phospholipid bilayer membrane that forms a boundary for the muscle fiber. Semi-permeability arises for two reasons. First, because lipid bilayer prohibits the free passage of many molecules (a unique combination of hydrophobic and hydrophilic components). Second, there are transport proteins within the membrane for selective forms of active, facilitated and passive transport.\footnotemark\footnotetext{As a refresher, active transport utilizes protein machines that require ATP and move very specific molecules, typically, against a concentration gradient. Facilitated transport utilizes protein machines to move very specific molecules across the membrane without the need for ATP. Facilitated transport requires a concentration gradient for at least one molecule, but can be coupled with another molecule that moves against a concentration gradient. Meaning, facilitated transport can utilize the potential energy of one molecule to provide the energy needed to move another molecule against a gradient. Passive transport occurs due to a concentration gradients (such as simple or channel guided diffusion).} The membrane separates the intracellular from the extra cellular space. Through selective transport the membrane establishes the conditions for cellular activities within the cell and thus influences (but does not completely establish) conditions outside the cell. Sarcolemma (all cell membranes) are dynamic structures that adapt in response to the intracellular and extra cellular conditions to meet the needs of the cell.

Components of the sarcolemma that are discussed below and that play an important role for establishing the resting membrane potential as well as the action potential are depicted in Figure \ref{fig:cell_membrane}. 

\begin{figure}[!ht]
    \centering
    \includegraphics[width=1\linewidth]{./figure/cell_membrane.png}
    \caption{Cell Membrane Components and their general contribution to RMP and an AP \footnotesize{Created with BioRender.com}}
    \label{fig:cell_membrane}
\end{figure}


\paragraph{Pumps \& Transporters}
$Na^+/K^+$ ATPase pumps are embedded throughout excitable membranes. These protein based molecular machines convert the energy of ATP to the movement of three $Na^+$ outside the cell and two $K^+$ inside the cell with each cycle of the pump (each use of ATP). Given the natural tendency of molecules to move based on a concentration gradient (high concentration to low concentration) the resultant higher concentration of  $Na^+$ outside and higher concentration of $K^+$ inside the cell requires the work of these active transport pumps against the concentration gradient.
The $K^+$ - $Cl^-$ co-transporter (KCC2) utilizes the $K^+$ concentration gradient created by the $Na^+$/$K^+$ ATPase pumps to move $Cl^-$ out of the cell as an exchange with $K^+$.


%\begin{figure}
%    \centering
%   \includegraphics{.\figure/pumps.png}
%    \caption{$Na^+/K^+$ ATPase Pumps \footnotesize{Created with BioRender.com}}
%    \label{fig:pumps}
%\end{figure}


\paragraph{Channels \& Receptors}
Channels are proteins embedded in the cell membrane that allow passage of molecules between the intracellular and extra cellular space. Channels can be gated and therefore only allow passage under certain circumstances. 

Voltage gated channels only allow passage when the membrane potential is at certain values. They may open at a certain value and close a certain period of time later; or they may close at a certain value and open a period of time later. 
Ligand gated channels only allow passage when the channel is activated due to the binding of a ligand (a molecule that can bind). A ligand gated channel is also a receptor, or closely associated with a receptor. 

$Na^+$ channels allow the passage of $Na^+$ with its concentration gradient and can be either ligand gated or voltage gated. $K^+$ channels allow the passage of $K^+$ with its concentration gradient and in nerve and muscle excitable membranes are both non gated and voltage gated. Cl channels allow the passage of $Cl^-$ along its concentration gradient and in the sarcolemma are not gated ($Cl^-$ channels are not depicted in \ref{fig:cell_membrane}, but would resemble the $K^+$ channel to the far right of the figure.).

Receptors are proteins that allow ligand binding. Some receptors are ligand gated channels (a receptor where the activity that binding produces is opening a channel gate). Other receptors perform different activities within the cell. 

%\begin{figure}
%    \centering
%    \in$Cl^-$udegraphics{.\figure/receptors.png}
%    \caption{Receptors \footnotesize{Created with BioRender.com}}
%    \label{fig:receptors}
%\end{figure}

\subsection{Establishing the Resting Membrane Potential}

For our purposes the process of establishing the RMP in the sarcolemma focuses on roles played by the concentration gradients and electrical potential created by $Na^+$, $K^+$ and $Cl^-$. Each of these elemental molecules exist in the body as ions (they carry a charge based on whether the have too many electrons ($Cl^-$) or two few electrons ($Na^+$, $K^+$).\footnotemark\footnotetext{Ca is an ion that has 2 more protons than electrons (hence the 2+ superscript in its abbreviation).} 

\subsubsection{Electrical Potentials \& Equilibrium Potential}

% Consider adding Nernst, and Goldman's equations

Movement of molecules across the cell membrane is influenced by the concentration gradient of the ion across the membrane, the permeability of the membrane for that ion and the and the electrical equilibrium potential across the membrane. When we describe electrical potentials the convention is to speak from inside the cell (intra cellular). A negative potential means at the membrane inside the cell there is a negative charge. A positive potential means that at the membrane inside the cell there is a positive charge. We can, for the sake of understanding, simply invert this and say a negative potential means that at the membrane OUTSIDE the cell has a positive charge; and that a positive potential means that at the membrane OUTSIDE the cell has a negative charge.

When an ion moves across a membrane it carries a charge and creates an electrical potential. An important point is that the electrical potential created at the membrane is not created by the concentration gradient, but from the movement of ions which is based on the concentration gradients.  The $Na^+$/$K^+$ ATPase pumps work at a rate in response to the concentration gradients (speed up if there is a need to, slow down to resting levels when able). But even for long periods of high frequency excitations there is no risk of significant change to the ionic concentrations outside or inside the cell. Thinking that there an absolute change in ionic concentration is required for an action potential is a common misconception about action potentials\cite{silverthorn_uncovering_2002}. When $K^+$ moves across the membrane (out of the cell) it creates a negative potential. When $Na^+$ moves across the membrane (into the cell) it creates a positive potential. When $Cl^-$ moves across the membrane (into the cell) it creates a negative potential. 

To generalize these facts. When a positive ion moves into the cell it creates a positive potential; when it moves out of the cell it creates a negative potential. When a negative ion moves into a cell it creates a negative potential; when it moves out of a cell it creates a positive potential. These potentials are based on the movement of the ions, not on their concentration. 

The electrical potential created at the membrane by the movement of an ion then has an influence on the movement of ions. The negative electrical potential created at the membrane by the movement of $K^+$ out of the cell will eventually resist the continued movement of $K^+$ out of the cell. How negative that negative potential needs to be to stop the movement of $K^+$ out of the cell is dependent on the concentration gradient.\footnotemark\footnotetext{It is also dependent on the temperature (which is kept within tight boundaries in the body), Faraday's constant, and the universal gas constant. From these variables the potential generated by any ion can be determined with the Nernst Equation, which is not be covered here at this time.} The electrical potential that stops the movement of an ion is called the \textbf{equilibrium potential}. Figure \ref{fig:equilibrium_potential} shows the concentration gradient and resultant equilibrium potential for our primary ions of interest ($Na^+$, $K^+$, $Cl^-$). Each equilibrium potential in \ref{fig:equilibrium_potential} is based on the assumption that it is the only ion moving across the membrane. In reality they are all moving across the membrane with varying rates which are based on the permeability of the membrane to each ion at any moment in time.\footnotemark\footnotetext{The Goldman-Hodgkin-Katz (GHK) Equation (model) allows the estimation of the membrane potential based on $Na^+$, $K^+$, $Cl^-$ concentration gradients and membrane permeability (along with several other relevant variables. A calculator for the GHK Equation can be found at \url{https://www.physiologyweb.com/calculators/ghk_equation_calculator.html}.}

\begin{figure}[!ht]
    \centering
    %\includegraphics[width=1\linewidth]{./figure/cell_membrane.png}
    \includegraphics{./figure/equilibrium_potential.png}
    \caption{Generation of Equilibrium Potentials \footnotesize{Created with BioRender.com}}
    \label{fig:equilibrium_potential}
\end{figure}

\paragraph{}
The resting membrane potential is an equilibrium potential. It is the consequence of two physiological situations. First, the presence of large concentration gradients for $K^+$, $Na^+$  and $Cl^-$ across the membrane. Second, the relative permeability of the membrane to these ions. For example, based on Figure \ref{fig:equilibrium_potential}, if $K^+$ was the only ion that the membrane was permeable to at rest the RMP would be equal to the $K^+$ equilibrium potential (about -97 mV).

The concentration gradients for $K^+$ and $Na^+$ are established primarily by the $Na^+$-$K^+$-ATPase pumps. The concentration gradient maintains a large outwardly directed $K^+$ gradient, and a large inwardly directed $Na^+$ gradient. The concentration gradient for $Cl^-$ is maintained primarily through the actions of a $K^+$ - $Cl^-$ co-transporter known as KCC2. Relative to the other ions, there is a moderate inwardly directed $Cl^-$ gradient.

%Change the figure to reduce the size of the gradient vector for Cl-

Since these ions do not pass through the lipid bi-layer of the cell membrane the permeability of the membrane to each ion is tuned by the presence of ion specific channels. Therefore, the relative permeability of the plasma membrane to $Na^+$, $K^+$ and $Cl^-$ is dependent on the number and state (open or closed) of ion-selective membrane channels. Different membranes display different degrees of permeability to different ions, and due to gating mechanisms on many of the ion-selective channels the permeability can be changed (this is a hugely important point considering the entire purpose of an RMP (polarization) is to have the opportunity to generate an AP (depolarization)).

In resting conditions the membrane of nerve and muscle cells are most permeable to $K^+$. In addition to KCC2 allowing movement of $K^+$ out of the cell, there are also non-gated $K^+$ specific ion channels. When combined with the relatively high permeability of $K^+$, the concentration gradient results in a $K^+$ flow out of the cell. As $K^+$ flows out it creates a negative electrical potential. However, the negative electrical potential resists further flow of $K^+$ outside of the cell (and eventually reaches an equilibrium potential). Since $K^+$ has the highest permeability at rest, the resting membrane potential is primarily influenced by the equilibrium potential of $K^+$. It is important to note that while there is no net flow of $K^+$ at the $K^+$ equilibrium potential, $K^+$ is still flowing out of the cell and continues to generate the negative membrane (equilibrium) potential, and $K^+$ continues to be pumped into the cell by the $Na^+$/$K^+$ ATPase pumps. The term "equilibrium" for equilibrium potential means equilibrium in flows (no net flow, no net gain in or outside the cell).

Note from Figure \ref{fig:equilibrium_potential} that the equilibrium potential of $Cl^-$ is close to $K^+$. The implication is that $Cl^-$, despite its concentration gradient encouraging inward flow, does not at RMP, have net inward flow. $Cl^-$ has a higher permeability than $Na^+$ at rest, but a much lower permeability than $K^+$. An important attribute about $Cl^-$ for the sarcolemma and axon membranes is that the permeability of the membrane does not change much, even during an AP. This contributes to the stability of the RMP and helps prevent recurrent self excitation in these membranes.

The equilibrium potential of $Na^+$ is positive (opposite of $K^+$ and $Cl^-$). At RMP the permeability for $Na^+$ is very low, and therefore $Na^+$ has very little impact on the RMP. The importance of $Na^+$ starts when gated $Na^+$ channels start to open - that's when the action begins.

% Old text - don't delete yet
% Under normal, healthy, conditions different cell membranes have slightly different resting membrane (equilibrium) potentials. These differences are due to differences in the permeability of the membranes to $K^+$ (or to other ions, such as $Na^+$, $Cl^-$, $Ca^{2+}$). The extra cellular concentration of these ions is held constant throughout the body and therefore under normal circumstances differences in the concentration gradients are not a major source of differences in resting membrane potentials. 

\subsection{Trigger \& Spread of Sarcolemma Excitation (AP)}

As can be noted in Figure \ref{fig:equilibrium_potential} the equilibrium potential for $Na^+$ is positive. When the membrane relative permeability for $Na^+$ increases (becomes greater than $K^+$ and $Cl^-$) the membrane potential starts to depolarize and even to switch to a positive polarity. This is an action potential. At the molecular level, an action potential is the sudden increase in relatively permeability of the membrane to $Na^+$.  

\paragraph{Triggering the Action Potential (Initial Excitation)}
The events leading to an action potential on the sarcolemma are initiated by the opening of ligand gated $Na^+$ channels by the binding of ACh to these receptors at the motor end plate. The micro potentials created by the opening of the ligand gated $Na^+$ channels by the binding of ACh then open voltage gated $Na^+$ channels due to a change in voltage in the membrane potential (becoming less negative). The number of voltage gated $Na^+$ channels that open is proportional to the change in the membrane potential. This cycle of opening in response to a voltage change is self perpetuating since once a $Na^+$ channel opens $Na^+$ flows into the cell due to the $Na^+$ concentration gradient. Due to the movement (not a change in the actual gradient) the membrane potential voltage changes (becomes less negative) as the potential moves toward the $Na^+$ equilibrium potential. The point that approximately half of the voltage gated $Na^+$ channels open is considered a threshold potential. At the threshold potential the voltage gated $Na^+$ channels continue to rapidly open until the equilibrium potential approaches that of $Na^+$ (as opposed to $K^+$) since at this point the membrane is more permeable to $Na^+$. The equilibrium potential for $Na^+$ is approximately 66 mV.  Once a membrane reaches the threshold potential an action potential is inevitable. It is at this point that the action potential is said to be “all or none.” Prior to the threshold potential membranes can undergo micro potential changes. These micro potentials can be depolarizing (excitatory potentials that move the membrane potential toward the threshold), or hyperpolarizing (inhibitory potentials that move the membrane potential further away from the threshold potential). The use of micro potentials to regulate muscle tension is an important part of Chapter \ref{chp:regulation} since it is a process that occurs at the $\alpha$-motor neuron dendrites in the spinal cord, not at the NMJ.\footnotemark\footnotetext{An $\alpha$-motor neuron, once excited, will release only one neurotransmitter, ACh at its axon terminal. ACh will only have one effect at the NMJ, creating excitatory micro-potentials on the motor end plate. However, in the Central Nervous System (such as the Spinal Cord) competing neurotransmitters can be released at synapses of the $\alpha$-motor neuron dendrites. The impact of these neurotransmitters can be excitatory or inhibitory. Since they occur after the synaspse they are called excitatory post synaptic potentiations (EPSPs) or inhibitory post synaptic potentiations (IPSPs). Whether, and how frequently, an $\alpha$-motor neuron sends an excitation to the muscle fibers it innervates is balanced by competition between the EPSPs and the IPSPs.}

\paragraph{Spread of the Action Potential (Wave of Excitation)}

The wave of excitation of the AP across the sarcolemma and T-tubules occurs due to the self-perpetuating opening of voltage gated $Na^+$ channels. When a region of the membrane depolarizes it opens the nearby voltage gated $Na^+$ channels which depolarize that region, which then opens the nearby voltage gated $Na^+$ channels which depolarize that region, "And so it goes."\footnotemark\footnotetext{A quote for other Kurt Vonnegut fans.)}

Of course, there cannot be an AP if there is no RMP. Therefore, the AP must end. The membrane must repolarize to be ready for another depolarization. Considering muscle tension is, partly, regulated by how frequently it experiences excitation (which can reach upwards of 180 Hz, that is 180 excitations per second, for those excitations to occur there need to be an equal number of repolarizations).

The process of repolarization occurs in two steps. Voltage gated $Na^+$ channels have two gates. One is an activation gate and it is closed under resting conditions and opens during excitation. The second is an inactivation gate and it is open under resting conditions but closes soon after the activation opens. The inactivation gate closure is timed, it is not triggered by events outside of the molecule itself. Closing the inactivation gate is the first step to stopping an action potential. The closure of the inactivation gates reduces the relative permeability of $Na^+$. While the activation gate is open and the inactivation gate is closed the membrane is in the absolute refractory period (cannot undergo depolarization regardless of how great an excitation stimulus may be). To be reset and completely ready for another action potential the activation gate must close and the inactivation gate must open (these are timed events).

The second step (really second part, do not think of these in an overly temporal manner since events overlap in time) to repolarization is opening of voltage gated $K^+$ channels. These are delayed (take longer to open than the voltage gated $Na^+$ channels). Once the voltage gated $K^+$ channels open the membrane permeability to $K^+$ increases higher than under resting conditions which further expedites the repolarization process and can also result in a hyper polarized state (slightly more negative membrane potential than the normal resting membrane potential). Closing the voltage gated $K^+$ channels is also a timed event. While they are open the membrane is in a relative refractory period. It is resistant to excitation, however, with an increased stimulus it can be excited.

The overall process of repolarization and return to the resting membrane potential takes approximately 3-4 milliseconds (ms). Assuming 3 ms is the fastest interval for another AP, the upper limit on excitations would be approximately 333 Hz (which exceeds the maximum capacity for muscle fiber twitches per second).

Spread of excitation occurs in one direction (meaning it does not result in cyclic repeating action potentials of the entire membrane) because of the refractory periods. If a membrane is visualized as a set of dominoes and the resting potential is when the dominoes are upright, and an action potential is when a domino falls, then the refractory period is the period when the domino is still fallen. Because of the fallen domino on one side of the next falling domino, the dominoes fall in one direction away from the starting stimulus. A further safeguard against cyclic repeating actions potentials in the sarcolemma is the constant permeability for $Cl^-$ in the sarcolemma. Recall that at RMP $Cl^-$ has no net flow because its equilibrium potential is near the $K^+$ potential. When the membrane potential fluctuates away from the the $K^+$ (and $Cl^-$) equilibrium potential there is net movement of $Cl^-$. This $Cl^-$ net movement (into the cell) contributes to repolarization but also to stability of the RMP in the moments following an AP. In situations such as myotonia congenita altered membrane $Cl^-$ permeability leads to a higher occurrence of myotonia (sustained muscle tetany) due to recurrent muscle excitation from cyclic repeating action potentials \cite{adrian_action_1976}.

\subsection{Excitation to Regulation}

This section has taken a deeper dive into the molecular events of excitation than the previous section. It should be clear that many events must occur for one excitation, particularly since one excitation creates a twitch, and a twitch is not enough to produce meaningful movement. A more efficient system of excitation-activation coupling exists in the cardiac muscle. However, in the skeletal muscle the efficiency of creating tetany from one excitation is sacrificed for the sake of regulation of muscle tension. Since creating tetany requires many action potentials, then the amount of tetany, and therefore the amount of tension, can be regulated by the number of action potentials. The regulation of muscle fibers is the topic for Chapter \ref{chp:regulation}.


\section{\textit{Clinical Physiology Connections}}

The concepts of excitable membranes, receptors, ligand channels and signal transduction are pervasive throughout clinical physiology. This final section of the chapter introduces some clinical physiology connections that are now possible given your understanding of excitable membranes, including the neuroendocrine system, sensory receptors and pharmacodynamics. These topics return throughout the book in particular contexts.

\subsection{Neuroendocrine}

The neuroendocrine system regulates several critical physiological systems and coordinates cellular functions throughout the body. Examples include mobilizing the body for fight or flight reactions by getting all cells in the body ready to meet demands that require energy mobilization at the expense of other maintenance activities. For example, release of hormones for fight or flight signals liver cells to hold off on using energy to maintain the cell membrane and instead use energy to convert glycogen (a stored fuel) into glucose (a usable fuel). The neuroendocrine system can alert cells all over the body to do activities that each of those cells can do for the fight or flight situation, even though those activities are varied between the cells. While the liver cells are releasing glucose, the heart cells are prepared to use more glucose and are working more frequently (higher rate) and harder (more pressure). The differentiation and variety of tasks completed by these cells is due to the specialization of the cells, and due to the receptors around the body that are responding to the same ligand. Most cells in the body have a ligand receptor for epinephrine (a major catecholamine of the fight or flight response). Epinephrine is also called adrenaline. Receptors for epinephrine are therefore called adrenergic receptors. The adrenergic receptors on all cells have a similar binding location for epinephrine, but the receptor itself transducts different signals into the cell depending on the receptor and depending on the cell. For example, the adrenergic receptors on smooth muscle in arterioles in the systemic circulation signal contraction which results in vaso-constriction. However, the adrenergic receptors on smooth muscle in bronchioles in the lung signal relaxation, which results in broncho-constriction.

The fight or flight (alert) response from the neuroendocrine is a coordinated response of the component of the central nervous system called the autonomic nervous system, and in particular the sympathetic nervous system (SNS). The SNS is a truly neuroendocrine system because it has both nervous system and endocrine components. Ligands for the SNS, epinephrine and norepinephrine, are also both neurotransmitters and hormones.

For homeostasis (balance), if there is a fight or flight (alert) system then there should also be a rest, digest and reproduce (chill) system. The chill system is the parasympathetic system (PNS). The PNS is a nervous (but not neuroendocrine) system. The connections through the body of the PNS are all mediated with nerve function and neurotransmitters (not endocrine, or hormone signals). This is not to say that the during a chill situation hormones are not involved. Just that if there are pathways through which the PNS mediates those chill (rest, digest, reproduction) hormones they have not been discovered yet (at least as far as the author knows as of \today). The primary neurotransmitter for the PSN is acetylcholine (ACh). The PNS receptors that bind ACh are called cholinergic receptors. 

\subsection{Sensory Receptors}

Sensory receptors are specialized receptors at the terminal end of the axon of a sensory neuron. They differ from ligand and voltage gated receptors in the signals that provokes their excitation. When a sensory receptor is excited it sends an afferent signal (to the central nervous system, as opposed to efferent signal, away from (or exiting) the central nervous system). In general, the signal the provokes excitation in a sensory neuron is whatever phenomenon or sensation that sensory neuron is associated. If the sensory neuron is excited by changes in temperature then it is a temperature sensory neuron; if by changes in tension, it is a tension sensory receptor; if by vibrations, it is a vibration (as in the ear) sensory receptor; if by photons, it is a light sensory receptor (as in the retina); if by certain chemicals or ions (such as $H^+$ for pH, or carbon dioxide, or oxygen) they are chemoreceptors; if by changes in pressure they are baroreceptors. Some sensory receptors terminate in areas of the brain that allow perceptions of the signal in a way that gives them additional meaning. For example, some chemoreceptors proceed to the brain and produce pain signals, once in the brain a complicated cascade of connections are made that can, over time, change from being protective to themselves harmful to movement. Other chemoreceptors that detect an increase in carbon dioxide proceed to areas in the nervous system to produce unconscious reflex responses (changing respiratory rate) and also to the brain (a general state of anxiety and air hunger). Some receptors don’t go to the brain at all, but interact at areas in the nervous system where responses are automatic (baroreceptors as part of the autonomic nervous system to regulate blood pressure). Since baroreceptors don’t go beyond the brain stem we do not perceive what our blood pressure is directly - making elevated blood pressure a “silent killer.” To be a homeostatic regulatory system there needs to be a sensory receptor of some type that contributes to the monitoring of the homeostatic variable. So a homeostatic regulatory system includes a homeostatic variable that is sensed by a sensory receptor that sends an afferent signal to be processed (often not perceived), which then sends an efferent signal to exert an action which aims to adjust the homeostatic variable. The neuroendocrine system is commonly involved in such systems, receiving afferent signals, and sending efferent signals. Most physiological homeostatic systems are negative feedback systems, which means sensation of an increase in a variable triggers response to decreases that same variable (and vice versa).

\subsubsection{Sensory Receptor Dynamics}
Sensory receptors are proteins embedded in cell membranes. Therefore, the cell has control over some dynamic and important behavior of these receptors (indeed, all receptors).  Up-regulation is a term for sensory receptor dynamics that includes an increase in the number of receptors. Down-regulation is a term for sensory receptor dynamics that includes a decrease in the number of receptors. With up-regulation or down regulation there are changes to the strength of a excitation and therefore signal sent to the respective neurons. For example, with up-regulation of chemoreceptors for carbon dioxide an individual will have a higher respiratory rate and maintain a lower carbon dioxide level since their response to carbon dioxide is increased. The up-regulation then continues to propagate itself. Since it results in lower carbon dioxide levels, the receptors remain up-regulated in an attempt to be able to respond to the lower carbon dioxide levels. A balance point is eventually achieved. But this new balance point may be enough to result in what is converted altered homeostasis. In individuals taking a beta-blocker (beta-adrenergic blocker medication) the lack of cardiac response to neuroendocrine stimulus results in higher adrenaline levels. The adrenergic receptors of the heart down-regulate in response to higher circulating adrenaline which further blunts cardiac responsiveness to adrenergic stimulation. A balance point is eventually achieved as with the previous example, however, the individual has an alteration to homeostasis. 

\subsubsection{Neuromuscular Sensory Receptors} 
There are several important neuromuscular sensory receptors that sense signals from the muscle, musculotendonous junction and/or the tendon. Signals for tension (such as golgi tendon organs), signals for changes in length (such as the muscle spindle), and signals for the chemical constituents of the extra cellular fluid surrounding muscle fibers (chemoreceptors). These receptors and their role in muscle function is covered in upcoming chapters on regulation, energetics and micro-circulation.

\subsection{Pharmacodynamics}

Pharmacology is divided into pharmacotherapeutics and toxicology. Toxicology focuses on the harmful effects that chemicals, such as medications, may have on the body. Pharmacotherapeutics is the development and study of medications. Pharmacotherapeutics is divided into pharmacokinetics and pharmacodynamics. Pharmacokinetics focuses on the absorption, distribution and metabolism of medications. Pharmacodynamics focuses on the intended systemic and cellular effects of medications. Pharmacodynamics is therefore a study of the actions of medications in the body. The cellular and systematic effects of a medication is often related to what receptors it binds, what neurotransmitter or hormone it effects of mimics, or what signal transduction mechanism it influences. Pharmacodynamics often includes changes to the process of excitation discussed in this chapter.

\subsubsection{Curare}
Curare is paralyzing agent. It stops the excitation of the motor end plate. It does so by competitively and reversibly binding to ACh receptors on the motor end plate which blocks those receptors from binding with ACh. When curare binds to an ACh receptor it does not open the ligand gated $Na^+$ channel and therefore prevents the initiation of depolarization. As a competitive binding, curare blocks the ACh receptor and thus inhibits the ACh binding and response. Another term for competitive binding is that curare is an antagonist (blocks without producing an effect). Some medications are agonists, they bind and create an effect and are thus utilized to facilitate additional cellular activity beyond that of the normal ligand. As a reversible binding medication, curare can be removed from the ACh receptor. Medications can also create irreversible binding. Curare is also selective, it only binds to skeletal muscle motor end plate ACh receptors. Which means it selectively inhibits these receptors and not other ACh receptors throughout the body. This means the systemic response is limited to skeletal muscle paralysis. If used during surgery without other anaesthetics it would produce paralysis but no change in pain sensation, a very unpleasant surgery that would be stopped when the pain and anxiety of the procedure produced a rapid heart rate and rising blood pressure due to the fight or flight response of the neuroendocrine system (which would not be stopped by curare). In addition to surgical or intensive care situations, curare can be used as a poison, causing death by asphyxiation since the muscles of ventilation (breathing) are skeletal muscles. 

\subsubsection{Medication Induced Receptor Dynamics}
A consequence of medications is that they can produce changes in receptors. Up-regulation and down-regulation of receptors due to their pharmacodynamics are a common reason for the need to carefully, and only under supervision, withdraw a dose of medications. If the medication led to up or down regulation and depending on whether it is a receptor antagonist or agonist, could have big implications with sudden cessation. For example, if a medication leads to down regulation of receptors, cessation of that medication could result in a large drop in cellular activity (whatever that activity may be). For example, serotonin re-uptake inhibitors are antagonists that block the re-uptake of serotonin in synapses of the central nervous system. This increases the amount of serotonin which can lead to down-regulation of the serotonin receptors. If the medication is suddenly stopped there will be less serotonin and less receptors (due to down-regulation) which magnifies the effect of having less serotonin, potentially worsening of the original reasons for taking the medication. 

\subsubsection{Autonomic Nervous System Medications}

There are several medications that influence the autonomic nervous system function for a variety of health conditions. Applying what we know about receptors and pharmacodynamics it should be clear that an adrenergic agonist will increase SNS activity in the cell that the medication binds to, whereas an adrenergic antagonist (blocker) will reduce the SNS activity in the cell that the medication binds. A cholinergic agonist will increase the PNS activity in the cell, whereas a cholinergic antagonist will decrease the PNS activity in the cell. Very simple. Two types of receptors that can effect the receptors in two ways and therefore can create four different cellular responses. 

Thankfully there is enough cholinergic receptor specificity that a cholinergic agonist with the therapuetic goal of changing gut motility is specific to the cholinergic receptors in the gut. If all cholinergic receptors were the same, then such a cholinergic agonist could cause wide spread muscle spasms.

\section{Summary \& Next Steps}

Muscle fiber excitation starts with an excited motor axon ($\alpha$-motor neuron) and ends with the binding of calcium to troponin, which generates a twitch (the fundamental unit of active tension). Binding calcium to troponin allows a crossbridge to go from an inactivated to an activated state. The final step in muscle excitation is the coupling between excitation and activation (Excitation - Activation Coupling). The underlying molecular mechanisms involve the characteristics and actions of excitable membranes. Excitable membranes require ion channels, pumps and receptors. Actions include using transport to establish a resting membrane potential and the ability to have an action potential (excitation). Understanding excitable membranes creates Clinical Physiology Connections opportunities in three areas: 1. system wide regulatory function of the neuroendocrine system; 2. function of sensory receptors; and 3. pharmacodynamics.

A twitch is the fundamental unit of muscle fiber active tension. A twitch is activated by excitation of the muscle fiber by the $\alpha$-motor neuron. Excitation of the muscle fiber is the fundamental unit of regulating active tension in the muscle fiber. The next chapter considers regulation as the process of varying muscle fiber excitation - both the frequency of excitation as and the number of muscle fibers excited.


\printbibliography[heading=subbibintoc]




% !TEX root = ../notes_template.tex
\chapter{Regulation}\label{chp:regulation}
Updated on \today
\minitoc

This chapter introduces the mechanisms utilized by the central nervous system (CNS) to regulate active tension. To regulate active tension means to vary its force from its lowest value (resting tone) to its maximal value (maximal voluntary contraction (MVC)). Active tension is developed at the level of the sarcomere, excitation at the level of the muscle fiber, regulation occurs at the level of motor units. A motor unit includes an $\alpha$-motor neuron and all of the muscle fibers it innervates. The CNS has two strategies to regulate active tension. One strategy is to manipulate the number of twitches per second (frequency summation or rate coding); and the second strategy is to manipulate the number of motor units twitching (motor unit summation or motor unit recruitment). Between these two approaches there is a nearly continuous variation in active tension possible, with some muscles (fine motor) having a greater capacity for continuous variation than others (gross motor).

\vspace{5mm}

\textbf{Objectives include:}
\begin{enumerate}
    \item Explain motor unit structure–function relationships.
    \item Explain muscle fiber differentiation (types) structure-function relationships.
    \item Explain motor unit excitation, twitch and tetany.
    \item Compare and contrast frequency summation (rate coding) and motor unit summation (recruitment).
    \item Explain the order of recruitment.
    \item Explain how the muscle spindles and the golgi tendon organs influence motor unit excitation.
    \item Explain the physiological basis of the electromyogram (EMG) and limitations of EMG interpretation.
    \item Explain the effect of amyotrophic lateral sclerosis and aging on motor units.
    \item Explain the effect of peripheral nerve demyelination on muscle regulation.
\end{enumerate}\

\section{Motor Units}

A motor unit is an $\alpha$-motor neuron and all of muscle fibers that its terminal axon branches form a neuromuscular junction at a motor end plate. Each muscle fiber has one motor end plate, and therefore one neuromuscular junction. However, each $\alpha$-motor neuron has many terminal axon branches. The muscle fibers of a single motor unit are randomly dispersed throughout the muscle epimysium. Therefore, the muscle fibers of a motor unit contribute to the tension developed in several fascicles. This arrangement ensures that when only a few motor units are excited there are muscle fibers throughout the entire muscle generating tension at the tendon attachments to avoid asymmetrical pull from one fascicle or isolated location within the epimysium.

\begin{figure}[!ht]
    \centering
    \includegraphics[width=1\linewidth]{./figure/motor_unit.png}
    \caption{Motor Unit Distribution in a Muscle \footnotesize{\href{https://commons.wikimedia.org/wiki/File:Motor_unit.png}{Wikimedia Commons, CC BY 4.0})}}
    \label{fig:motor_unit}
\end{figure}

\subsection{Innervation Ratio}
Motor unit innervation ratio (IR) refers to the number of muscle fibers per motor unit and is usually given as the average number of muscle fibers per motor unit.

\begin{equation}
    IR = \frac{N_{mf}}{N_{mu}}
\end{equation}

The IR varies between muscles and differentiates muscles equipped for fine or gross motor control (See Table \ref{table:Innervation_Ratios}). IRs vary from 5 in muscles in the eye (i.e. lateral rectus) to approximately 1700 in muscles in the calf for locomotion (i.e. gastrocnemius).

\begin{table}[h!]
\centering
\begin{tabular}{||c c c c||} 
 \hline
 Muscle & $\alpha$-motor neurons & Muscle Fibers ($\times 10^3$) & Innervation Ratio \\ [0.5ex] 
 \hline\hline
 Lateral Rectus (eye) & 30* &.15* & 5 \\
 Lumbricals (hand) &  98 & 10.3 & 105 \\ 
 Brachialis & 330 & 130 & 393 \\
 Biceps  & 774 & 580 & 749 \\ 
 Gastrocnemius & 580 & 1000 & 1724 \\[1ex] 
 \hline
\end{tabular}
\caption{Variability of Innervation Ratio in Human Muscles (\footnotesize{From \cite{buchthal_motor_1980}, (* Estimates that need confirmation)})}
\label{table:Innervation_Ratios}
\end{table}

IRs for whole muscles are explanatory for the concept of fine motor and gross motor control. Considering tension is developed by muscle fibers, the number of muscle fibers recruited with each new motor unit recruited influences the size of jumps in active tension. Muscles with lower IRs have greater regulation (control) over the amount of tension they produce because they better accommodate continuous increments in tension with each new motor unit recruited. Of five muscles in Table \ref{table:Innervation_Ratios}, the lateral rectus (eye) and lumbricals (muscle in the hand) have the greatest fine motor control. While the gastrocnemius is responsible for the greatest need for high tension (force) output (gross motor control).

\subsection{Motor Unit \& Muscle Fiber Types}

While IRs for whole muscles help explain variation between muscles with differences in motor control they ignore the systematic variation in the number of muscle fibers per motor unit within a muscle. There are different types of motor units. The types include differences in the $\alpha$-motor neurons, the muscle fibers and the IR. The differences are structural and functional. Types exist on a spectrum with transitions in many (but not all) of the characteristics between them. Despite the spectrum, and using different characteristics, experts have identified and agree on classification systems that converge on three types of human motor units and associated muscle fibers \cite{lieber_skeletal_2010}.

\begin{itemize}
    \item Fast Fatigable (FF) Motor Units with Fast Glycolytic (FG) (Type 2a) Muscle Fibers
    \item Fast Resistant (FR) Motor Units with Fast Oxidative (FOG) (Type 2x) Muscle Fibers
    \item Slow (S) Motor Units with Slow Oxidative (SO) (Type 1) Muscle Fibers
\end{itemize}

The above classification names are all utilized, however the Type 1, 2x and 2a are probably the most common. The other classifications are informative about the functional characteristics of the motor units and muscle fibers they innervate. Beyond their coherence, the motor unit classifications FF, FR and S share a feature with the muscle fiber classifications FG, FOG and SO. They both use the letters F and S to represent Fast and Slow, and they primarily refer to faster or slower excitation (and in the case of muscle fibers also activation). 

The second letter F in FF, and the letters R and S (dual use of the meaning slow for the Slow motor units), and the letters G and O for muscle fibers are related to functional characteristics of fatigue and energetics. For motor unit classifications the second F in FF stands for fatigable, meaning FF fibers are fatigable and cannot sustain tetany for very long. The R in FR motor units stands for resistant, as in fatigue resistant. They resist (for a time) the fatigue of FF motor units. And the S motor units are slow, both in excitation, and slow to fatigue (hence the "dual use" of slow). In Chapter \ref{chp:energetics} on Energetics, the concept of fatigue denoted in muscle fiber classification letters G and O refer to the biochemical pathways that produce ATP, glycolosis and oxidative (including both the citric acid (Kreb's) cycle and electron transport). The connection between FF and FG, FR and FOG, and S with SO, has to do with the rate and sustainability of ATP production by these pathways. Glycolosis has a relatively high rate and low sustainability, and oxidative pathways have a relatively low rate and high sustainability. These pathways, but not just these pathways, result in the fatigability, fatigue resistant or slow to fatigue characterizations of motor units and their fibers. While fatigability comes up in this chapter, the focus is on the excitation and activation characteristics of the fast and slow motor units and muscle fibers.

\subsubsection{Fast vs. Slow - An excitable concept}

The terms fast and slow refer to relative differences in functional characteristics between $\alpha$-motor neuron and the muscle fibers. The functional characteristics fast and slow are associated with a set of functionally coherent structural differences. The section below focuses on the most distinguishing structural characteristics that influence the functional characteristics of excitation and activation.

\paragraph{Fast \& Slow $\alpha$-Motor Neurons}
Fast and slow for the motor unit $\alpha$-motor neuron refers to how quickly excitations can be sent which influences how frequently they can be excited. The faster an excitation propagates down an axon the sooner it is ready for another excitation. In a sequence of excitations, if a second excitation catches up to the first excitation it will cease propagation once it reaches a section of the axon membrane that is its refractory period from the first excitation. A fast neuron has a higher nerve conduction velocity (NCV) and a higher maximum excitation frequency (Hz) than a slow neuron.

NCV is related to the size of the neuron (diameter) because diameter influences resistance to current. Current is the flow (flow is a rate) of electrical impulses and therefore directly related to velocity. Diameter is inversely proportional to resistance, and resistance is inversely proportional to current, therefore diameter is directly proportional to current and velocity (See Figure \ref{fig:Current_Resistance}). The large diameter of the $\alpha$-motor neurons, combined with the myelin sheaths (The role of myelin was introduced in Chapter \ref{chp:excitation}), contribute to its relatively high NCV. 

\begin{figure}[!ht]
    \centering
    \includegraphics[width=1\linewidth]{./figure/Current_Resistance.png}
    \caption{Relationship between Diameter, Resistance and Current \footnotesize{(Created in BioRender.com)}}
    \label{fig:Current_Resistance}
\end{figure}

Estimated NCVs for three nerve fiber types are presented in Table \ref{table:NCV}. The $\alpha$-motor neuron has the highest NCV. The variation in diameter in the $\alpha$-motor neuron is explained by the motor unit types with the fast (FF) motor units innervating the FG fibers having the larger diameter ($\simeq 22 \mu m$) on the higher end for NCV ($\simeq 120 m \cdot s^{-1}$), and the S motor units with the small diameters ($\simeq 12 \mu m$)and the lower end of NCV ($\simeq 70 m \cdot s^{-1}$) (and FR motor units innervating FOG somewhere in the middle).  The terms fast and slow for the $\alpha$-motor neurons of motor units are relative to one another. It is clear in Table \ref{table:Innervation_Ratios} that both fast and slow $\alpha$-motor neurons are faster than the other neurons listed. They are faster due to both diameter (as compared to $\delta$-sensory neurons) and additionally myelination (as compared to C sensory neurons).

\begin{table}[h!]
\centering
\begin{tabular}{||c c c c ||} 
 \hline
Nerve Fiber & Diameter (\mu m) & NCV ($m \cdot s^{-1}$) & Function \\ 
 \hline\hline
 $\alpha$-motor & 12-22 & 70-120 & motor \\ 
 $\delta$-sensory & 1-5 & 12-30 & sharp pain \\
 C & 0.5-1.2 & 0.2-2 & dull pain \\ [1ex] 
 \hline
\end{tabular}
\caption{NCV for different nerve fiber types. Note: C fibers are unmyelinated, whereas $\alpha$-motor and $\delta$-sensory fibers are myelinated. Therefore, C fibers have a larger drop in NCV than would be estimated just from diameter. \footnotesize{Data from \cite{feher_quantitative_2017}}}
\label{table:NCV}
\end{table}

A fast $\alpha$-motor neuron has a larger diameter and myelin (structure) which both contribute to the NCV (function). Since the neurons of slow motor units are also myelinated the primary structural differentiation between fast and slow motor unit neurons are their size (larger diameters in fast motor unit neurons). Structurally, these larger neurons also innervate a higher number of muscle fibers. In mammalian muscles the IR for FF motor units is approximately twice (2x) that of S motor units, but this is an estimate that varies considerably between muscles (range from 1.1x to 3x greater than S motor units) \cite{bodine_maximal_1987}. 

A consequence of a larger neuron is that it requires more CNS excitation to become excited (excitation inertia). Excitation inertia has to do with how many ligand voltage gated channels must be opened before the threshold potential is reached. The larger (and thus faster) neuron has more excitation inertia. Fast $\alpha$-motor neurons are larger. While fast, they take more excitation to be excited. This is an important feature for motor unit recruitment and the size principle of recruitment order covered later in the chapter.

\paragraph{A note on motor unit and muscle fiber integrity}
Fast (larger) and slow (smaller) $\alpha$-motor neurons play a role for motor unit and muscle fiber integrity. The narrative of this chapter is that there is coherence between the neuron doing the excitation, and the muscle fibers being excited so that they can be activated. The neuron excitations play a critical, causal, role in determining the type of muscle fiber \cite{buchthal_motor_1980}. It is not a coincidence that faster and larger motor neurons innervate faster and larger muscle fibers. Studies using electrical stimulation and crossover (changing which neurons innervate which muscles), many muscle fiber characteristics change from slow to fast or from fast to slow. These experiments conclude that there is plasticity at the level of muscle fiber type with such experimental manipulations, and that this plasticity is related to the differences in electrical (not chemical) stimulus. The challenge to anyone wishing to alter their fiber type dominance is finding a way to  tap into the plasticity of the neurons.\footnotemark\footnotetext{Certain characteristics such as myosin and myosin ATPase isoforms rely on gene expression (one isoform is expressed or the other is expressed) and therefore may provide a limit to how much, or how many, muscle fibers have this plasticity.}

\paragraph{Summary of Fast \& Slow $\alpha$-Motor Neurons}
The functional characteristics that make the fast $\alpha$-motor neuron of the fast motor unit fast as compared to the slow classification are a higher NCV and a higher excitation frequency. The predominant structural difference that leads to these functional differences is the size of the neuron (larger, larger diameter). These structural and resultant functional differences fit on a spectrum that, in Table \ref{table:NCV}, result in a wide range of neuron diameters. These neuronal excitation characteristics innervate, and actually contribute to, coherence between the neuron and the muscle fibers of the motor unit.


\paragraph{Fast \& Slow Muscle Fibers}

The functional characteristics of muscle fibers that make them fast vs. slow include both structures related to excitation-activation coupling as well as structures related to activation (crossbridge kinetics). 

For muscle fibers in a fast motor unit to respond to the faster and higher excitation frequency of the neurons they require a well developed T-tubule and sarcoplasmic reticulum (SR) system. The SR and T-system of fast fibers can have up to three times the volume of slow fibers. This allows fast motor units to convert the higher frequency of nerve excitation into a higher frequency of muscle fiber activation.

With a higher frequency of muscle fiber activation there must be structural changes that also allow for faster crossbridge kinetics, otherwise the faster activation (faster release and update of $Ca^{2+}$ for example) would not result in faster twitches. The crossbridge kinetic structural changes tend to be discrete rather than on a spectrum.\footnotemark\footnotetext{By discrete we mean that there are no intermediate or transition steps between the structure. This is compared to changes such as the size of a SR or the diameter of a neuron or the number of ACh receptors at a NMJ where there can any number of transitional steps.} Fast fibers express an isoform of the protein myosin (particularly the heavy chain) and the enzyme (also a protein) myosin ATPase (responsible for hydrolyzing ATP on the myosin head). The combination of these two expressions allow for faster crossbridge kinetics (cycling through the sliding filament model) which leads to slightly more tension at the level of the sarcomere and a higher velocity for shortening \cite{larsson_maximum_1993, schiaffino_molecular_1996}. Therefore, when isolated or when either of these types of fibers dominate in a particular muscle, the force-velocity curve is shifted upward in fast as compared to slow motor units, with fast (FF) motor units having a higher velocity for any given force, and a higher force for any given velocity. 

These differences result in higher specific tension ($T_s$) in fast muscle fibers. $T_s$ refers to the tension developed per sarcomere or, more commonly considered, tension developed per unit of cross-sectional area. Since there are more muscle fibers in fast motor units means there are more sarcomeres and a greater cross sectional area, the $T_s$ makes the tension of a motor unit relative to its cross sectional area to allow comparison that avoids the confounding effect of greater cross sectional area.

The molecular (signal transduction) events that lead to the expression of myosin and myosin ATPase isoforms have not yet been completely uncovered.\footnotemark\footnotetext{Though, the author (SC) is still reading about the possibilities.} It is assumed that, due to the plasticity of muscle fiber types reported in experimental conditions, at least some muscle fibers retain the ability to express either isoform and that there exists a molecular mechanism that can theoretically be switched \cite{schiaffino_molecular_1996}.

\paragraph{Summary of Fast \& Slow Muscle Fibers}
The functional characteristics of muscle fibers for fast motor units allow for faster excitation-activation coupling and faster crossbridge kinetics. The structural characteristics that enable these include a higher volume of T-tubules and SR (for activation) and an isoform of myosin and myosin ATPase (for crossbridge kinetics). These changes result in an upward shifted force-velocity curve for fast motor units as compared to slow motor units (Figure \ref{fig:fiber_type_FV}). 

\begin{figure}[h]
    \centering
    \includegraphics[width=1\linewidth]{./figure/fiber_type_FV.png}
    \caption{Force and Shortening Velocity of FF/FG and S/SO Motor Units \footnotesize{(Created in BioRender.com, Modified from \cite{feher_quantitative_2017})}}
    \label{fig:fiber_type_FV}
\end{figure}


\section{Motor Unit Twitch \& Tetany}

A single excitation of a muscle fiber produces enough activation to create a twitch, making a twitch the fundamental unit of active tension in the muscle fiber. The concept of a muscle fiber twitch is scaled up to a motor unit twitch. A muscle fiber twitch is the fundamental unit of active tension, a motor unit twitch is the fundamental unit of movement.\footnotemark\footnotetext{Just a note that by "unit" we don't mean unit as in a unit of measurement. Other words for how we are using the word unit may be "element" or "atom" or "particle".} Movements are created by muscle \textit{in situ}, and this means muscle fiber excitations and twitches occur in response to motor unit excitations, and the active tension developed by muscles that results in movement are based on motor unit twitches. Motor units have fidelity. When the neuron of a motor unit is excited all of the muscle fibers of the motor unit are excited. Recall from the data in Table \ref{table:Innervation_Ratios} that for the gastrocnemius this means a motor unit twitch includes approximately 1720 muscle fiber twitches. The fibers that are parallel increase the cross sectional area (A) and the tension.

\subsection{Motor Unit Type - Twitch \& Tetany Characteristics}

The three motor unit types display different twitch tension, twitch time and fatigue index (See Table \ref{table:Motor_Unit_Types}). These measurable functional characteristics are the result of the structural differences of both the motor unit neuron and the associated muscle fibers discussed in the previous section.

\begin{table}[h!]
\centering
\begin{tabular}{||c c c c||} 
 \hline
 Motor Unit Type & Twitch Tension & Twitch Time & Fatigue Index \\ [0.5ex] 
 \hline\hline
 Fast-Fatigable (FF)  & High & Fast & Low \\ 
 Fast-Resistant (FR)  & Moderate & Fast & Moderate \\
 Slow (S) &  Low & Slow & High \\ [1ex] 
 \hline
\end{tabular}
\caption{Motor Unit Types}
\label{table:Motor_Unit_Types}
\end{table}

\paragraph{Twitch Time}
Twitch time refers to the total time of the twitch with includes its upslope, time at peak or plateau, and time for downslope. A FF motor unit has a fast rise, short lived peak, and fast downslope. An S motor unit has a slower rise, longer plateau, and slower downslope. These characteristic curves (See Figure \ref{fig:mu_twitch}) contribute to the excitation frequency required for achieving tetany in the different motor unit types. An S motor unit can achieve tetany with an excitation frequency of 20-30 Hz since the twitches last longer and fuse (summate) more easily. A FF motor unit requires a higher frequency (80-100 Hz) to achieve tetany. 

\begin{figure}[!ht]
    \centering
    \includegraphics[width=1\linewidth]{./figure/mu_twitch.png}
    \caption{Motor Unit Twitch Characteristics \footnotesize{(Created in BioRender.com, Modified from \cite{jones_skeletal_2006}})}
    \label{fig:mu_twitch}
\end{figure}

\paragraph{Twitch Tension}
Twitch tension refers to the measurable force of a the tension developed during a twitch. Since tetany includes a summation of the twitches, a higher twitch tension results in a higher tetanic tension as well. Twitch tension is highest in the FF motor units for three reasons. First, these motor units have the highest number of muscle fibers (N, which is directly related to innervation ratio (IR)). Second, these motor units have, overall, a higher cross sectional area. Part of the larger cross area, but not all of it, is due to the higher number of fibers. The IR of a FF motor unit is approximately 2x higher than an S motor unit. However, the cross sectional area (A) is approximately 2.3x higher than an S motor unit \cite{bodine_maximal_1987, buchthal_motor_1980}. Third, these motor units have a slightly higher specific tension ($T_s$). Estimates are that the $T_s$ of FF motor units when tetanized is approximately $22 \ N \cdot cm^{-2}$ compared to S motor units approximately $17 \ N \cdot cm^{-2}$. Based on these estimates the $T_s$ of a FF motor unit is approximately 1.3x that of a S fiber. 

\paragraph{}

The tension of a motor unit twitch $T_{mu}$ is the sum of the motor units muscle fiber twitches $T_{mf}$. The equation:

\begin{equation}
    T_{mu} = N \cdot T_{mf} = N \cdot A \cdot T_s
\end{equation}

\begin{itemize}
    \item $N$ refers to the number of muscle fibers. This either is (for one motor unit) or can be estimated by (for an entire muscle) the innervation ratio.\footnotemark\footnotetext{$IR = \frac{N_{mf}}{N_{mu}}$ (Note, when there is one motor unit, $N = IR = N_{mf}$)}
    \item $A$ refers to the cross sectional area - an indication of how many sarcomeres are in parallel in the fiber.
    \item $T_s$ refers to the tension created for each unit of cross sectional area.
\end{itemize}

Using the above equation and estimates of FF motor units compared to S motor units we can derive an estimate of the FF motor unit $T_{mu}$ relative to the S motor unit. 

\paragraph{}
Assumptions:
\begin{itemize}
    \item FF motor unit IR, and therefore $N$ is 2x that of S motor units (more fibers)
    \item FF motor unit $A$ is 2.3x that of S motor units (greater cross sectional area)
    \item FF motor unit $T_s$ is 1.3x that of S motor units (higher tension per sarcomere)
\end{itemize}

An equation for the $T_{mu}$ for a FF compared to a S motor unit simply adjusts the variables based on a FF motor unit:

\begin{equation}
    T_{FF} = (N \cdot 2) \cdot (A \cdot 2.3) \cdot (T_s \cdot 1.3)
\end{equation}

This equation estimates that a FF motor unit can develop 6x more tension than an S motor unit. If based just on the number of fibers and the cross section area the estimate would be 4.6x. Therefore the number of fibers and the cross sectional area account for 76\% of the difference in tension between the FF and the S motor units. And the specific tension accounts for the remaining 24\%. Based on these estimates it is not surprising that differences in the specific tension between FF and S motor units is an area of controversy. When considering the likely variability between muscles, between samples, between mammalian species, differences as small as 1.3x are difficult to demonstrate \cite{lieber_skeletal_2010}. Some sources equate the $T_s$ between the two fiber types, $T_s \simeq 20 N \cdot cm^{-2}$ \cite{feher_quantitative_2017}, whereas others report them separately with albeit different estimates, FG $T_s \simeq 22 N \cdot cm^{-2}$, and SO $T_s \simeq 15 N \cdot cm^{-2}$ \cite{lieber_skeletal_2010}.

\paragraph{Fatigue Index}
Fatigue Index (FI) refers to how long tetany can be sustained for the FF, FR and S motor units. As Table \ref{table:Motor_Unit_Types} indicates, and Figure \ref{fig:mu_twitch} depicts the FF (fatigable) motor units have a low FI, whereas the S motor units have a high FI.

\subsection{Summary of Motor Unit Twitch \& Tetany}
The structural and functional differences between motor unit types result in expected differences in motor unit twitch and tetany characteristics. These differences lend themselves to the functional movement benefits of each type of motor units. FF motor units are optimized for fast, high force, high power - ballistic and temporary or intermittent movements. S motor units are optimized for slower, lower force and lower power - sustained, postural, stabilizing postures and movements. Further support of this story comes from how the motor units are recruited, that is, how tension is regulated.

\section{Active Tension Regulation}

Active tension is regulated through motor unit excitation. The two approaches include frequency summation (rate coding) of each motor unit and motor unit summation (recruitment) of more motor units. Both of these approaches start in the spinal cord with motor unit excitation.

\subsection{Motor Unit Excitation}

An $\alpha$-motor neuron, once excited, will release one neurotransmitter, ACh, at its axon terminal. ACh has one effect at the NMJ, creating excitatory micro-potentials on the motor end plate. In the CNS, more specifically for motor neurons in the spinal cord, competing neurotransmitters are released at synapses of the $\alpha$-motor neuron dendrites. The impact of these neurotransmitters can be excitatory or inhibitory. Since they occur after the synapse they are called excitatory post synaptic potentiations (EPSPs) or inhibitory post synaptic potentiations (IPSPs). Whether, and how frequently, an $\alpha$-motor neuron sends an excitation to the muscle fibers it innervates involves competition between the EPSPs and the IPSPs. For example, the IPSPs of certain protective reflexes associated with receptors in the muscle (see Golgi Tendon Organs below) can reduce the tension developed in a motor unit muscle fibers due to IPSPs lowering both the frequency of stimulation of a motor unit and the number of motor units recruited.

\subsection{Motor Unit Tetany - Frequency Summation}

Motor unit tetany involves the frequency of excitations. For any given motor unit, across all types, there is a range of excitation frequencies (rate coding) that can be used to vary the tension developed. It is estimated that frequency summation accounts for small fine tuning differences in tension. Frequency summation is not capable of resulting in large variations in tension in large part because it involves a set of muscle fibers that have a narrow functional frequency bandwidth. S type motor units can tetanize as low as 20 Hz, but don't develop any more force beyond perhaps 60 Hz (with a plateau occurring as low as 40 Hz). Similarly, FF motor units don't tetanize until they achieve upwards of 100 Hz and plateau at approximately 140-160 Hz. These differences, assuming linear increases in tension occur as frequency increases, represent a roughly 50\% increase in tension. If someone' grip strength is 100 pounds of force, then getting from 1 to 10 pounds of force requires a 10 times increase, getting from 1 to 100 pounds requires a 100 times increase. These increases in force cannot be accounted for by frequency summation. This has implications in situations discussed later which include a loss of motor units.

\paragraph{}

Frequency summation fine tunes and adjusts the tension in motor units but it is not capable of large fold changes in tension (or force). But, due to the role that excitation plays in motor unit tetany and the importance of tetany of motor unit muscle fibers for functional movements, the differences in the frequency of excitation to achieve tetany between the motor unit types is an important concept as we consider motor unit recruitment.

\subsection{Motor Unit Summation - Recruitment}

The primary way that muscle tension is regulated is through motor unit summation (recruitment). The recruitment of more motor units has the capability of greatly increasing the tension. From the example above, to go from 1 pound to a 100 pound hand grip requires the recruitment of few to perhaps all of the motor units involved in grip strength. Figure \ref{fig:Motor_unit_recruitment} depicts what happens when the three different motor units are recruited, first the SO fibers (S motor units), then the FOG fibers (FR motor units), and finally the FG fibers (FF motor units).\footnotemark\footnotetext{A future version of this figure will change upslope of these curves to represent the Twitch Time characteristics of the motor unit types).}


\begin{figure}[!h]
    \centering
    \includegraphics[width=1\linewidth]{./figure/Motor_unit_recruitment.png}
    \caption{Motor Unit Order of Recruitment \footnotesize{\href{https://commons.wikimedia.org/wiki/File:Motor_unit_recruitment.png}{Wikimedia Commons, CC BY 4.0})}}
    \label{fig:Motor_unit_recruitment}
\end{figure}

\paragraph{Size Principle of Motor Unit Recruitment}

The size principle of motor unit recruitment is based on the experimental observation that low tension requirements are met by first recruiting smaller motor units (S) with their associated SO fibers. As tension requirements increase the motor units recruited increase in size \cite{henneman_rank_1974}. This is due to two interrelated characteristics of the motor units themselves. First, smaller motor units require less excitation potentials to reach a membrane threshold potential. Lower CNS output into the region of the spinal cord (dorsal horn) with the dendrite pool of $\alpha$ motor neurons for a particular muscle first excites small dendrite membranes (they have less excitation inertia). As the CNS output increases in intensity, signaled as an increase in frequency of excitatory neurotransmitter impulses, there is an increase in excitation of the larger motor neurons. The second characteristic is that the larger motor neurons (FR and FF) require higher frequency excitations in order to achieve tetany.

\paragraph{Tension Developed during Motor Unit Summation - Recruitment}

Figure \ref{fig:n_motor_unit_recruitment} is a simulation based on the previously derived estimate that a FF motor unit generates 6x more tension than a S motor unit. This estimate accounts for the higher number of fibers, the greater cross sectional area of the fibers, and the greater specific tension of the fibers in an FF motor unit. Additional assumptions for this simulation include that the muscle has 100 motor units generating up to 100\% of of its maximal active tension; and that there are 20 S motor units, 20 S to FR transition motor units; 20 FR motor units; 20 FR to FF transition motor units; and 20 FF motor units. The S motor units each contribute 1 arbitrary unit (au) of tension, the FR motor units each contribute 3 au of tension, and the FF motor units each contribute 6 au of tension. The transition motor units are scaled linearly from 1 to 3, and from 3 to 6 in their respective transition zones. What is clear is that based on the order of recruitment (S to FF) there is an upward slope of tension with each new motor unit recruited. Jumps in tension as more motor units are recruited get larger as more motor units are recruited. Frequency summation tuning offers can further adjust these jumps. This tuning is more effective at lower tension with the small motor units since there are smaller jumps to accommodate.

\begin{figure}[!ht]
    \centering
    \includegraphics[width=1\linewidth]{./figure/n_motor_unit_recruitment.png}
    \caption{Simulation of Tension Developed during Motor Unit Summation - Recruitment (See text for simulation assumptions) \footnotesize{Created with Excel and BioRender.com})}
    \label{fig:n_motor_unit_recruitment}
\end{figure}

\paragraph{}

A commonly experienced consequence of the order of recruitment, that is clear in Figure \ref{fig:n_motor_unit_recruitment}, is that regardless of the muscle innervation ratio, there is less precision in tension regulation, and therefore less coordination and fine motor control when a muscle is being near maximal or maximally recruited. Muscles with lower IRs (such as the eye muscle described earlier with an IR of 5), will retain precision to a much higher level of tension, but it will still be reduced at higher tensions. If you have ever tried to fix your apple laptop and remove the small screws that require precision but a lot of force to remove, you understand the challenge of high tension precision.

\section{Regulatory Feedback}

It is estimated that between 40\% to 70\% of the nerves in a nerve bundle to a skeletal muscle fiber are $\alpha$-motor neurons (\underline{e}fferent nerves, \underline{e}xiting the spinal cord). The remaining axons are from sensor nerves (\underline{a}fferent nerves, \underline{a}ccessing the spinal cord). Regulation of tension is critically dependent on knowing whether the intention of tension is being met (whatever that intention may be) so that tension can be adjusted as necessary. Since neither motor control or neuroscience is a focus of this book (or course) this section is brief and focused on the muscle spindles and golgi tendon organs (GTOs) and how they influence tension regulation in the the spinal cord. 

\subsection{Muscle Spindles}

Muscle spindles are receptors that are excited when stretched and provide the central nervous system with information about muscle length and rate of stretching. With this information the muscle spindles are important contributors to proprioception (sensation and perception of position and movement). They are located within fascicles along side (parallel to) muscle fibers. It is conventional, once speaking of the muscle spindle, to refer to muscle fibers as extrafusal fibers (not part of the muscle spindle). The muscle spindles have both afferent and efferent innervation and their own small set of intrafusal contractile fibers. The intrafusal fibers are innervated by $\gamma$-motor neurons and allow the muscle spindle to adjust to the length of the muscle to remain sensitive to changes in length (See Figure \ref{fig:MuscleSpindle}). 

\begin{figure}[!ht]
    \centering
    \includegraphics[width=1\linewidth]{./figure/MuscleSpindle.png}
    \caption{Muscle Spindle \footnotesize{\href{https://commons.wikimedia.org/wiki/File:MuscleSpindle.svg}{Wikimedia Commons, CC BY 4.0})}}
    \label{fig:MuscleSpindle}
\end{figure}

Afferent signals from the muscle spindles terminate at the spinal cord as well as higher up in the central nervous system for long loop reflexes and coordination (cerebellum) and the cerebral cortex (proprioception). The response of muscle spindles to a rapid increase in length (quick stretch) signals the dendrites of the $\alpha$-motor neurons of the agonist (same muscle that is being stretched) with excitatory post synaptic potentiations (EPSPs); and the dendrites of the $\alpha$-motor neurons of the antagonist (opposite muscles of those that are being stretched) with inhibitory post synaptic potentiations (IPSPs). Collectively these actions are referred to as the stretch reflex. If the stretch is quick enough and excites enough muscle spindles the EPSPs are sufficient to excite $\alpha$-motor neurons which provoke muscle excitation and activation. Testing stretch reflexes is an important component of a physical exam when the integrity of the $\alpha$-motor neuron, or any part of the sensory - motor reflex loop, is in question.

\subsection{Golgi Tendon Organs}

Golgi tendon organs (GTOs) are on the mostly tendon side of the transition space known as the musculotendinous junction. They are so close to muscle fibers that each GTO is connected in series with 5-20 muscle fibers. In this position the GTOs detect tension in muscle and tendinous fibers \cite{macefield_physiological_2005}. The GTO is excited by tension within the muscle fibers or tendon. Afferent impulses from the GTOs terminate in the spinal cord and cerebellum. GTO afferents in the spinal cord result in IPSPs of the agonist muscle. Inhibiting the agonist limits active tension as a protective mechanism.

\section{\textit{Clinical Physiology Connections}}

\subsection{Electromyogram (EMG)}

Each time a muscle fiber is activated it is caused by an excitation (action potential) from the nerve to the motor end plate, and along the sarcolemma of all the fibers that are innervated by that $\alpha$-motor neuron. Therefore, each muscle fiber activation is associated with an electrical signal. An EMG electrode near the fiber can record this electrical activity. The electrode can either be a small needle that acts as an antenna to record local electrical activity or a surface electrode (sEMG). If using a needle, the raw EMG signal can be used to accurately determine when a particular muscle is active (regardless of depth or size of the muscle). sEMG can be used to determine when a particular muscle is active when the muscle of interest is superficial and large enough to avoid cross talk at the sEMG electrode, but cannot be accurately used to determine specific activation of deep or small muscles. For example, sEMG just lateral to the spine cannot be expected to accurately identify activation of deep muscles such as the multifidi.

\paragraph{EMG quantification} It is difficult to quantify the activity of a muscle from a raw EMG signal. The EMG contains electrical signals from every muscle fiber in the vicinity of the EMG electrode which creates a very noisy signal (See Figure \ref{fig:EMG_ARV}). From the raw signal the Average Rectified Voltage (ARV) EMG can be computed. ARV is a time-windowed mean of the absolute value of the signal (turns all voltages into positive values). ARV is useful for further analysis such as recording the excitation patterns of a muscles that may include changes in excitation or as the first step towards more advanced analyses \cite{merletti_surface_2016}. The relationship between EMG activity and muscle tension is most valuable during isometric activation because the association between activation and tension changes with changes in length (length-tension relationship), and with different velocities (force-velocity relationship).

\begin{figure}[!ht]
    \centering
    \includegraphics[width=1\linewidth]{./figure/EMG_ARV.png}
    \caption{Raw (upper) and ARV (lower) EMG Signals \footnotesize{Data collected using Biopaq acquisition hardware and Acknowledge software in the author's lab (SC)}}
    \label{fig:EMG_ARV}
\end{figure}

\paragraph{Biofeedback}

sEMG is a valuable form of biofeedback to indicate that a certain muscle has been activated. Commercially available devices provide visual or auditory signals so a patient can learn to activate a muscle. Common applications of the EMG for biofeedback is activation in the vastus medialis after anterior cruciate ligament surgery, or the lower trapezius during shoulder flexion or abduction.

\paragraph{Motor Unit Number and Size Indices (MUNIX \& MUSIX)}

The motor unit number index (MUNIX) and the motor unit size index (MUSIX) are based on sEMG data interpreted through a mathematical model to estimate an index of the number and size of functional $\alpha$-motor neurons for a muscle \cite{nandedkar_motor_2004}. MUNIX is not an absolute count of motor units, and MUSIX is not an absolute size of the motor units. But they do provide an estimate that correlates with other measures and they have reasonable reliability. Considering the test only takes 3-5 minutes per muscle, is non-invasive and relatively comfortable (as compared to other approaches), they have gained popularity in clinical neurology and neurophysiology. MUNIX is considered an effective bio-marker for assessing disease progression and an alternative clinical trial outcome for conditions associated with a decrease in motor units such as amyotrophic lateral sclerosis (ALS). It has also been proposed for use in the evaluation of peripheral neuropathies, particularly in demyelinating neuropathies with conduction block. While there have been less studies, MUNIX is starting to be evaluated in a variety of central nervous system disorders (such as spinal cord injury, stroke, and cerebral palsy \cite{fatehi_utility_2018}. 

A study of healthy individuals between 20 and 80 years of age demonstrated a reduction in MUNIX and MUSIX values with aging, particularly over the age of 60, indicating that they may be valuable biomarkers for research on the prevention and treatment of sarcopenia \cite{cao_reference_2020}. Sarcopenia is an age-related progressive loss of muscle mass and strength \cite{dao_sarcopenia_2020}. It is a type of muscle atrophy primarily caused by the aging process and certainly confounded by physical inactivity and malnutrition. Sarcopenia tends to result in a larger loss of FF motor units and FG (Type 2a) muscle fibers. If a loss of motor units is inevitable with aging, then healthy aging would strive for lifestyles that promote increasing MUSIX despite an inevitable decline in MUNIX. A consequence of this shift (reduced MUNIX and increased MUSIX) would be a reduction in muscle tension regulation. Therefore, the choice of lifestyle would need to consider the motor control (balance, coordination) capabilities in such individuals.

\subsection{Nerve Conduction Velocity (NCV)}

Nerve conduction velocity (NCV) refers to the time it takes for an excitation to travel a certain distance. It can be measured clinical with electrodiagnostic testing that includes EMG. A stimulus is presented proximally at a peripheral nerve and the distal muscle excitation (via EMG) or axonal excitation is measured.

The tight regulation of active tension for coordinated movement requires fidelity in the signals coming from the central nervous system.\footnotemark\footnotetext{Fidelity in this context means that intended signals are accurately sent where and when they are requested.} If the myelin sheath of an $\alpha$-motor neuron is damaged the NCV is reduced. If the NCV is reduced then the ability to activate muscles with high fidelity signals is impaired resulting in delays in the generation of active tension, limiting the ability to increase active tension when needed. These changes occur for two reasons, first, limitations in frequency summation, and second, limitations in motor unit summation. This is precisely what happens in peripheral demyelinating diseases such as Guillain-Barre syndrome, Chronic Inflammatory Demyelinating Polyradiculoneuropathy, and Anti-Myelin Associated Glycoprotein Neuropathy. The demyelinating disease known as Multiple Sclerosis (MS) also impairs signal fidelity, however, it is a central nervous system condition. Therefore, in MS the loss of signal fidelity does not directly alter the peripheral NCV, it impairs central NCV which interrupts signals that excite the motor unit dendrites in the spinal cord which contributes to a more complicated, nuanced and variable clinical presentation in terms of the impact it can have on motor function such as coordination.


\section{Summary \& Next Step}

Motor units connect the central nervous system to muscle fibers. Varying the frequency of excitation and the number of motor units excited allows the CNS to vary the active tension provided by a muscle. Tight regulation of this active tension is required for motor control. There are three accepted classifications of motor units and they correspond to three accepted accepted classifications of muscle fibers. The structural characteristics of these motor units and muscle fibers are related to distinct functional characteristics in twitch and tetany. Recruitment of motor units is the primary mechanism for varying tension with variations in the frequency of excitation providing a small range of tension variation that allows for fine tuning the nearly continuous range of possible tensions. Muscle spindles and golgi tendon organs provide the CNS with feedback about muscle tension, stretch and length. EMG can capture valuable information about motor unit and muscle excitation as well as the determination of NCV. The next chapter expands on the metabolic characteristics of muscle fibers and the impact muscle energetics has on function as well as the demands that muscle energetics places on other body systems.

\printbibliography[heading=subbibintoc]
% !TEX root = ../notes_template.tex
\chapter{Muscle Energetics}\label{chp:energetics}
Updated on \today
\minitoc

Muscles transform chemical energy into mechanical energy. The chemical energy utilized in this transformation is ATP. However ATP is not ingested and then circulated through the body to the cells. It is present in the cells in small quantities and regenerated in the cells at a rate that approximates its rate of utilization. The primary utilization ATP in the muscle cell for active tension occurs when it is hydrolyzed by myosin ATPase during crossbridge cycling. The change in shape of the myosin head is potential mechanical energy that becomes kinetic mechanical energy during crossbridge activation. Muscle excitation and activation requires ATP in two other key steps. The $Na^+ / K^+$-ATPase pumps to establish the resting membrane potential, and the $Ca^{2+}$-ATPase pumps to return $Ca^{2+}$ into the sarcoplasmic reticulum. Beyond these clear ATP needs for excitation, activation and tension the muscle fibers continually require ATP, even while at rest, for membrane transport and for synthesis of chemical compounds throughout the fiber that are critical for integrity (maintenance and adaptation). 
The chemical energy to regenerate ATP comes from food we ingest, digest, absorb and metabolize into substrates. Not all substrates require metabolism, but some sort of biochemical metabolic pathway is involved when anything other than glucose is used as the substrate (this includes the use of glycogen, a stored form of glucose).
This chapter considers the rate and capacity of five pathways to regenerate ATP in muscle cells. These five pathways vary across a spectrum with rate of regeneration inversely related to capacity. They are all always being utilized to some extent during muscle cell energetic transformations. Structural variations that exist on a spectrum between muscle fibers result in functional variation in their ability to carry out these five pathways with a high degree of adaptability. Clinical physiology connections emerge based on an understanding of energetics, this chapter will focus on the implications of sarcopenia, the concept of fatigue, and the pathophysiological concepts of hypoxia, hypoxemia and ischemia.

\vspace{5mm}

\textbf{Objectives include:}
\begin{enumerate}
    \item Explain the factors underlying the storage, use and regeneration of ATP.
    \item Explain the structure and function of energetic components of muscle fibers.
    \item Compare and contrast the energetic system pathways that transform macro nutrient substrates to ATP.
    \item Compare and contrast muscle fiber differentiation (types) based on energetic structures and resultant functional capacities.
    \item  Apply the concepts of mass balance, flow gradients, and energy to the analysis of patient/client problems related to the muscular system.
     \item Evaluate the different energetic pathways used in different activities, and analyze the response to different activities to energetic pathways.
     \item Explain the implications of sarcopenia on muscle strength and endurance.
     \item Explain the concept of, and the energetic basis of fatigue.
     \item Explain the energetic basis of hypoxia, hypoxemia and ischemia.
\end{enumerate}

\section{Energetic Transformation Overview}

Energetic Transformation is a slightly more explanatory way of describing what is commonly referred to as metabolism. Of course, there are many different types of energetic transformations occurring when considering all metabolism throughout the body. And not all metabolic reactions are specifically about the regeneration of ATP. Here we'll be focusing on metabolic processes and pathways in muscle that are utilized to regenerate ATP. However, most of these processes and pathways occur in cells throughout the body (actually, throughout the animal kingdom). 

Muscle fibers continuously require ATP. Despite how important ATP is to muscle fiber function, even at rest, it is maintained at relatively low concentrations (5 $mmol \cdot L^{-1}$) \cite{feher_quantitative_2017, jones_skeletal_2006}. For cells with a steady and relatively low energetic need the tendency for cells to store a small amount of ATP is not a problem. Even for muscles at resting metabolism it is not a problem. Unlike most cells the muscle fibers can increase their metabolic rate (rate of ATP regeneration through energetic transformation) 60-100 fold during transitions from rest to high levels of tension. Resting muscle has approximately 1 $mmol \cdot L^{-1}$ of ADP and 10 $mmol \cdot L^{-1}$ of $P_i$. One hypothesis of energetic transformation regulation is the ATP:ADP ratio, which is maintained at approximately 5:1.

There are two proposed reasons that ATP is not stored in large quantities. First, it is a heavy molecule. It is estimated that an individual regenerates their body weight worth of ATP. ATP is 507.18 g/mole. Whereas glucose is 180 g/mole and even fatty acids with up to 29 carbon chains and double bonds (high energy) have molar masses lower than ATP (<500 g / mole). Since mass (and weight) influence the energetic costs of movement then storing ATP is a much less efficient approach to energy storage than storing glucose (such as glycogen) and fatty acids (as adipose tissue). Second, ATP is a highly reactive molecule that can trigger several cellular events by binding and hydrolyzing. One source refers to the situation of too much ATP being stored as causing "chaos" to cellular functions \cite{jones_skeletal_2006}. The implications of not storing ATP includes the continuous need to regenerate ATP at the rate that it is being utilized with a small amount of reserve or surge capacity.

% No switch - all of the biochemical pathways are occuring at the same time, the flux through each is dependent on the need for ATP and that need is based on the need for energy. High metabolism refers to the situation of needing more ATP. The entire system of using $O_2$ consumption as a biometric of metabolism is predicated on the ultimate use of $O_2$ to energize all ATP. Even when PCr is used to energize a muscle activation the creatine (Cr) is ultimately regenerated (PCr) by an ATP 

\section{ATP Regeneration Pathways}

Overall there are six ATP regeneration pathways in the muscle fiber. The myokinase pathway is briefly described as part of the immediate ATP/Phosphocreatine (PCr) pathway. The pathways are summarized in Figure \ref{fig:Energetics_Overview}. The rates and capacities of the pathways are later directly compared in Table \ref{table:ATP_Rates}. 

Some of the pathways occur outside of the mitochondria (PCr, Myokinase, Muscle Glycogen $\rightarrow$ Lactate Pathway) and are therefore not rate limited by the structural requirements of mitochondria. Generally speaking, pathways occurring outside of the mitochondria do not involve cellular respiration and can be referred to as anaerobic (not requiring O2). Pathways occurring in the mitochondria involve cellular respiration and are referred to as aerobic (Muscle Glycogen $\rightarrow$ $CO_2$; Liver Glycogen $\rightarrow$ $CO_2$; Fatty Acid $\rightarrow$ $CO_2$ pathways). 

Pathways outside of the mitochondria follow standard chemical substrate phosphorylzation. The chemical reactions of these pathways are catalyzed (expedited) by enzymes and produce ATP at high rates but low overall yield. The low yield is due to the fact that they tend to be self limiting. They either recycle existing energy and quickly exhaust their capacity (myokinase and PCr) or they create byproducts that inherently limit their process (the glycolytic process within the Muscle Glycogen $\rightarrow$ Lactate pathway). Either way they produce ATP at a higher rate than their support systems allow. They meet a particular need, provide a high rate of ATP regeneration, but fatigue quickly and must recover. Such work:recovery cycles are a recurring theme.

% Stopped here

\subsection{ATP / Phosphocreatine (PCr) (immediate) Pathway}

% Include the concept of diffusion of ATP through the muscle cell and the hypothesized use of PCr to facilitate the transport of ATP through the sarcoplasm

Creatine is synthesized in the liver, or ingested, digested and then absorbed from meat or ingested and absorbed from supplementation.\footnotemark\footnotetext{Creatine is a molecule that can be absorbed directly through the gut into the blood. Creatine supplementation will be a clinical connection in Chapter \ref{chp:blood_nutrients} on Visceral Support.}

$PCr \rightleftharpoons ATP$

Back and forth, high rate reaction catalyzed by the enzyme creatine kinase (CK)

At rest about 80\% of creatine is in the energized (phosphoralated) form of PCr, and there is approximately five times as much PCr than ATP (5:1 ratio).

The $PCr \rightleftharpoons ATP$ is high rate it can generate upwards of 4.4 moles/minute of ATP. However, it is only capable of regenerating approximately 0.7 moles of ATP overall. Therefore, when working at maximum capability (rate of 4.4 moles/minute) it would only be sustained for 9.5 seconds (See the Summary of Max Rates of ATP Regeneration by Pathway in Table \ref{table:ATP_Rates}. This time estimate is dependent on the assumption of the max rate (4.4 moles/minute). If a lower rate of ATP was being utilized, say .5 moles/minute, then PCr would be able to sustain ATP regeneration much longer. And if that rate of ATP utilization was lower than other pathways then PCr would continue, itself, to be regenerated and the need for ATP from PCr would not approach maximum because other pathways would be contributing to the ATP need.

% Stable form of energy - not reactive (volatile) or heavy has ATP - so PCr is used even at rest

The PCr pathway is also hypothesized to be utilized as a ATP transport mechanism throughout the sarcoplasm (PCr shuttle). There are substantial diffusion barriers in a muscle fiber due to the number of highly structured protein complexes. The mitochrondria are near the sarcolemma (along with the nuclei largely because the rest of the fiber is packed tightly with myofilaments of sarcomeres) and thus the diffusion of ATP from the mitochrondria encounters many barriers.  PCr is diffused through the entire sarcoplasm and due to its rapid regeneration rate can easily shuttle (or pass) the energy of ATP through the sarcoplasm from just outside a mitochondria to near the myosin ATPase. During muscle activation when the rate of ATP use by the myosin ATPase increases the nearly PCr regenerate and then that regeneration can propagate from all around the sarcoplasm to move ATP toward the myosin ATPase (See Figure \ref{fig:PCr}). This movement - either the diffusion of ATP directly or the use of a PCr shuttle would be, of course, based on the flow gradient. The more ATP being utilized by the myosin ATPase, the $Na^+/K^+$-ATPase or the $Ca^{2+}$ ATPase the lower the concentration of ATP near those proteins and the greater the diffusion, and shuttling, of ATP towards those locations in the sarcoplasm.

\begin{figure}[h!]
    \centering
    \includegraphics{}
    \caption{Movement of ATP through the sarcoplasm using a PCr shuttle}
    \label{fig:PCr}
\end{figure}

\paragraph{Myokinase (Adenylate Cyclase)}

Another "immediate" high rate reaction to regenerate ATP is the myokinase (an enzyme, also referred to as adenylate cyclase) that catalyzes the reaction: $2ADP \rightleftharpoons AMP + ATP$. During this reaction a $P_i$ is hydrolyzed from ADP and the energy is utilized to regenerate ATP. While this reaction can happen at a high rate, it is limited and probably only utilized in extreme "ATP need" situations. In such situations this reaction allows continued cell functions for a short period of time. Estimates are that under conditions of maximum ATP utilization (upwards of 4.4 moles/minute), this system could be sustained for milliseconds. For our purposes it therefore simply added to the ATP/PCr row of Table \ref{table:ATP_Rates} given its relatively insignificant contribution.

\subsection{Glycolosis $\rightarrow$ Lactate Pathway}

\subsection{Muscle Glycogen $\rightarrow$ $CO_2$ Pathway}

If $CO_2$ is created it is cellular respiration as part of the citric acid cycle (TCA). $O_2$ is consumed as part of the electron transport chain (ETC). Note that $CO_2$ is produced and $O_2$ is consumed at different steps in the pathways that produce $CO_2$ a byproduct and consume $O_2$.

$CO_2$ produced as part of TCA (also referred to as Kreb's cycle)
$O_2$ consumed as part of ETC (also referred to as oxidative phosphorylation)
TCA and ETC combined can be referred to as oxidative metabolism. 

Even though the $CO_2$ production and the $O_2$ consumption does not occur at the same steps in the pathways, whenever you see that $CO_2$ is produced, you can assume that $O_2$ is being consumed. 

\subsection{Liver Glycogen $\rightarrow$ $CO_2$ Pathway}

\subsection{Fatty Acids $\rightarrow$ $CO_2$ Pathway}



Mitochondria Extract Energy From Nutrients (p. 24) The principal substances from which the cells extract energy are oxygen and one or more of the foodstuffs—carbohydrates, fats, and proteins—that react with oxygen. In humans, almost all carbohydrates are converted to glucose by the digestive tract and liver before they reach the cell; similarly, proteins are converted to amino acids, and fats are converted to fatty acids. Inside the cell, these substances react chemically with oxygen under the influence of enzymes that control the rates of reaction and channel the released energy in the proper direction.

Mitochondria Extract Energy From Nutrients (p. 24) The principal substances from which the cells extract energy are oxygen and one or more of the foodstuffs—carbohydrates, fats, and proteins—that react with oxygen. In humans, almost all carbohydrates are converted to glucose by the digestive tract and liver before they reach the cell; similarly, proteins are converted to amino acids, and fats are converted to fatty acids. Inside the cell, these substances react chemically with oxygen under the influence of enzymes that control the rates of reaction and channel the released energy in the proper direction.
Oxidative Reactions Occur Inside the Mitochondria, and Energy Released Is Used to Form ATP ATP is a nucleotide composed of the nitrogenous base adenine, the pentose sugar ribose, and three phosphate radicals. The last two phosphate radicals are connected with the remainder of the molecule by high-energy phosphate bonds, each of which contains about 12,000 calories of energy per mole of ATP under the usual conditions of the body. The high-energy phosphate bonds are labile so they can be split instantly whenever energy is required to promote other cellular reactions. When ATP releases its energy, a phosphoric acid radical is split away, and adenosine diphosphate (ADP) is formed. Energy derived from cell nutrients causes ADP and phosphoric acid to recombine to form new ATP, with the entire process continuing over and over again. Most of the ATP Produced in the Cell Is Formed in Mitochondria After entry into the cells, glucose is subjected to enzymes in the cytoplasm that convert it to pyruvic acid, a process called glycolysis. Less than 5\% of ATP formed in the cell occurs via glycolysis. Pyruvic acid derived from carbohydrates, fatty acids derived from lipids, and amino acids derived from proteins are all eventually converted to the compound acetyl coenzyme A (acetyl-CoA) in the mitochondria matrix. This substance is then acted on by another series of enzymes in a sequence of chemical reactions called the citric acid cycle, or Krebs cycle. In the citric acid cycle, acetyl-CoA is split into hydrogen ions and carbon dioxide. Hydrogen ions are highly reactive and eventually combine with oxygen that has diffused into the mitochondria. This reaction releases a tremendous amount of energy, which is used to convert large amounts of ADP to ATP. This requires large numbers of protein enzymes that are integral parts of the mitochondria. The initial event in ATP formation is removal of an electron from the hydrogen atom, thereby converting it to a hydrogen ion. The terminal event is movement of the hydrogen ion through large globular proteins called ATP synthetase, which protrude through the membranes of the mitochondrial membranous shelves, which themselves protrude into the mitochondrial matrix. ATP synthetase is an enzyme that uses the energy from movement of the hydrogen ions to convert ADP to ATP, and hydrogen ions combine with oxygen to form water. The newly formed ATP is transported out of the mitochondria to all parts of the cell cytoplasm and nucleoplasm, where it is used to energize the functions of the cell. This overall process is called the chemiosmotic mechanism of ATP formation.



\subsubsection{Pathways for Protein Energetics}



\subsection{Rates and Capacities}

\begin{table}[h!]
\centering
\begin{tabular}{||c c c c||} 
 \hline
Source & Max Rate of ATP (mol/min) & Amount of ATP (mol) & Time at Max (s or min)\\ [0.5ex] 
 \hline\hline
 ATP/PCr/Myokinase & 4.4  & 0.7 & 9.5 s \\
 Muscle Glycogen $\rightarrow$ Lactate &  2.4 & 1.6 & 40 s \\ 
 Muscle Glycogen $\rightarrow$ $CO_2$ & 1.0 & 84 & 84 min\\
 Liver Glycogen $\rightarrow$ $CO_2$  & 0.4 & 19 & 48 min \\ 
 Fatty Acids $\rightarrow$ $CO_2$ & 0.4 & 4000 & 10,000 min \\[1ex] 
 \hline
\end{tabular}
\caption{Summary of Max Rates of ATP Regeneration by Pathway (\footnotesize{Data from \cite{feher_quantitative_2017}})}
\label{table:ATP_Rates}
\end{table}

When considering the time estimates based on the quantity of ATP can be be regenerated by various energetic pathways it is important to consider that two factors influence the sustainability of the pathway. First whether the pathway, at its max rate, is rate limiting. Second, the availability of the substrate (macro-nutrients) in storage. Another consideration for the time estimates in Table \ref{table:ATP_Rates} is that the times are calculated based on the assumption of the max rate. For example, the 10 second estimate for ATP/PCr (which is a commonly cited estimate for this system) is entirely predicated on the max rate of ATP regeneration from ATP/PCr, which is based on the demand for ATP by the muscle. If the demand for ATP by the muscle is 2.2 mol/min then the time that ATP/PCr could contribute would be approximately 19 seconds. If the demand for ATP by the muscle is 1.1 mol/min then the time that ATP/PCr could contribute would be approximately 38 seconds. By this time the Muscle Glycogen $\rightarrow$ $CO_2$ pathway could be contributing and meeting most of the ATP demand of the muscle. 

An important point is that there is no such thing as ATP dept. Energy is needed for cellular functions. Either the pathways provide a way to regenerate the ATP needed for the demand, or the cellular functions do not occur.


\begin{table}[h!]
\centering
\begin{tabular}{||c c c c||} 
 \hline
Events & Rate of ATP Consumption (mol/min) & Amount of ATP (mol) & Estimated Time (s or min)\\ [0.5ex] 
 \hline\hline
 Rest & 0.12  & 173 & One Day \\
 100 Meter Sprint & 2.8 & 0.5 & 10 seconds \\ 
 800 Meter Run & 2.0 & 3.4 & 192 seconds\\
 1500 Meter Run & 1.7 & 6 & 210 seconds \\ 
 42,0000 Meter Run & 1.0 & 150 & 150 minutes \\[1ex] 
 \hline
\end{tabular}
\caption{Estimated Rate of ATP Consumption during Various Timed Running Events (\footnotesize{Data from \cite{feher_quantitative_2017}. Resting data is estimated based on basal energy expenditure of 1200 calories, and a conversion rate of about 6.93 calories per mole of ATP.})}
\label{table:Event_ATP_Rates}
\end{table}


\section{Motor Unit \& Muscle Fiber Types}

\begin{table}[h!]
\centering
\begin{tabular}{||c c c c c||} 
 \hline
 Muscle Fiber Type & Mitochondria & Myoglobin & Glycolytic Enzymes & PCr \\ [0.5ex] 
 \hline\hline
 Fast Glycolytic (FG)  & + & + & +++ & +++ \\ 
 Fast Oxidative Glycolytic (FOG) & ++ & ++ & ++ & ++ \\
 Slow Oxidative (SO) &  +++ & +++ & + & + \\ [1ex] 
 \hline
\end{tabular}
\caption{Energetic Characteristics of Muscle Fiber Types}
\label{table:Muscle_Fiber_Energetics}
\end{table}

\subsection{Slow Oxidative (S/SO/Type 1}

\subsection{Fast Oxidative Glycolytic (FR/FOG/Type 2x}

\subsection{Fast Glycolytic (FF/FG/Type 2a}




\section{\textit{Clinical Physiology Connections}}

\subsection{Sarcopenia}

Sarcopenia is an age related loss of muscle fibers with a bias toward FG fibers (FF motor units). A consequence of sarcopenia is a loss of the ability to attain the upper ranges of muscle tension. Expected consequences are a reduction in peak tension, high forces and high velocities of movement. An unexpected, paradoxical, consequence is a reduction in the ability to sustain lower ranges of muscle tension. This paradoxical consequence is based on the need to recruit the remaining motor units with increased frequency to meet the tension requirements of movement. Recruiting an S motor unit and its SO fibers more frequently (frequency summation) requires a higher rate of ATP than can be sustained. There is no question that S motor units and SO muscle fibers can achieve a higher rate of ATP than FF motor units and FG muscle fibers. But at a high level of tension (relative to the motor units capacity) the higher rate of crossbridge activation requires a higher rate of ATP regeneration. If that higher rate of ATP regeneration is greater than the rate capacity of the Glycolosis $\rightarrow$ $CO_2$ pathway, then the fiber will need to utilize the Glycolosis $\rightarrow$ Lactate pathway, which is self limiting (cannot be sustained). 

Let's walk through it with a simple model example. Assume walking requires 50\% output (tension) of 1/4 of the motor units. At this output the fibers require 0.7 moles of ATP / minute. In this scenario the Glycolosis $\rightarrow$ $CO_2$ pathway is sufficient and the activity is sustainable. Now assume there has been a loss of motor units. Now there are 1/2 of the prior motor units, and they are predominantly lower tension generating (S/SO) because the loss (due to sarcopenia) results in a loss of FG/FOG muscle fibers. Now walking (at the same pace as before) requires 75\% output from 3/4 of the fibers (since a twitch generates less tension there needs to be more frequency and motor unit summation). At this output the fibers require 1.3 moles of ATP / minute. In this scenario the Glycolosis $\rightarrow$ $CO_2$ pathway is not sufficient and the Glycolosis $\rightarrow$ Lactate pathway must be utilized to supplement ATP regeneration, which is not sustainable. Continuing to walk at this pace will result in fatigue. That is the paradoxical consequence of the loss of FF motor units in sarcopenia. Everyone expects a loss in the ability to attain tension (as measured by peak force). But there is also a loss in the ability to sustain tension (as measured by the ability to continue developing tension for an activity).


\subsection{Fatigue}

% Robustness (surge capability) - sensing the reserve
% Fidelity - objective (outward) 
% Efficacy - subjective (inward - how am I doing this act, how is it being supported?)
% Integrity - you need the reserve, so don't completely exhaust it / use it
% Fatigue - roll up, signal, regarding sustaining (and then can't attain what you started with)

Biological systems include cycles. Ca+ is released and taken in, muscle fibers develop active tension and then they don't, movements happen then they don't, we are awake and then asleep, etc. Across all scales in biological systems there are cycles. These cycles include spending resources, capabilities, and reserves and then building them back up. The concept of fatigue is fundamentally about these cycles. At the most general level fatigue occurs when resources, capabilities and reserves have been utilized and it is time for the cycle to be completed and go to its next phase (work to rest for example). If the cycle leading to fatigue is not completed, then fatigue occurs.

Fatigue is a pervasive concept because it can be applied to any aspect of a system with cycles occurring at any scale, which is any system doing an act, which is any system. Reading about fatigue is sure to result in several different interpretations. Muscle fatigue is the inability continue to sustain a tension with repeated excitation, and with fatigue the inability to attain a previously attainable tension. Running fatigue is the inability to sustain a pace with continued running, and the inability to attain that pace until recovery has occurred. Chronic fatigue syndrome is characterized by a severe lack of energy following exertions. The word fatigue can be like the description of the elephant by blind monks in Figure \ref{fig:elephant}. One monk describes the elephant as like a rope (while feeling the tail), another like a snake (while feeling the trunk), another like a wall (while feeling the body), another like a tree trunk (while feeling a leg), another like a spear (while feeling the tusk), and another like a large sheet (while feeling the ear).    

\begin{figure}[!h]
    \centering
    \includegraphics[width=1\linewidth]{./figure/elephant.jpg}
    \caption{Five Blind Monks Describing an Elephant (\footnotesize{Public Domain from \href{https://commons.wikimedia.org/wiki/File:Blind_men_and_elephant.png}{Sophie Woods, World Stories for Children, Ainsworth & Co. (Chicago), p. 14}})}
    \label{fig:elephant}
\end{figure}

of a system at any scale with one meaning, two perspectives and many applications. The one meaning of fatigue is an inability of a system\footnotemark\footnotetext{In our case a biological system.} to accomplish its act. It represents a failure to achieve act objectives. Note that this meaning is objective. The fact that the act cannot achieve its objective is measurable in some way. So the meaning of fatigue and the objective perspective are coherent. 

The subjective perspective, at least in terms of human fatigue, is the feeling (perception) of fatigue. Sometimes people confuse the subject and the objective. Notice again that the objective perspective of fatigue includes the inability of the system to achieve its act based on some measurable objective. If the system is meeting the objective of its act then it is not fatigued. However, there may be a perception of fatigue, and the person can report that they are fatigued. Both need to be considered and both pieces of information are important. For example, if a person walking 4 mph on a treadmill starts to feel fatigued during this act but they can accomplish the act then they are not objectively fatigued (they are doing the act), but they are subjectively feeling fatigued (or perhaps tired; or out of breath which can be confused symptomatically with fatigue). The new symptom of (perception of) fatigue in this individual despite being able to meet the objective of the act means that the person is not fatigued for the act of walking at the point they are doing the walking. But that perception is probably arising because some supporting system for that act of walking is fatigued and perhaps in an objective measurable (but not currently measured) act. That failure results in the need for other support systems to provide more support than usual to the walking act. Whether the fatigue in these support system can compensate or not, whether they are sustainable or not, influences whether and if so how long the walking act at 4 mph can continue. Once all possible support systems are fatigued then the walking act will no longer be able to continue at 4 mph, so the person slows down the treadmill to 3 mph and they are doing a new act that they can possibly continue. At this new walking act, the energetic requirements may be low enough that all of the support systems recover to full capability and the person can go back to 4 mph for a time being. The change in the ability - previously being able to do 4 mph without fatigue to now not being able to do 4 mph without fatigue - can be due to a large number of factors. Relevant to this chapter it may be due to the amount of substrate available, or state of the muscle fibers (perhaps there has been a loss of mitochondria due to reduced use of the muscles), or perhaps deficiency in creatine that is slowing down transport of ATP through the sarcoplasm so that the Muscle Glycogen $\rightarrow$ $CO_2$ pathway ATP needs to be supplemented with the Muscle Glycogen $\rightarrow$ Lactate pathway.

Based on the above example it is hopefully clear that the applications of the concept of fatigue vary based on time and space scales, and that these scales are superimposed. As we consider longer time scales all of the acts that can occur within that time scale are important. As we consider higher spatial scales, all of the acts at spatial scales below it are important. If the act is walking for 4 hours at a 4 mph pace then all of the acts within 4 hours, at all of the scales (walking requires tension in a particular set of muscles, each muscle must generate the right tension, each motor unit contributes the tension it must contribute, each muscle fiber contributes its share of the tension which requires the fiber to be utilizing and regenerating ATP). Walking today for 4 hours at 4 mph may be achieved (no fatigue for walking at 4 mph for 4 hours) with the perception of fatigue (subjective, so most likely some acts in supportive roles were fatigued). The next day the act of walking for 4 hours at 4 mph may not be achieved - there may be objective fatigue the next day. An act today with no fatigue, but performed everyday may result in fatigue in upcoming days. The point is that system must have time to recover from fatigue.

The perspective of objective and subjective fatigue does not intend to diminish the importance of subject fatigue experiences. Just to keep both perspectives in mind. These perspectives give rise to the perspectives of absolute and relative workload. Walking at 4 mph is an absolute workload. If that workload can be achieved without objective fatigue at the level of the required act for 4 hours then that absolute workload is 4 mph for 4 hours. The subjective fatigue is the relative workload, how hard did the person need to work (or push, or strain) to complete that absolute workload? Relative workload can be measured with a rating of perceived exertion (technically different but relatable to subjective fatigue), or by measures of other systems that support the absolute workload of walking such as heart rate, blood flow, ventilation rate, blood lactate, oxygen consumption, carbon dioxide production. 


\subsection{Ischemia, Hypoxemia \& Hypoxia}

The oxidative energetic pathways provide most of the ATP for cellular functions and are critically involved in the restoration of the energetic resting state following short term rate increases in ATP production with CP or glycolosis. These pathways require oxygen and macro-nutrient substrate (seconds-minutes). Over longer time periods (hours-days) they require cellular synthesis of enzymes which require ATP and amino-acids (to build proteins). And over longer time periods (days-weeks) they require maintenance of the health of mitochondria and replacing damaged or dead mitochondria. Interruptions in the availability of O2 interrupts the process of ATP production with these pathways and can lead to the accumulation of metabolic waste products that alter the pH of the cell and the extra-cellular fluid. These interruptions form the basis of a large number of relatively common and life-threatening chronic medical conditions such as heart disease, stroke, peripheral vascular disease, COVID-19, pulmonary disease, and hematologic (blood) conditions; as well as sudden onset (acute) conditions such as a heart attack and acute altitude sickness. 
Some of the conditions have a rapid onset and immediately threaten life; others exert their effect gradually. The variation is related to rate of onset of O2 deprivation, the magnitude of O2 deprivation (how much deficit, how many cells), and whether the impact is just in the availability of O2 or whether there is also an impairment in waste product removal. But they can all be analyzed based on an understanding of cellular energetics and the role that ATP plays in cellular fidelity, efficacy and integrity.

\paragraph{Hypoxia}
Hypoxia refers to the situation in which there is not enough O2 getting to cells for them to sustain the oxidative production of ATP. Hypoxia can be caused by a variety of situations. It is a local condition because it depends on local O2 levels and local O2 needs. Local O2 needs are dependent on local metabolism. Cells of the body that have relatively high and constant O2 needs, and are therefore more susceptible to hypoxia, are the heart and brain. While metabolism is always related to cellular activity, these two organs tend to have a higher resting metabolism than other body cells. For example, muscle metabolism can far exceed that of both heart and brain, that is only during periods of high muscle activity which require high levels of ATP production. 
The two primary causes of hypoxia are hypoxemia and ischemia.

\paragraph{Hypoxemia}
Hypoxemia is a specific situation in which the blood isn't carrying adequate oxygen to the body’s tissues. Common causes of hypoxemia include pulmonary and blood conditions (for example, obstructive pulmonary disease and anemia) or environmental conditions (altitude, carbon monoxide). 

Hypoxemia can cause hypoxia. The severity of hypoxia in a cell caused by hypoxemia is dependent on the severity of hypoxemia as well as the metabolic activity of the cell (O2 demand), which fluctuates based on several factors. In someone with mild hypoxemia there may be no hypoxia in cardiac or skeletal muscles. However, with exertion that increases cardiac and skeletal muscle activity and thus need for O2 (O2 demand) there may be hypoxia. In these situations cardiac and skeletal muscle function will be impaired by the lack of O2. The cellular adjustment will be to provide ATP using a higher rate of CP and glycolosis which is not sustainable. The by-products of these activities will decrease the cellular and extra-cellular pH which will further impair the production of ATP. The acidosis and reduced ATP relative to need can impact the ability of the cells to repolarize, reduce the frequency of excitations, reduce the pumping of Ca+ back into the sarcoplasmic reticulum, reduce the rate of myosin head release. The consequences of these changes include lower tension production, spasm and potentially damage to the cell membrane. But typically, if the problem originates with hypoxemia that is adequate for resting levels of O2 demand then simply ceasing the activity will restore balance and not result in damage to the cell membrane.
 
\paragraph{Ischemia}
Ischemia is a specific situation in which  blood supply to cells is reduced. The extent of cellular involvement depends on the extent of the reduction. For example, if an entire artery is impacted than an entire limb, or muscle can be involved. If the reduction occurs in capillaries then the reduction in blood flow is to far fewer cells. Common causes of ischemia are atherosclerosis, arteriosclerosis,\footnotemark\footnotetext{Arteriosclerosis refers to thick and stiff arteries that can restrict blood flow. Atherosclerosis is a type of arteriosclerosis that includes buildup of fats, cholesterol and other substances in and on artery walls (plaque). The subsequent reduction in vessel diameter can limit blood flow, and if a plaque becomes an embolus it can lodge and completely block blood flow (blood clot).} blood clots (arterial thrombosis or embolus, or in the case of pulmonary blood flow venous thrombosis or embolus), blood vessel spasm, and micro-circulatory inflammation (a clinical manifestation of hypoxia itself and seen in COVID-19). 

Ischemia can cause hypoxia. The severity of hypoxia in a cell caused by ischemia is dependent on the severity of ischemia as well as the metabolic activity of the cell (O2 demand). A sudden and complete blockage of blood flow to cells, with no alternative pathways to provide blood flow to the cell, is a serious situation that results in cellular death due to the inability to produce ATP for cell membrane functions (sudden complete hypoxia) and to remove waste products. The combination of these two situations results first in reversible damage to the cell membrane and then to irreverisble damage to the cell membrane. Without the cell membrane the cell has lost its integrity. Less extreme reductions in blood flow can create a wide variety of hypoxic and waste removal situations that allow sustained but reduced function for a cell (resulting in long term problems in cell maintenance), and reduced function of the cells. For example, a limit on how much activity the cardiac or skeletal muscle can perform prior to having hypoxia. It is common to simply refer to such situations as ischemia (which means local blood flow is reduced and O2 demands are high enough to cause hypoxia. 

Given the variable nature of blood flow supply and cellular O2 demand there are two situations that can arise for cardiac muscle cells in particular. Stable ischemia refers to the situation that blood flow is sufficient for resting O2 demand, but not sufficient during elevated O2 demand (increased cardiac muscle activity). The situation is considered stable because simply reducing cardiac activity will reduce cardiac O2 demand and restore balance to allow recovery. Unstable ischemia refers to the situation that blood flow is not sufficient for resting O2 demand. Unstable ischemia is unstable because balance cannot be restored by reducing the cardiac muscle activity back to rest since it is the resting demand that cannot be met. In such situations blood flow must be restored, or cardiac muscle O2 demand must be reduced below resting levels. Restoring blood flow is highly situational and can involve breaking up a blood clot (thrombus) with medications (thrombolytics); or restoring the diameter of blood vessels with an angioplasty or by re routing the blood (by-pass graft. Reducing cardiac muscle O2 demand can be accomplished by lowering blood pressure (blood pressure is the resistance that the cardiac muscle must work). Nitroglycerine is a very powerful and fast dilator of blood vessels that quickly lowers blood pressure and allows the cardiac O2 demand to be lowered below resting values. The hope is this restores balance between O2 supply and O2 demand and the cells to recover before damage.

\paragraph{Summary}
Hypoxia is the basis of the homeostatic imbalances caused by many conditions and diseases. It is based on the cellular requirements for ATP, which are based on the mitochrondrial requirements for O2. Cells can produce ATP without O2, however they cannot sustain the production of ATP without O2. Understanding these mechanisms and that of hypoxia offers a wellspring of conceptual insights for many diseases and pathophysiological conditions that go well beyond the above discussion. Hypoxia caused by hypoxemia or ischemia continues to be topic for several of the upcoming chapters on Muscle Support.



\section{Summary \& Next Steps}




\printbibliography[heading=subbibintoc]


\part{Muscle Support}
\input{./chapter/microcirculation.tex}
\input{./chapter/circulation.tex}
\input{./chapter/filtration_reabsorption_secretion.tex}
\input{./chapter/respiration}
\input{./chapter/ventilation}
\input{./chapter/digestion_absorption_metabolism}

\part{Muscle Integrity}
% !TEX root = ../notes_template.tex
\chapter{Defense}\label{chp:defense}
Updated on \today
\minitoc
This chapter introduces the immune system and the basics of immunological function

% Include basics of immunological function

\vspace{5mm}

\textbf{Objectives include:}
\begin{enumerate}
    \item
    \item
    \item
    \item
    \item
\end{enumerate}

\section{White Blood Cells}

\subsection{Immunity - White Blood Cells}
% Role of lymph 
% Role of white blood cells
% Role of platelets

\subsection{Coagulation - Platelets}

\printbibliography[heading=subbibintoc]
% !TEX root = ../notes_template.tex
\chapter{Myostasis}\label{chp:myostasis}
Updated on \today
\minitoc
This chapter introduces readers to the physical stress theory \cite{mueller_tissue_2002}. It extends this theory with new insights into the hypertrophy, isotrophy and atrophy signal transduction pathways. It also covers the molecular epigentic basis for muscle memory and the emerging understanding of the gut microbiome - muscle axis. Adaptation (responsiveness to training) and plasticity (ability to adapt) based on genetic factors, and in response to interventions that do not cause injury are considered.

% Include basics of immunological function

\vspace{5mm}

\textbf{Objectives include:}
\begin{enumerate}
    \item
    \item
    \item
    \item
    \item
\end{enumerate}

\subsection{Muscle Integrity}
Muscle Integrity focuses on muscle fibers being muscle fibers, capable of generating tension. Muscle fibers exist to execute the act of generating tension so integrity is about the parts and capabilities that make a muscle fiber a muscle fiber. Muscle Integrity considers both ECF in support of, and movement influencing, muscle fibers. ECF supports the long term act of creating muscle tension by providing the necessary resources for the muscle to maintain its integrity (it continues to be a muscle and therefore to be able to do the act of creating tension). Movement is also involved (the arrow from movement to muscle in Figure \ref{fig:muscle_centered_approach}) because movement, as you probably know, provides (or does not provide) a stimulus for the muscle fiber. Muscle cells create movement and movement creates stress and strain on muscle fibers that act as triggers for myostasis (muscle fibers continue to be, which we will call isotrophy) or to grow (hypertrophy) and when not needed to reduce their need for resources (atrophy). Isotrophy, hypertrophy and atrophy occur on a spectrum covered in Part III Muscle Integrity. Muscle Integrity includes the cellular activity occurring within muscle fibers, and is supported by the ECF and stimulated by movement.

Metrics of Muscle Integrity include the ability to continue to do what a muscle cell does (persevere) which occurs through maintaining the various parts, capabilities and interactions. A measurement of integrity could be the cross section area of a muscle, or the lean mass of muscle, as indicators of how well the muscle (as a muscle collection of muscle fibers) is being maintained.

\section{Muscle Development}

\section{Physical Stress Theory}

\section{Hypertrophy - Isotrophy - Atrophy Spectrum}

%How an activity influences muscle fatigue is based on the overload. The overload training concept is the stimulus for training related muscle adaptations. Training adaptations often occur in response to training fatigue in an attempt to build system capacity to avoid future fatigue. The relationship between fatigue and training adaptations is provided in Chapter \ref{chp:myostasis} on Myostasis.

\subsection{Signal Transduction Pathways}

\subsection{Transcription - Translation}

\subsection{Molecular Muscle Memory}

\subsection{Gut Microbiome - Muscle Axis}

\section{Adaptation \& Plasticity}

\subsection{Training Responses}

\subsection{Hypoxia}

\subsection{Blood Flow Restriction}

\printbibliography[heading=subbibintoc]
% !TEX root = ../notes_template.tex
\chapter{Repair}\label{chp:repair}

\minitoc
This chapter introduces readers to muscle repair processes. Cellular threats to muscle lead to either reversible or irreversible injury. The chapter then focuses on reversible injury and interventions that intentionally provoke reversible injury to prompt adaptation.

\vspace{5mm}

\textbf{Objectives include:}
\begin{enumerate}
    \item
    \item
    \item
    \item
    \item
\end{enumerate}

\section{Cellular Threats}

\section{Reversible Damage}

\section{Paths to Recovery}

\section{Molecular Responses to Intentional Threats (Interventions)}

\subsection{Dry Needling}

\printbibliography[heading=subbibintoc]

\part{Integrative Exercises}
% !TEX root = ../notes_template.tex
\chapter{Fick's Equation}\label{chp:fick_equation}
Updated on \today
\minitoc
This chapter introduces Fick's equation as a model to connect the muscle and supporting systems roles in attaining and sustaining activity.


\begin{equation}
    \dot{V}O_2 = \dot{Q} \cdot (a-v)O_2
\end{equation}

\vspace{5mm}

\textbf{Objectives include:}
\begin{enumerate}
    \item
    \item
    \item
    \item
    \item
\end{enumerate}

\section{Determinants of Cardiac Output}

\section{Determinants of Arterial Oxygen Content}

\section{Determinants of Venous Oxygen Content}

\section{Attain-Sustain Spectrum}

\printbibliography[heading=subbibintoc]
% % !TEX root = ../notes_template.tex
\chapter{Calcium Homeostasis}\label{chp:calcium_homeostasis}
Updated on \today
\minitoc
This chapter provides an integrative exercise on calcium homeostasis. Calcium homeostasis is critical for muscle function and requires integration of gastrointestinal absorption, endocrine, renal and skeletal function.

\vspace{5mm}

\textbf{Objectives include:}
\begin{enumerate}
    \item
    \item
    \item
    \item
    \item
\end{enumerate}

\section{Calcium Ingestion}

$Ca^{2+}$


\printbibliography[heading=subbibintoc]

% \begin{appendices}
% \input{./chapter/appendix_formula.tex}
% \end{appendices}

\backmatter

%%%%%%%%%%%%%%% Reference %%%%%%%%%%%%%%%

\printbibliography[heading=bibintoc]
\printindex


\end{document}

