% !TEX root = ../notes_template.tex
\chapter*{Introduction}
\addcontentsline{toc}{chapter}{Introduction}
\minitoc

This chapter introduces core concepts for knowing and applying clinical physiology in physical therapy practice. First we consider what we mean by clinical physiology. Second we explain the axioms of common sense and critical realism. We will not dwell on these axioms, they are introduced here and will come up throughout the book as necessary. At this point the most important aspect of understanding critical realism is that it opens the door to the effective use of models as an imperfect but useful approach to organizing and utilizing knowledge. Third, we make sure all readers have the same basic understanding of the core concepts of physiology. Finally, we introduce readers to Unifying Systems Theory as a method of approaching systems behind our "muscle centered approach" to clinical physiology. 

\vspace{5mm}

\textbf{Objectives include:}
\begin{enumerate}
    \item Understand the role of clinical physiology for the practice of physical therapy.
    \item Define the axioms of common sense and critical realism.
    \item Explain the value of a model.
    \item Provide an example of a model that is useful in physical therapy practice.
    \item Explain the core concepts of physiology.
    \item Define the key terms of Unifying Systems Theory
\end{enumerate}

\subsection{An important note on objectives}

Each chapter has objectives. They include specific behavioral terms related to the expectation for the objective. There is an ordered relationship between the behavioral terms that represents increasing and nested expectations.  To define is to simply understand the words being used, to know what they represent. Whether you can define something can be tested with a straightforward question such as identifying the correct definition from a set of definitions; or whether you can write the definition; or respond true or false to a proposed definition for a term, or identify the correct term from a set of terms when presented with a definition. To explain something goes beyond defining it, but it also implies that you can define the required terms. To explain you typically need to simultaneously hold a set of definitions together and understand how they relate to one another. So you can explain a model by demonstrating that you know all the definitions involved, and understand how they relate to one another, and can communicate that explanation. Whether you can explain something can be tested by asking about several aspects and asking you to infer which part is missing; or by asking about one part and asking you to infer the other part. The point is, to explain goes beyond defining and has an expectation of definition. But to define does not require you to explain. To evaluate something requires you to make judgements and to infer beyond the thing you are asked to evaluate. To evaluate implies that you can define and explain the thing you are asked to evaluate. Being tested for whether you can evaluate can include asking you to infer to or from something that has not been directly covered or discussed, going beyond the thing you are asked to evaluate to consider its implications in different situations. Providing an example is an objective based on a specific task and is considered a form of asking you to evaluate since it goes beyond a particular thing, or an abstract thing, to a totally new particular or abstract thing. For example, asking you to provide an example of of a model requires you to know the concept of a model well enough that you can use your background knowledge to come up with something you're already familiar with that fits what you know about models. That level of knowledge requires you to at least be able to evaluate a model because you have essentially evaluated your background knowledge and determined that it is a model. The chapter objective above ask you to provide an example of a model that is useful to physical therapy practice. This assumes you will be able to define, explain and evaluate tje concept of a model, and that you have background knowledge about physical therapy practice that you can use to evaluate models used in practice to consider what those models are and whether they are useful. 

\section{Clinical Physiology}




\section{Common Sense \& Critical Realism}

Critical realism is an approach to knowledge that accepts both what we can directly observe and what we cannot directly observe, but that can be inferred, as real. It is a philosophical approach to ontology (the way things are) and epistemology (how and what we can know about things). Critical realism is built on top of what is called common sense realism.

There are four axioms for common sense realism. First, the law of non contradiction. Something cannot be and not be in the same way at the same time. Second, the law of causation, any effect has a cause and many causes are themselves the effects of other causes. Third, the analogical use of language. The words we use are not the things they represent and therefore it is easy for words to have different meanings, and things to have many words used to represent them. Fourth, basic reliability of sensory perception. 

Critical realism adds to axioms to common sense realism. First, ontology determines epistemology. This means the way things are, determines how and what we can know those things. Second, reality is stratified, reality exists across multiple scales from the very small to the very large. Combining these axioms results in an understanding that the way things are (its scale for instance is part of the way things are) influences how and what we can know about those things. 

%add the McGrath quotation about holding things in our mind that we don't directly observe

%refer to Bhaskar and to my linda crane paper

As axioms we accept these from the start and with them we free ourselves to pursue knowledge about the world around us - in our case for this book that entails knowledge about muscles, what they do and how they do what they do (fidelity), how well they do what they do (efficacy) and what they need in order to continue to do what they do (integrity). 

\section{Models}

Models are abstractions of reality that we build in order to summarize, test and utilize our knowledge. Models come in many forms - graphic, causal, logical, mathematical, computational, even stories are examples of models. There are many models about muscles that we utilize frequently to summarize, test and utilize our knowledge about muscles. For example, the sliding filament theory is a model about how the muscles develop active tension that will be featured in Chapter 2. The sliding filament theory was first developed based on the appearance of muscle and was further developed with the observation that the length of muscle influenced how much tension it could actively generate (length - tension relationship). The inference goes something like this, if these filaments slide past each other to generate tension then how much overlap there is much influence how much tension can be generated. And if the length of the muscle influences how much overlap there is then the length of the muscle will influence how much tension can be generated. So that model of sliding filaments received experimental support when the length - tension relationship was observed. The muscle acted as we would expect if the sliding filament theory was true. Based on the sliding filament theory (model), and based on the length - tension curve (behavior predicted by the model), there are ideal positions from which to test a muscles ability to generate active tension (force). So this model of muscle, sliding filaments and length - tension, allows us to use our knowledge of muscle in practical situations. It is a model of muscle that applies to all striated muscle (even as we will see cardiac muscle, which is striated but still differentiated from skeletal muscle).

This book is full of models that summarize, test and utilize our knowledge of muscles and the physiological systems that support them. One thing we need to remember about models is that they are abstractions of reality, not reality. "All models are wrong, but some models are useful" (attributed to George Box). This is an important consideration. Compared to reality all models are limited. We must ask whether the model we have is useful - useful to summarize what we know, useful to test what we know, or useful to utilize what we know. 

\section{Core Physiology Concepts}

\subsection{Causality}
Living organisms have causal mechanisms whose functions are explainable by a description of the cause-effect relationship that are present. Accepting that these functions are explainable is a philosophical assumption, just because many cause-effect relationships are explainable, we still must deal with what is called the problem of induction.

\subsection{Cells}
Cell theory states that all cells making up an organism have the same DNA. Cells have many common functions but also many specialized functions that are required for the organism (differentiation). Of course there are all sorts of questions to be raised about what cells make up an organism given the reliance we have on the symbiotic relationship with our gut microbiome. Cells have a plasma membrane that is a complex structure that determines what substances enter or leave the cell. They are essential for cell signaling, transport and other processes. The function of an organism requires that cells pass information to one another through "cell-cell communication" to coordinate their activities. These communication processes include endocrine (hormones delivered system wide in the blood) and neural signaling. 

\subsection{Genes to proteins}
The genes (DNA) of every organism code for the synthesis of proteins (enzymes are proteins). The genes that are expressed determine the functions of every cell.

\subsection{Energy}
The life of the organism requires the constant expenditure of energy. The acquisition, transformation, and transportation of energy are essential functions of the body.

Including conservation of energy.

\subsection{Mass Balance}
The quantity of ”stuff” in any system, or in a compartment in a system is determined by the inputs into the system and the outputs from that system or compartment.

Including conservation of mass.

\subsection{Flow Down Gradients}
The transport of “stuff” (ions, molecules, blood, and gas) is a central process at all levels of organization in the organism, and a simple model (derivatives of Ohm’s Law) describes such transport.

\subsection{Homeostasis}
The internal environment of the organism is actively maintained constant by the feedback function of cells, tissues, and organs organized into primarily negative feedback systems.\\
Important points about homeostasis:
\begin{itemize}
    \item Not everything is regulated (i.e. heart rate)
    \item Homeostasis is not an on/off switch
    \item ”Relatively constant” – meaning there can be acceptable variation
    \item Set points can change
    \item There is a hierarchy of homeostatic regulation
\end{itemize}

Relate to supported supporting and integrity

\subsection{Interdependence}
Cells, tissues, organs, organ systems interact with one another (are dependent on the function of one another) to sustain life.

\subsection{Structure - Function}
The function of a cell, tissue, or organ is determined by its form. Structure and function (from the molecular level to the organ system level) are intrinsically related to each other. 
And what do we add as an important consideration?


\section{Unifying Systems Concepts as applied to a muscle centered approach}

% I'm struggling with this entire section. I'm not sure whether this is of any value to the course as I'm teaching it and wonder whether I should pull back on these topics for now until we have them worked out on our own.

Unifying Systems Theory (UST) concepts are, mostly implicitly, utilized throughout this book. The reader should not attempt to commit these to memory at this time. The concepts are discussed throughout the book when there are important or useful connections to be emphasized. The emphasis is highlighting any "so what" contributions for understanding clinical physiology from a muscle centered approach. It is perfectly reasonable that someone would learn the material from this book with no explicit discussion about UST, but implicitly the concepts are always being utilized.

\subsection{Muscle as a System}
We consider a muscle cell (fiber) a system. A system is whole and it acts (works towards a common objective). A system has agency (ability to act to satisfy values). 

\subsection{Acts}
The act of muscle is tensioning (act of creating tension). A muscle fiber (as a system) acts by creating tension. As a system that acts by creating tension muscles are correct, complete and concise. A muscle fiber correctly acts to create tension, it completely has what it needs for that act), and it is concise in creating tension. As a system a muscle fiber is supported in order to satisfy its act. A large part of clinical physiology for a physical therapist taking a muscle centered approach is considering Muscle Support (Part II of the book).

\subsection{Capabilities \& Parts}

Parts manifest as capabilities and capabilities are applied to perform acts. A muscle cell has parts that provide its capabilities, and those capabilities enable the muscle cell to create tension.

\subsection{Quality Attributes}

Quality attributes consider "How well" acts are achieved. Qualities define and consider what is important for the acts to accomplish. What does the system want to achieve? What are the success qualities of the act. There are three broad quality attributes for the muscle act of tensioning. Muscle fidelity, efficacy and integrity. Each of these quality attributes can also be considered based on possible metrics that let us consider how well that quality is being met. Table \ref{table:1} defines each quality attribute of muscle and provides some possible metrics.

\begin{table}[h!]
\centering
\begin{tabular}{||c c c c||} 
 \hline
 Quality & Define & Metrics & Measurement \\ [0.5ex] 
 \hline\hline
 Fidelity & Application & Attain/Sustain & Force \\ 
 Efficacy & Transformation & Sustain/Persevere & Endurance \\
 Integrity & Generation & Persevere/Maintain & Cross sectional area \\[1ex] 
 \hline
\end{tabular}
\caption{Muscle Quality Attributes}
\label{table:1}
\end{table}

\subsection{Fidelity}
Fidelity focuses on the application of tension. The parts and capabilities of muscle that allow it apply tension when needed, that is in the right contexts. Metrics of fidelity can include any metrics that assess how well tension is attained, and once attained how well it is sustained. A measurement of attaining and sustaining tension may be a measure of force being applied and for how long it can be applied.

\subsection{Efficacy}
Efficacy focuses on the transformation of tension. That is - how well does muscle transform tension to its intended use, typically in interaction with something outside of or beyond muscle itself. Metrics of efficacy relate to how effectively (including efficiently) muscle fulfils its purpose and includes whether the muscle can sustain (continue to generate a tension) and persevere (continue to repeat the process of sustaining tension). An example of measurements are any that consider time along with force, commonly referred to as endurance. 

\subsection{Integrity}
Integrity focuses on muscle being muscle, capable of generating tension. As a system muscle fibers exist to execute the act of generating tension so integrity is about the parts and capabilities that make a muscle figure a muscle fiber. Metrics include the ability to continue to do what a muscle cell does (persevere) which occurs through maintaining the parts and capabilities. A measurement of integrity could be the cross section area of a muscle, or the lean mass of muscle, as indicators of how well the muscle (as a muscle collection of muscle fibers) is being maintained.

\subsection{Fidelity - Efficacy - Integrity as a Spectrum}
Note that there is overlap between the metrics of fidelity, efficacy and integrity in a three dimensional interactive spectrum making it difficult to have metrics and measurements that are completely unique. For example, a measure of muscle force is an example of fidelity of the muscle to generate active tension, efficacy in transmitting that tension to the bones and the force recording device, and none of that can take place without integrity (the muscle is the muscle and has the necessary parts and capabilities).

\section{Summary \& Next Steps}

Part I of the book is focused on Muscle Fidelity \& Efficacy. We cover what it takes for muscle to attain, sustain and persevere in the act of tensioning. Part II of the book is focused on Muscle Support, asking whether the muscle cell has what it needs for fidelity and efficacy or whether it needs support (hint, it does need support). Part III of the book focuses on Muscle Integrity, how does muscle persevere and maintain under steady state and variable conditions and even in the face of threats to integrity (damage, injury and illness).  